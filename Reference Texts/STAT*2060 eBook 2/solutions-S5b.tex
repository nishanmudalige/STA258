\begin{enumerate}

\item Determine which of the following sets are subspaces of
$\mbox{\tebbb R}^3$.

(a) $(x,y,z)$, where $z=\frac{1}{2}$ \quad \quad (b) $(x,y,z)$,
where $y=z-x$

(c) $(x,y,z)$, where $x=y=z$.

\noindent \textbf{Solution}

\begin{enumerate} \item[(+)]

\noindent To check to see if the sets are subspaces of
$\mbox{\tebbb R}^3$ we must show:

($1$)The set is non-empty, i.e., ${\bf 0}$ is in the set.

($2$) For all ${\bf u}$ and ${\bf v}$ in the set, ${\bf u} + {\bf
v}$ is in the set (closed under addition).

($3$) For all $k\in$ {\tebbb R} and for all ${\bf u}$ in the set,
$k{\bf u}$ is in the set (closed under multiplication).

\end{enumerate}

\noindent \textbf{Solution} Part (a)

\noindent Let $W=\{(x,y,z),where\
z=\frac{1}{2}\}=\{(x,y,\frac{1}{2}); x,y\in \mbox{\tebbb R}\}$

\noindent PROOF OF ($1$)

\noindent (*) By inspection, we already see that $W$ is not a
subspace of $\mbox{\tebbb R}^3$ (i.e., $(0,0,0)$ is not in $W$).

\smallskip

\noindent \textbf{Solution} Part (b)

\noindent Let $W=\{(x,y,z), where\ y=z-x\}=\{(x,z-x,z); x,z\in
\mbox{\tebbb R}\}$.

\noindent PROOF OF ($1$)

\noindent (*) We know $W$ is non-empty by inspection ( $(0,0,0)$
is a vector in $W$).


\noindent PROOF OF ($2$)

\noindent Let ${\bf u}=(u_{1},u_{2}-u_1,u_{2})$ and ${\bf
v}=(v_{1},v_{2}-v_1,v_{2})$, where ${\bf u},{\bf v}\in W$.

\begin{enumerate} \item[${\bf u}+{\bf v}$]
$=(u_{1},u_{2}-u_1,u_{2})+(v_{1},v_{2}-v_1,v_{2})$

$=(u_{1}+v_{1}, (u_{2}+v_{2})-(u_1+v_1),u_{2}+v_{2})$

$(u_{1}+v_{1}, (u_{2}+v_{2})-(u_1+v_1),u_{2}+v_{2})\Rightarrow\ W$
is closed under addition. (**)

\end{enumerate}

\noindent PROOF ($3$)

\noindent Let $k\in\mbox{\tebbb R}$ and ${\bf
u}=(u_{1},u_{2}-u_1,u_{2})\in W$.

\begin{enumerate} \item[$k{\bf u}$]

$=k(u_{1},u_{2}-u_1,u_{2})$

$=(ku_{1},ku_{2}-ku_1,ku_{2})$

$(ku_{1},ku_{2}-ku_1,ku_{2})\in W \Rightarrow W$ is closed under
multiplication. (***)

\end{enumerate}

\noindent Since (*),(**),(***) are true $\Rightarrow W$ is a
subspace of $\mbox{\tebbb R}^3$.

\smallskip

\noindent \textbf{Solution} Part (c)

\noindent Let $W=\{(x,y,z), where\ x=y=z\}=\{(x,x,x);
x\in\mbox{\tebbb R}\}$.

\noindent PROOF ($1$)

\noindent (*)We know $W$ is non-empty by inspection ($(0,0,0)$ is
in $V$).

\noindent PROOF ($2$)

\noindent Let ${\bf u}=(u_{1},u_{1},u_{1})$, ${\bf
v}=(v_{1},v_{1},v_{1})$, where ${\bf u},{\bf v}\in W$.

\begin{enumerate} \item[${\bf u}+{\bf v}$]
$=(u_{1},u_{1},u_{1})+(v_{1},v_{1},v_{1})$

$=(u_{1}+v_{1},u_{1}+v_{1},u_{1}+v_{1})$

$(u_{1}+v_{1},u_{1}+v_{1},u_{1}+v_{1}) \Rightarrow\ W$ is closed
under addition. (**)

\end{enumerate}

\noindent PROOF ($3$)

\noindent Let $k\in\mbox{\tebbb R}$ and ${\bf
u}=(u_{1},u_{1},u_{1})\in W$.

\begin{enumerate} \item[$k{\bf u}$]

$=k(u_{1},u_{1},u_{1})$

$=(ku_{1},ku_{1},ku_{1} )$

$(ku_{1},ku_{1},ku_{1} )\Rightarrow W$ is closed under
multiplication. (***)

\end{enumerate}

\noindent Since (*),(**),(***) are true $\Rightarrow W$ is a
subspace of $\mbox{\tebbb R}^3$.

\bigskip

\item Determine which of the following sets are subspaces of
$\mbox{\tebbb R}^4$.
\begin{enumerate}
\item $(x_1,x_2,x_3,x_4)$, where  $x_1\leq x_2\leq x_3 \leq x_4$
\item $(x_1,x_2,x_3,x_4)$, where $x_2+x_4=1$
\item $(x_1,x_2,x_3,x_4)$, where $|x_1| = |x_4|$.
\end{enumerate}

\bigskip

\noindent \textbf{Solution}

\noindent Recall (+) from question \#$1$.  The problems are solved
in the same way as the previous question.
\begin{description}\item (a)
Using the same approach as in part (a) we determine that
the subset is non-empty and that it is closed under addition but
not under scalar multiplication. Since one of the criteria has
failed, we can say that the subset is not a subspace.
\item (b)
Using the same approach as in part (a) we determine that
the subset does not contain ${\bf 0}$, and it is not closed under
addition or multiplication. Since all the criteria have not been
fulfilled, we can say that the subset is not a subspace.
\item (c)
Using the same approach as in part (a) we determine that
the subset is non-empty and that it is closed under multiplication
but not under addition. Since all the criteria have not been
fulfilled we can say that the subset is not a subspace.
\end{description}

\item Let ${\bf u}=(0,2,-2,1)$, ${\bf v}=(1,-3,4,0)$ and ${\bf
x}=(1,-1,0,-1)$. Express

(a) ${\bf w}=(4,2,-8,-3)$ \quad \quad \quad (b) ${\bf
w}=(1,-7,14,4)$

\noindent as a linear combination of {\bf u}, {\bf v} and {\bf x}.

\noindent \textbf{Solution} \begin{description}\item (a)

$\begin{array}{rl}{\bf w}=(4,2,-8,-3)&=a{\bf u}+b{\bf v}+c{\bf
x}\\ &=a(0,2,-2,1)+b(1,-3,4,0)+c(1,-1,0,-1)\\
&=(b+c,2a-3b-c,-2a+4b,a-c)\end{array}$

From this we get the following system:

$b+c=4;\ 2a-3b-c=2;\ -2a+4b=-8;\ a-c=-3$

Solving the system we get $a=2,\ b=-1,\ c=5$.  Therefore, ${\bf
w}=2{\bf u}-{\bf v}+5{\bf x}$.

\item (b)
Using the same system (just change ${\bf w}$), we find that ${\bf
w}={\bf u}+4{\bf v}-3{\bf x}$.
\end{description}
\item \begin{enumerate}
\item For which value of $b$ will the vector ${\bf w}=(3,b,4)$ be a
linear combination of ${\bf u}=(6,5,-2)$ and ${\bf v}=(7,1,-4)$.
\item For which value of $a$ will the vector ${\bf w}=(2,a,-1,7)$ be a
linear combination of ${\bf u}=(5,-1,6,-3)$, ${\bf v}=(12,6,5,3)$
and ${\bf x}=(1,3,-2,4)$.
\end{enumerate}

\noindent \textbf{Solution} \begin{description}\item (a)

$\begin{array}{rl}{\bf w}=(3,b,4)&=m{\bf u}+n{\bf v}\\
&=m(6,5,-2)+n(7,1,-4)\\ &=(6m+7n,5m+n,-2m-4n)\end{array}$

From this we get the following system:

$6m+7n=3;\ 5m+n=b;\ -2m-4n=4$

Solving the system we get $m=4$ and $n=-3$. Then $b=17$.\\
\item (b)
\begin{align*}{\bf w}=(2,a,-1,7)&=p{\bf u}+q{\bf v}+r{\bf x}\\
&=p(5,-1,6,-3)+q(12,6,5,3)+r(1,3,-2,4)\\
&=(5p+12q+r,-p+6q+3r,6p+5q-2r,-3p+3q+4r)\end{align*}

From this we get the following system:

$5p+12q+r=2;\ -p+6q+3r=a;\ 6p+5q-2r=-1;\ -3p+3q+4r=7$

Solving the system formed with just the $p,q,r$ we get $p=2,\
q=-1$ and $r=4$. Then sub into equation with the $a$ to get $a=4$
for ${\bf w}$ to be a linear combination of ${\bf u},\ {\bf v}$,
and ${\bf x}$.
\end{description}
\item Determine which of the following sets of vectors are
linearly independent.
\begin{enumerate}
\item $\mbox{\tebbb R}^2: \quad \{(1,3),\ (3,1)\}$
\item $\mbox{\tebbb R}^3: \quad \{(2,-2,0),\  (2,0,-1),\ (-1,7,-1),\ (8,-9,0)\}$
\item $\mbox{\tebbb R}^3: \quad \{(1,1,3),\ (-4,1,5),\ (6,1,1)\}$
\item $\mbox{\tebbb R}^4: \quad \{(0,3,-4,-8),\ (3,-1,2,-7),\ (3,-4,6,1)\}$
\item $\mbox{\tebbb R}^4: \quad \{ (-2,2,-3,5),\ (2,-1,4,2),\ (1,0,-5,0),\ (2,1,0,-1) \}$
\item $\mbox{\tebbb R}^4: \quad \{ (3,1,0,2),\ (1,-5,1,3),\ (3,2,-1,0),\ (1,-6,2,5)\}$
\item $\mbox{\tebbb R}^5: \quad \{(0,5,1,2,3),\ (0,0,0,7,9),\ (0,0,7,2,3), \ (3,7,1,1,9),\ (0,0,0,0,8) \}$
\end{enumerate}

\noindent \textbf{Solution} \begin{description}\item (a)
Let $c_1(1,3)+c_2(3,1)=(0,0)$. Then $(c_1, c_2)$ is the
solution to the system with augmented matrix $$ \left[
\begin{array}{rrcr} 1 & 3 & \vline & 0 \\  3 & 1 & \vline & 0
\end{array} \right],~or~just~\left[ \begin{array}{rr} 1 & 3 \\
3 & 1 \end{array} \right].$$ This reduces to $\left[ \begin{array}{rr} 1 & 0 \\
0 & 1 \end{array} \right]$ so the only solution is $c_1=c_2=0$.
Hence the vectors are independent. One could also observe that
$$\left| \begin{array}{rr} 1 & 3 \\ 3 & 1 \end{array} \right|=1-9=-8 \neq
0.$$ This implies that the matrix reduces to $\left[ \begin{array}{rr} 1 & 0 \\
0 & 1 \end{array} \right]$ so the vectors are independent.
\item (b)
Since there are four vectors in $\mbox{\tebbb R}^3$, the
set of vectors is linearly dependent.
\item (c)
Let $c_1(1,1,3)+c_2(-4,1,5)+c_3(6,1,1)=(0,0,0)$. Then
$(c_1,c_2,c_3)$ is the solution to the solution to the system with
augmented matrix $$\left[ \begin{array}{rrrcr} 1 & -4 & 6 & \vline
& 0 \\ 1 & 1 & 1 & \vline & 0 \\ 3 & 5 & 1 & \vline & 0
\end{array} \right],~or~just~\left[ \begin{array}{rrr}
1&-4&6 \\ 1&1&1 \\ 3&5&1 \end{array} \right] {\leadsto} \left[
\begin{array}{rrr} 1&0&2 \\ 0&1&-1 \\ 0&0&0 \end{array} \right].$$
Since there is a free a variable the solution is not unique and so
the vectors are dependent. You could also observe that the
determinant of the matrix in $0$ and this means there is a
non-unique solution.
\item (d)
Following (a) and (c) we reduce the matrix $$\left[
\begin{array}{rrr} 0&3&3 \\ 3&-1&-4 \\ -4&2&6 \\ -8&-7&1 \end{array} \right]
{\leadsto} \left[ \begin{array}{rrr} 1&-\frac{1}{2}&-\frac{3}{2} \\ 0&1&1 \\
0&0&0 \\ 0&0&0 \end{array} \right].$$ Since there is a free
variable the solution is not unique and the vectors are dependent.
Notice that you cannot use the determinant here, because the
matrix is not square.
\item (e)
Set up the augmented matrix with the vectors as its
columns. Since the matrix in reduced-row echelon form is the
identity matrix (i.e., the matrix is row equivalent to $I$), the
set of vectors is linearly independent. Alternatively, the
determinant $\neq 0$.
\item (f)
Again, set up the augmented matrix with the vectors as
its columns. Since the row-echelon form of the matrix has a free
variable, the set of vectors is linearly dependent. Alternatively,
the determinant $=0$.
\item (g)
Again, set up the augmented matrix with the vectors as
its columns. Notice that we may interchange several rows to
produce a matrix in echelon form. The matrix reduces to the
identity matrix, so the set of vectors is linearly independent.
Alternatively, the determinant of the matrix $\neq 0$.
\end{description}

\item Determine if the following vectors span $V$.
\begin{enumerate}
\item $V=\mbox{\tebbb R}^3$; \{ (-2,-2,1),\ (-1,2,2),\ (-3,0,1)\}
\item $V=\mbox{\tebbb R}^3$; \{ (4,0,1),\ (1,2,-4), \ (-2,3,1)\}
\item $V=\mbox{\tebbb R}^4$; \{ (5,1,4,0),\ (-1,-7,3,2),\ (3,1,3,0)\}
\item $V=\mbox{\tebbb R}^4$; \{ (1,4,2,1),\ (4,0,0,2),\ (-2,1,0,-1),\ (3,3,-2,0),\ (1,0,1,-1)\}
\end{enumerate}

\noindent \textbf{Solution} \begin{description}\item (a)
We must show that for all $\omega_1$, $\omega_2$,
$\omega_3$, $\exists$ $a,b,c$ such that

$(w_{1}, w_{2},w_{3})= a(-2,-2,1)+b(-1,2,2)+c(-3,0,1) $

$~~~~~~~~~~~~~~~=(-2a-b-3c,-2a+2b,a+2b+c) $

$\Leftrightarrow  \begin{array}{rcl}
                -2a-b-3c&=&w_1\\
                -2a+2b&=&w_2\\
                a+2b+c&=&w_3 \end{array}$

\noindent Written in matrix form:

$ \left [ \begin{array}{rrr}
                -2&-1&-3\\
                -2&2&0\\
                1&2&1 \end{array} \right]$
$\left [ \begin{array}{r}
                a\\
                b\\
                c \end{array} \right]$ \quad $=$ \quad
$\left [ \begin{array}{r}
                w_{1}\\
                w_{2}\\
                w_{3} \end{array} \right]$

\noindent We must now determine whether this system is consistent.
We know that there is a solution for all $w_{1}$, $w_{2}$, $w_{3}$
if and only if the coefficient matrix is invertible (if the
determinant is non-zero). In this case, the determinant is 12;
therefore, $v_{1}$, $v_{2}$ and $v_{3}$ span $\mbox{\tebbb
R}^3$.
\item (b)
Use the same method as in (a), except the vectors are
$v_1=(4,0,1),\ v_2=(1,2,-4), \ v_3=(-2,3,1)$. OR, even easier,
just form a matrix with the vectors as the columns and find the
determinant.

\noindent In this case, the determinant is $63$, therefore
$v_{1}$, $v_{2}$ and $v_{3}$ span $\mbox{\tebbb R}^3$.
\item (c)
There must be at least four vectors to span
$\mbox{\tebbb R}^4$.  Convince yourself.
\item (d)
We must determine whether, $\forall$ $\omega_1,
\omega_2, \omega_3, \omega_4$, $\exists$ $a,b,c,d,e$ such that \\$(
\omega_1, \omega_2, \omega_3, \omega_4 )=a(1,4,2,1)+ b(4,0,0,2)+
c(-2,1,0,-1)+ d(3,3,-2,0)+ e(1,0,1,-1),$ i.e. whether $$ \left[
\begin{array}{rrrrrcr} 1&4&-2&3&1 & \vline & \omega_1 \\ 4&0&1&3&0
&\vline & \omega_2 \\ 2&0&0&-2&1 & \vline & \omega_3 \\
1&2&-1&0&-1 & \vline & \omega_4 \end{array}
\right]~is~consistent~\forall~\omega_1,~\omega_2,~\omega_3,~\omega_4.$$
The right hand portion of the matrix reduces to

$ \left[
\begin{array}{rrrrr} 1&0&0&-1&\frac{1}{2} \\
0&1&-\frac{1}{2}&\frac{1}{2}&-\frac{3}{4} \\ 0&0&1&7&-2 \\
0&0&0&1&\frac{7}{4} \end{array} \right]$. Hence the system is
consistent and the 5 vectors span $\mbox {\tebbb R}^4$.
\end{description}

\item Determine which sets of vectors are bases.
\begin{enumerate}
\item $\{ (-1,4,1),\ (4,1,1), \ (2,0,1) \}$ in $\mbox {\tebbb
R}^3$
\item $\{(2,4,1,0), \ (2,1,1,0), \ (3,-2,-1,3), \
(0,0,1,-2)\}$ in $\mbox {\tebbb R}^4$
\item $\{ (3,2,1,0,0), \ (-1,2,-2,-2,5),\ (4,9,4,0,9),\ (2,0,1,-1,0),\\$
$(0,5,4,3,4)\} $ in $\mbox {\tebbb R}^5$
\end{enumerate}

\noindent \textbf{Solution} \begin{description}\item (a)
Let $S=\{ (-1,4,1),\ (4,1,1), \ (2,0,1) \}$.

\noindent Recall that a set $S$ is a basis for a vector space $V$
if the vectors in $S$ span $V$ and are linearly independent. BOTH
of these are proven if we show that the matrix formed with the
$n-tuple$ vectors in $S$ as its columns reduces to the $n\times n$
identity matrix, i.e., the determinant of the matrix is not zero.

$ \left [ \begin{array}{rrr} -1&4&2 \\ 4&1&0 \\ 1&1&1 \end{array}
\right]\ \leadsto\ \left [ \begin{array}{rrr} 1&0&0 \\ 0&1&0 \\
0&0&1 \end{array} \right]$

\noindent Since the matrix reduces to the identity matrix, we know
that the vectors are a basis for lin($S$).
\item (b)

Let $S=\{(2,4,1,0), \ (2,1,1,0), \ (3,-2,-1,3), \
(0,0,1,-2)\}$

\noindent Forming the matrix with these vectors as its columns, we
find that its determinant is $-12$.  Therefore, the vectors form a
basis for lin($S$).
\item (c)
Let $S=\{ (3,2,1,0,0), \ (-1,2,-2,-2,5),\ (4,9,4,0,9),\
(2,0,1,-1,0),\\$ $(0,5,4,3,4)\}$

\noindent Forming the matrix with these vectors as its columns, we
find that its determinant is $0$. Therefore, the vectors do not
form a basis for lin($S$).
\end{description}

\item For the following sets of vectors:

(i) $S=\{ (2,1,0), \ (-2,3,-4), \ (1,4,0) \}$

(ii) $S=\{ (1,0,2),\ (2,-1,1),\ (0,2,1)\} $

(iii) $S=\{ (1,-1,0),\ (0,2,0),\ (-1,1,2)\} $

\begin{enumerate}
\item Verify that $S$ is a basis for $\mbox{\tebbb R}^3$.
\item Find the coordinates of ${\bf u}=(3,1,1)$ relative to the basis $S$.
\end{enumerate}

\noindent \textbf{Solution} \begin{description}\item (i) (a)

\noindent For $S$ to be a basis :

\noindent ($1$) $S$ has to be linearly independent,

\noindent ($2$) $S$ has to span $\mbox{\tebbb R}^3$.

\noindent We can verify both of these conditions by forming the
matrix with the vectors in $S$ as its columns.  The vectors are a
basis for $\mbox{\tebbb R}^3$ if this matrix reduces to the
$3\times 3$ identity matrix (it is often easier to verify that the
determinant of the $3\times 3$ matrix is not zero).

For $S=\{ (2,1,0), \ (-2,3,-4), \ (1,4,0) \}$, the matrix formed
is $$\left[\begin{array}{rrr} 2&-2&1 \\ 1&3&4 \\ 0&-4&0
\end{array} \right].$$ Since the determinant is $28$, $S$ is a
basis of $\mbox{\tebbb R}^3$.
\item (i) (b)

${\bf u}=a(2,1,0)+b(-2,3,-4)+c(1,4,0)$

$(3,1,1)=(2a-2b+c,a+3b+4c,-4b)$

\noindent The augmented matrix form:

$ \left [ \begin{array}{rrr|r}
                2&-2&1&3\\
                1&3&4&1\\
                0&-4&0&1 \end{array} \right]$ $\leadsto$
$ \left [ \begin{array}{rrr|r}
                1&0&0&\frac{33}{28}\\
                0&1&0&-\frac{1}{4}\\
                0&0&1&\frac{1}{7} \end{array} \right]$

\noindent Thus, $(u)_{s}=(\frac{33}{28},-\frac{1}{4},\frac{1}{7})$
\item (ii) (a)
Use the same method as Part (i) (a) with $S=\{ (1,0,2),\
(2,-1,1),\ (0,2,1)\} $.  Constructing the matrix, we find that the
determinant is 5.  Since the determinant is not zero, $S$ is a
basis of $\mbox{\tebbb R}^3$.
\item (ii) (b)
Using the same method as Part (i) (b), we find that
$(u)_{s}=(-\frac{7}{5},\frac{11}{5},\frac{8}{5})$
\item (iii) (a)
Use the same method as Part (i) (a) with $S=\{
(1,-1,0),\ (0,2,0),\ (-1,1,2)\} $.  Constructing the matrix, we
find that the determinant is $4$.  Since the determinant is not
zero, $S$ is a basis of $\mbox{\tebbb R}^3$.
\item (iii) (b)
Using the same method as Part (i) (b), we find that
$(u)_{s}=(\frac{7}{2},2,\frac{1}{2})$
\end{description}
\item For the following sets of vectors:

(i) $S=\{ (1,3,2,0), \ (0,-1,0,1),\ (4,2,0,1), \ (-1,0,2,3)\}$\\
(ii) $S=\{ (2,0,1,1),\ (-1,3,2,0),\ (0,0,1,-3),\ (2,0,-1,1)\}$
\begin{enumerate}
\item Verify that $S$ is a basis for $\mbox{\tebbb R}^4$.

\item Find the coordinates of ${\bf u}=(2,-1,0,1)$ relative to the basis $S$.
\end{enumerate}
\noindent \textbf{Solution} \begin{description} \item (i)(a) Use the same method as Part (i) (a) with $S=\{
(1,3,2,0), \ (0,-1,0,1),$
$(4,2,0,1),\ (-1,0,2,3)\}$. Constructing the matrix, we find that the
determinant is $12$. Since the determinant is not zero, $S$ is a
basis of $\mbox{\tebbb R}^4$.
\item (i) (b) Using the same method as Part (i) (b), we find that
$(u)_{s}=(1,4,0,-1)$
\item (ii) (a) Use the same method as Part (i) (a) with $S=\{
(2,0,1,1),\ (-1,3,2,0),$
 $(0,0,1,-3),\ (2,0,-1,1)\}$. Constructing the matrix, we find that the
determinant is $-36$. Since the determinant is not zero, $S$ is a
basis of $\mbox{\tebbb R}^4$.
\item (i) (b) Using the same method as Part (i) (b), we find that
$(u)_{s}=(\frac{7}{9},-\frac{1}{3},-\frac{1}{18},\frac{1}{18})$.
\end{description}
\end{enumerate}
