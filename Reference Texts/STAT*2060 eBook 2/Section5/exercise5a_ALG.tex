\iffalse
\documentclass[12pt]{article}
\usepackage{amsmath}
\usepackage{latexsym}

\addtolength{\textwidth}{1in} \addtolength{\oddsidemargin}{-0.5in}
\addtolength{\textheight}{1.6in} \addtolength{\topmargin}{-0.8in}

\newfont{\tebbb}{msbm10 scaled\magstep1}

\newtheorem{theorem}{Theorem}[section]
\newtheorem{proposition}[theorem]{Proposition}
\newtheorem{lemma}[theorem]{Lemma}
\newtheorem{corollary}[theorem]{Corollary}
\newtheorem{remark}[theorem]{Remark}
\newtheorem{example}[theorem]{Example}
\newcommand{\beq}{\begin{equation}}
\newcommand{\eeq}{\end{equation}}
\newtheorem{definition}[theorem]{Definition}


\newcommand{\cross}[2]{{{\bf{#1}} \times {\bf{#2}}}}
\newcommand{\dotprod}[2]{{{\bf{#1}} \cdot {\bf{#2}}}}
\newcommand{\real}[1]{{\mbox{\tebbb R}}^{#1}}
\newcommand{\norm}[1]{\|{\bf{#1}}\|}
\renewcommand{\theequation}{\thesection.\arabic{equation}}

\baselineskip = 20pt plus 3pt minus 3pt

%\begin{document}
\fi

\section{Suggested Exercises}\label{ssec.se7}
\markright{\ref{ssec.se7}
\titleref{ssec.se7}}
\begin{enumerate}
\item For the following problems let
$${\bf u}=(2,2,3) \quad {\bf v}=(1,7,-4) \quad {\bf w}=(3,0,4).$$
Find
\begin{enumerate}
\item (i) ${\bf u}+{\bf w}$ \quad (ii) $2{\bf w}$ \quad (iii) $4{\bf v}$
\item (i) ${\bf u}+3{\bf w}$ \quad (ii) ${\bf v}-{\bf w}$ \quad (iii) $2({\bf w}+{\bf u})$
\item (i) ${\bf u} \cdot {\bf w}$ \quad (ii) ${\bf u} \cdot {\bf v}$
\quad (iii) ${\bf u} \cdot {\bf w}+ {\bf u}\cdot {\bf v}$ \quad (iv) ${\bf u}\cdot ({\bf v}+{\bf w})$
\item  $\|{\bf w} \|$
\item the angle type and value between ${\bf u}$ and ${\bf v}$
\item the distance between ${\bf u}$ and ${\bf w}$.
\end{enumerate}

\item For the following problems let
$${\bf x}=(1,2,3,4,5) \quad {\bf y}=(2,6,7,9,3).$$ Find
\begin{enumerate}
\item (i) ${\bf x}+ {\bf y}$ \quad (ii)${\bf y} - 2{\bf x}$
\item (i) $\|{\bf x}\|$ \quad (ii) $\|{\bf y}\|$ \quad (iii) $\|{\bf
x}\|\|{\bf y}\|$ \quad (iv) ${\bf x} \cdot {\bf y}$
\item  Verify the Cauchy-Schwarz inequality for {\bf x} and {\bf y}.
\end{enumerate}
\item Find $x$, $y$, and $z$ such that $x(1,2,3)=(2,y,z)$
\item Find the equation of the line through the point $P_0$ with normal vector ${\bf{u}}$.
\begin{enumerate}
\item $P_0 = (1,2)$, ${\bf{u}} = (4,\,-1)$.
\item $P_0 = (0,2)$, ${\bf{u}} = (1,\,2)$.
\item $P_0 = (4,1)$, ${\bf{u}} = (-3,\,1)$.
\end{enumerate}
\item Find the equation of the plane through the point $P_0$ with normal vector ${\bf {u}}$.
\begin{enumerate}
\item $P_0 = (0,0,0)$, ${\bf{u}} = (-1,\,-1,\,3)$.
\item $P_0 = (1,2,3)$, ${\bf{u}} = (0,1,-1)$.
\item $P_0 = (-1,2,7)$, ${\bf{u}} = (2,0,3)$.
\end{enumerate}
\item Find the equation of the plane through the points
\begin{enumerate}
\item $(1,1,2)$, $(2,-1, 3)$, $(-1,2,2)$.
\item $(-3,0,1)$, $(0,2,3)$, $(1,-1,2)$.
\end{enumerate}
\item The general form of a parabola (with a certain orientation) is
\begin{align*}
ay^2 + bx + cy + d = 0
\end{align*}
Using the determinant find the equation of the parabola through the points
\begin{enumerate}
\item $(1,2)$, $(2,0)$, $(5,6)$
\item $(0,2)$, $(-1,1)$, $(4,0)$
\end{enumerate}
\item A sphere (in three dimensions) has equation of the form
$$
a(x^2 + y^2 + z^2) + bx + cy + dz + e = 0.
$$
Using the determinant find the equation of the sphere through the points
$(1,1,0)$, $(1,0,0)$, $(0,0,1)$ and $(0,0,0)$.
\item For the vectors ${\bf {u}}$ and ${\bf {v}}$ given below
\begin{itemize}
\item[(i)] find the projection of ${\bf {u}}$ on ${\bf {v}}$,
\item[(ii)] find the projection of ${\bf {v}}$ on ${\bf{u}}$,
\item[(iii)] use the above projection to find a non--zero vector orthogonal to ${\bf {u}}$
\end{itemize}
\begin{enumerate}
\item ${\bf {u}} = (3, -1, 7)$, $v = (-2, 4, 0)$.
\item ${\bf {u}} = (-2, -1, 4)$, $v = (3, -2, 2)$.
\end{enumerate}




\end{enumerate}

\section{Answers to activity questions and suggested exercises}
\label{answers5.a}
\markright{\ref{answers5.a}
\titleref{answers5.a}}

{\bf Activity questions}

\bigskip

{\bf \ref{ssec.covector}:}
\begin{enumerate}
\item \begin{enumerate}
\item [(b)] $(-3,3)$
\item [(c)] $(12,-3)$
\item [(d)] $(18,-6)$.
\end{enumerate}
\item \begin{enumerate}
\item $\| {\bf u} \|=\sqrt{5}$, $\| {\bf v} \|=\sqrt{6}$ and $\| {\bf w} \|=\sqrt{10}$
\item ${\bf v}-{\bf u}=(-1,-3,1)$,$\| {\bf v}-{\bf u} \|=\sqrt{11}$
\item $-\frac{1}{\sqrt{15}}$.
\end{enumerate}
\end{enumerate}

\bigskip

{\bf \ref{ssec.LPequation}:}
\begin{enumerate}
\item \begin{enumerate} \item (3,\,1)
\item $\overrightarrow{QR\ }= (3,\,-8) - (-1,\,4) = (4,\,12)$ then $(3,1)\cdot(4,\,-12) = 12-12 = 0$
\end{enumerate}
\item
\begin{align*}
\bigl((x-1),(y-1)\bigr)\cdot(0,1) &= 0\\
0(x-1) + 1(y-1) &=0\\
y&= 1
\end{align*}
\item \begin{enumerate} \item $(1,1,1)$ \item
$\overrightarrow{QR\ } = (-1, 0, 4) - (2,1,0) = (-3,\,-1,4)$ \\ $(1,1,1)\cdot(-3,\,-1,4) = -3 -1+4 = 0$
\end{enumerate}
\item
\begin{align*}
(x-1, y-1, z-2)\cdot(1,0,1) &=0 \\
(x-1) + 0(y-1) + (z-2) &=0\\
x + z &= 3
\end{align*}
\end{enumerate}
\bigskip
{\bf \ref{ssec.Danal}:}
\begin{enumerate}
\item
\begin{align*}
\left| \begin{array}{cccc} y & x^2 & x & 1 \\ -5 & 1 & -1 & 1 \\ 1
& 1 & 1 & 1 \\ 10 & 4 & 2& 1 \end{array} \right| &\begin{array}{l}
{\leadsto}\ {R1-R4} \\ {\leadsto}\ {R2-R4} \\ {\leadsto}\ {R3-R4}
\\ \leadsto R4 \end{array} \left| \begin{array}{cccc} y-10 & x^2-4
& x-2 & 0 \\ -15 & -3 & -3 & 0 \\ -9 & -3 & -1 & 0 \\ 10 & 4 & 2&
1 \end{array} \right| \intertext{expanding along the last column}
& = \left| \begin{array}{ccc} y-10 & x^2-4 & x-2 \\ 15 & 3 & 3 \\ 9 & 3 & 1 \end{array} \right| \\
& = (y-10)(-6) - (x^2-4)(-12) + (x-2)(18) \\
& = -6y+60 + 12 x^2 - 48 + 18x - 36\\
& = -6(y-2x^2-3x+4)
\end{align*}
Hence the equation is $y=2x^2+3x-4$.
\end{enumerate}

\bigskip

{\bf \ref{ssec.vpro}:}
\begin{enumerate}
\item \begin{enumerate}
\item $\cos\theta = \frac{1}{\sqrt{2}}$.  Hence $\theta = \frac{\pi}{4}$.
\item $\|(1, 1)\|\cos(\tfrac{\pi}{4}) = \sqrt{2}\cdot\frac{1}{\sqrt{2}} = 1$.
\item $\frac{({\bf{u}}\cdot{\bf{v}})}{\|v\|} = \frac{(1, 1)\cdot(2, 0)}{\sqrt{(2, 0)\cdot(2, 0)}}
= \frac{1(2) + 1(1)}{\sqrt{2(2) + 0(0)}} = \frac{2}{\sqrt{4}} = 1$.
\item $\frac{{\bf{v}}}{\|{\bf{v}}\|} = \frac{(2, 0)}{\sqrt{(2, 0)\cdot(2, 0)}} = (1, 0)$.
\item $1\cdot(1, 0) = (1, 0)$.
\end{enumerate}
\end{enumerate}

\bigskip

{\bf \ref{ssec.vecnspace}:}
\begin{enumerate}
\item \begin{enumerate}
\item $(4,3,-4,0)$
\item $\sqrt{62}$
\item $\sqrt{15}$
\item $-\frac{2}{\sqrt{30}}$.
\end{enumerate}
\item ${\bf u}={\bf v}$.
\end{enumerate}


{\bf Suggested Exercises}

\begin{enumerate}
\item \begin{enumerate}
\item (i) (5,2,7)  \quad (ii) (6,0,8) \quad  (iii) (4,28,-16)
\item (i) (11,2,15) \quad (ii) (-2,7,-8) \quad (iii) (10,4,14)
\item (i) 18 \quad (ii) 4 \quad  (iii) 22 \quad (iv) 22
\item 5
\item $\cos ^{-1} (\frac{4}{\sqrt{1122}})$; acute
\item $\sqrt{6}$
\end{enumerate}
\item \begin{enumerate} \item (i) (3,8,10,13,8) \quad  (ii) (0,2,1,1,-7)
\item (i) $\sqrt{55}$ \quad  (ii) $\sqrt{179}$ \quad (iii) $\sqrt{55}\sqrt{179}$
\quad (iv) 86
\item $99.2 > 86$
\end{enumerate}
\item $x=2, y=4, z=6$
\item
\begin{enumerate}
\item $ y - 4x + 2 = 0$.
\item $2y + x - 4 = 0$.
\item $y- 3x + 11 = 0$.
\end{enumerate}
\item
\begin{enumerate}
\item $x+y - 3z = 0$.
\item $y-z+1=0$.
\item $2x+3z-19=0$.
\end{enumerate}
\item
\begin{enumerate}
\item $x+2y + 3z = 9$.
\item $4x + 5y - 11z = -23$.
\end{enumerate}
\item
\begin{enumerate}
\item $y^2 - 4x - 4y + 8 = 0$.
\item $6y^2 - 2x - 16y + 8 = 0 $.
\end{enumerate}
\item $(x^2 + y^2 + z^2) -x-y-z = 0$.
\item
\begin{enumerate}
\item (i) $(-1, 2, 0)$ (ii) $\left(-\frac{30}{59},\frac{10}{59},-\frac{70}{59}\right)$ (iii) $\left(-\frac{88}{59},\frac{226}{59},\frac{70}{59}\right)$.
\item (i) $(\frac{12}{17}, -\frac{8}{17}, \frac{8}{17})$ (ii) $\left(-\frac{8}{21},-\frac{4}{21},\frac{16}{21}\right)$ (iii) $\left(\frac{71}{21},-\frac{38}{21},\frac{26}{21}\right)$.
\end{enumerate}
\end{enumerate}
%\end{document}
