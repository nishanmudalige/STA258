\pagestyle{myheadings}  \markboth{
\titleref{noter1}}{}
\section*{Notation and Terminology}\label{noter1}

\bigskip

 \quad \quad The following symbolic conventions may be used in these notes.\\
\smallskip

{\tabcolsep=5mm
\begin{tabular}{ll}

  $\exists$ & there exists \\
  &\\
  $ \forall$ & for all \\
  &\\
  $u \ \in \ V $ & the element $u$ is contained in the set $V$ \\
  &\\
  iff & if and only if \\
  &\\
  $\perp$ & perpendicular to/at a ninety degree angle to/orthogonal to \\
  &\\
  $\notin $ & the element is not contained in the set\\
  &\\
  $ W \ \subset \ V$ & $W$ is a strict ($W$ cannot be all of $V$) subset contained in $V$ \\
  &\\
  $\subseteq$ & a subset, which can be the whole set\\
  &\\
  $\mbox{\tebbb R}$ & the set of real numbers\\
  &\\
  $\mbox{\tebbb R}^2$ & the plane; two space\\
  &\\
  $\mbox{\tebbb R}^3$ & three space\\
  &\\
  $\Rightarrow$ & implies.  For example: statement 1 $\Rightarrow$ statement 2  is read as \\ & ``if statement 1, then statement 2''\\
  &\\
  $\Leftrightarrow$ & if and only if;  For example, $[1] \Leftrightarrow [2]$ is read as ``statement 1 \\
  & implies statement 2, and statement 2 implies statement 1'', \\
  & or statement 1 and statement 2 are equivalent\\
  &\\
  $\{ \ \cdot \  \}$ & a set of objects\\
  &\\
  $\neq$  &  Not equal to\\
 \end{tabular}
}

\newpage
The following gives some of the terminology that will be defined
in the notes. A quick summary is given here, the actual
definitions are given in the body of the notes.
\\
\\
{\tabcolsep=5mm
\begin{tabular}{ll}

  $A_{m \times n}$ & used to denote a generic matrix \\
   &  \\
  $R_{m \times n}$ & used to denote a matrix in row echelon form, or reduced row\\
  &-echelon form \\
   &  \\
  $ A  \ \leadsto  \ R$ & $A$ is reduced to $R$ using row operations \\
   &  \\
  ${\bf v}$ & denotes a vector \\
   &  \\
  $x_i$ & denotes a variable \\
   &  \\
  $k_i, \ c_i $ & denotes a constant value \\
   &  \\
  $S$ & denotes a set of objects \\
   &  \\
  $A^t$ & the transpose of $A$ \\
   &  \\
  $ A^{-1}$ & the inverse of $A$ \\
   &  \\
  $ {\rm tr}(A)$ & the trace of $A$ \\
   &  \\
  $ \| \ \cdot \ \| $ & the length of a vector \\
   &  \\
  $ \lambda$ & an eigenvalue \\

\end{tabular}
} 
\newpage