\begin{enumerate}
\item Verify that: $$A^{-1}=\frac{1}{ad-bc}\left [
\begin{array}{rr} d&-b\\-c&a \end{array} \right ]$$ is the
correct formula for the inverse of $A$.

\noindent \textbf{Solution}

\noindent We can verify that $A^{-1}$ is the correct formula for
the inverse of $A$ by proving that $$AA^{-1}=I_{2}=A^{-1}A.$$
\noindent{\underline{Part 1:}}\ Prove $AA^{-1}=I_{2}$. \newline
Given $A= \left [\begin{array}{rr}a&b\\c&d \end{array} \right]$
and $A^{-1}=\dfrac{1}{ad-bc}
\left [\begin{array}{rr}d&-b\\-c&a\end{array} \right ]$. Then
\begin{align*}
AA^{-1} &= \left[\begin{array}{rr}
                         a&b\\
                         c&d \end{array} \right]
\left [\begin{array}{rr}\vspace{1mm}
                       \frac{d}{ad-bc}&\frac{-b}{ad-bc}\\ \vspace{1mm}
                      \frac{-c}{ad-bc}&\frac{a}{ad-bc} \end{array} \right]\\
&= \left[ \begin{array}{rr} \vspace{1mm}
                         \frac{ad-bc}{ad-bc}&\frac{-ab+ab}{ad-bc}\\ \vspace{1mm}
                         \frac{cd-cd}{ad-bc}&\frac{-bc+ad}{ad-bc} \end{array} \right] \\
&= \left[ \begin{array}{rr}
                        1&0\\
                        0&1 \end{array} \right]=I_{2}.
\end{align*}
{\underline{Part 2:}}\ Show that $A^{-1}A=I_{2}$.
\begin{align*}
A^{-1}A &= \left[ \begin{array}{ll} \vspace{1mm}
                       \frac{d}{ad-bc}&\frac{-b}{ad-bc}\\ \vspace{1mm}
                      \frac{-c}{ad-bc}&\frac{a}{ad-bc} \end{array} \right]
\left[ \begin{array}{ll}
                      a&b\\
                      c&d \end{array} \right] \\
&= \left[ \begin{array}{ll} \vspace{1mm}
                         \frac{ad-bc}{ad-bc}&\frac{db-bd}{ad-bc}\\ \vspace{1mm}
                         \frac{-ac+ac}{ad-bc}&\frac{-bc+ad}{ad-bc} \end{array} \right] \\
&= \left [\begin{array}{ll}
                        1&0\\
                        0&1 \end{array} \right ]=I_2
\end{align*}
\noindent Since $AA^{-1}=I_{2}=A^{-1}A$ we can say $A^{-1}$ is the
inverse of A.  From properties 3 and 4 in
section~\ref{ssec.propinv}, it is actually only necessary to show either Part 1 or Part 2 above.

\item Show that if $A$ and $B$ are invertible and the same size, then
$(AB)^{-1}=B^{-1}A^{-1}$. Hint: use the definition of an inverse.

\noindent \textbf{Solution}

Use the definition of an inverse and the fact that both
$A$ and $B$ are invertible. Then we must show that
$$
(AB)(B^{-1}A^{-1})=(B^{-1}A^{-1})(AB)=I
$$
{\underline{Proof:}}
\begin{align*}
(AB)(B^{-1}A^{-1})&=A(BB^{-1})A^{-1}\\
&=AIA^{-1}\\
&=AA^{-1}=I
\intertext{and}
(B^{-1}A^{-1})(AB)&=B^{-1}(A^{-1}A)B\\
&=B^{-1}IB\\
&=B^{-1}B=I.
\end{align*}
It then follows that $(AB)^{-1}=B^{-1}A^{-1}$.
\item Find the inverse of $\left [ \begin{array}{rrrrr}
                            a_{11}&0&0&0&0\\
                            0&a_{22}&0&0&0\\
                            0&0&a_{33}&0&0\\
                            0&0&0&a_{44}&0\\
                            0&0&0&0&a_{55} \end{array} \right ]$.

\noindent \textbf{Solution}

\noindent Assuming all $a_{ii}$'s do not equal zero the inverse
is:
\begin{eqnarray*}
\left [\begin{array}{rrrrr}
                        \frac{1}{a_{11}}&0&0&0&0\\
                        0&\frac{1}{a_{22}}&0&0&0\\
                        0&0&\frac{1}{a_{33}}&0&0\\
                        0&0&0&\frac{1}{a_{44}}&0\\
                        0&0&0&0&\frac{1}{a_{55}} \end{array} \right ]
\end{eqnarray*}

\item Find the inverse of the following matrices.

$\rm{(a)} \left [ \begin{array}{rr} -3&-8\\2&5 \end{array} \right
] \quad \quad  \rm{(b)} \left [ \begin{array}{rr} 4&-8\\-1&2
\end{array} \right ] \quad \quad \rm{(c)} \left [ \begin{array}{rr} 2&0\\-5&2\end{array}
\right ]$\\

\noindent \textbf{Solution}

\begin{description}\item (a) Using the formula we proved in Question \#$1$ of the
Assignment, the inverse of the matrix is:
\begin{eqnarray*}
\dfrac{1}{-15+16} \left[ \begin{array}{rr}5&8\\-2&-3 \end{array} \right]
=\left[ \begin{array}{rr}5&8\\-2&-3 \end{array} \right]
\end{eqnarray*}
\item (b)
\begin{eqnarray*}
\left |\begin{array}{rr}4&-8\\-1&2 \end{array} \right | = 8-8 = 0.
\end{eqnarray*}
\noindent Since the determinant is zero, you
can conclude that this matrix has no inverse.
\item (c)
\begin{eqnarray*}
\frac{1}{4-0} \left[ \begin{array}{rr}2&0\\5&2 \end{array} \right]= \left[ \begin{array}{rr}\frac{1}{2}&0\\\frac{5}{4}&\frac{1}{2} \end{array} \right]
\end{eqnarray*}
\end{description}
\item Find the inverse of the following matrices.

$\rm{(a)} \left [ \begin{array}{rrr} 2&1&0\\-4&1&1\\ 0&-2&-1
\end{array} \right ] \rm{(b)} \left [
\begin{array}{rrr} 3&0&0\\0&1&0\\ 0&0&-2\\ \end{array} \right ]$
$\rm{(c)} \left [ \begin{array}{rrr} 2&5&4\\5&-2&-1\\2&1&1\\
\end{array} \right ] \ \ \ \!\rm{(d)} \left [
\begin{array}{rrr} 2&1&8\\ 4&0&3\\-2&1&5 \end{array}  \right ]$\\

\noindent \textbf{Solution}

\begin{description} \item(a)
\begin{eqnarray*}
&&\left[ \begin{array}{rrrcrrr}2&1&0&\vline&1&0&0\\
                            -4&1&1&\vline&0&1&0\\
                             0&-2&-1&\vline&0&0&1
\end{array} \right]
\\&& \hspace{-12mm} \overset{\begin{smallmatrix}R_2+2R_1\\
\frac{1}{2}R_1\end{smallmatrix}}{\leadsto}
\left[ \begin{array}{rrrcrrr} 1&\frac{1}{2}&0&\vline&\frac{1}{2}&0&0\\
                               0&3&1&\vline&2&1&0\\
                               0&-2&-1&\vline&0&0&1
\end{array} \right]
\\&& \hspace{-12mm}
\overset{\begin{smallmatrix} \frac{1}{3}R_2\\R_3+2R_2\end{smallmatrix}}{\leadsto}
\left[ \begin{array}{rrrcrrr}1&\frac{1}{2}&0&\vline&\frac{1}{2}&0&0\\
                              0&1&\frac{1}{3}&\vline&\frac{2}{3}&\frac{1}{3}&0\\
                              0&0&-\frac{1}{3}&\vline&\frac{4}{3}&\frac{2}{3}&1
\end{array} \right]
\\&& \hspace{-12mm}
\overset{\begin{smallmatrix}R_{1}-\frac{1}{2}R_{2}\\-3R_3\end{smallmatrix}}{\leadsto}
\left[ \begin{array}{rrrcrrr}1&0&-\frac{1}{6}&\vline&\frac{1}{6}&-\frac{1}{6}&0\\
                              0&1&\frac{1}{3}&\vline&\frac{2}{3}&\frac{1}{3}&0\\
                              0&0&1&\vline&-4&-2&-3
\end{array} \right]
\\&& \hspace{-12mm}
\overset{\begin{smallmatrix}R_{2}-\frac{1}{3}R_{3}\\R_1+\frac{1}{6}R_3\end{smallmatrix}}{\leadsto}
\left[ \begin{array}{rrrcrrr}1&0&0&\vline&-\frac{1}{2}&-\frac{1}{2}&-\frac{1}{2}\\
                               0&1&0&\vline&2&1&1\\
                               0&0&1&\vline&-4&-2&-3
\end{array} \right]
\end{eqnarray*}
Then the matrix on the righthand side is the inverse.
\item (b)
$$
\left[ \begin{array}{rrr}
                       \frac{1}{3}&0&0\\
                       0&1&0\\
                       0&0&-\frac{1}{2}
\end{array} \right]
$$
\item (c)
$$
\left [\begin{array}{rrr}
                       1&1&-3\\
                       7&6&-22\\
                       -9&-8&29\
\end{array} \right]
$$
\item (d) Expand the determinant along the top row.
\begin{align*}
\left| \begin{array}{rrr} 2 & 1 & 8  \\ 4 & 0 & 3 \\ -2 & 1 & 5 \end{array} \right|
&= 2(-3) -1(20+6) + 8(4)\\
&= -6 -26 + 32 = 0.
\end{align*}
Since the determinant is zero, the matrix is not invertible.
\end{description}

\item Find the inverse of the following matrices.

$$
\rm {(a)} \left [ \begin{array}{rrrr} -1&0&2&0\\ 2&0&-1&0\\
0&-1&0&2\\ 0&2&0&-1\end{array} \right ] \quad \quad \rm {(b)}
\left [
\begin{array}{rrrr} 2&1&8&-2\\0&1&-1&2\\0&0&3&-1\\0&0&0&-1\end{array} \right ]
$$

$$\ \rm {(c)} \left [ \begin{array}{rrrr}
-1&4&1&0\\2&0&0&3\\1&-1&-3&2\\0&1&14&-4\end{array} \right ] \quad
\quad \rm {(d)} \left [ \begin{array}{rrrr} 1&0&3&-2\\ 1&0&1&2\\
4&6&2&8\\ 1&0&-1&7\end{array} \right ].
$$

\noindent \textbf{Solution}

\begin{description}
\item (a)
$$ \left [\begin{array}{rrrr}
                       \frac{1}{3}&\frac{2}{3}&0&0\\
                       0&0&\frac{1}{3}&\frac{2}{3}\\
                       \frac{2}{3}&\frac{1}{3}&0&0\\
                       0&0&\frac{2}{3}&\frac{1}{3}\end{array} \right ]
$$
\item (b)
$$
\left[ \begin{array}{rrrr}\vspace{1mm}
                       \frac{1}{2}&-\frac{1}{2}&-\frac{3}{2}&-\frac{1}{2}\\ \vspace{1mm}
                       0&1&\frac{1}{3}&\frac{5}{3}\\ \vspace{1mm}
                       0&0&\frac{1}{3}&-\frac{1}{3}\\
                       0&0&0&-1
\end{array} \right]
$$
\item (c) Expand the determinant along the second row.
\begin{align*}
\left| \begin{array}{rrrr} -1 & 4 & 1 & 0 \\ 2 & 0 & 0 & 3 \\ 1 & -1 & -3 & 2 \\ 0 & 1 & 14 & -4 \end{array} \right|
&= -2 \left| \begin{array}{rrr} 4 & 1 & 0 \\ -1 & -3 & 2 \\ 1 & 14 & -4 \end{array} \right| +
3 \left| \begin{array}{rrr} -1 & 4 & 1 \\ 1 & -1 & -3 \\ 0 & 1 & 14 \end{array} \right| \\
&= -2 \bigl[ 4(12-28) - (4-2) \bigr] + 3\bigl[-1(3-1) + 14(1-4) \bigr] \\
&= -2[-64-2] + 3[-2-42]\\
&= 132 - 132 = 0.
\end{align*}
Hence the matrix is singular.
\item (d)
$$
\left[ \begin{array}{rrrr} \vspace{1mm}
                       -\frac{9}{2}&\frac{19}{2}&0&-4\\ \vspace{1mm}
                       \frac{5}{6}&-\frac{13}{6}&\frac{1}{6}&\frac{2}{3}\\ \vspace{1mm}
                       \frac{5}{2}&-\frac{9}{2}&0&2\\
                       1&-2&0&1
\end{array} \right]
$$
\end{description}

\item For what values of $a$ does $A^{-1}$ fail to exist?
\begin{align*}
&\rm{(a)}\ A= \left [ \begin{array}{rr} 3&a\\-4&5 \end{array}
\right ] &\rm{(b)}\ A&=\left [\begin{array}{rc}
-3&(1+a)\\a&-2\end{array} \right]\\
&\rm{(c)}\ A= \left [ \begin{array}{rcc}
-1&(4+a)&\frac{1}{2}\\0&(a-5)&3\\ 0&0&-3a \end{array} \right]
&\rm{(d)}\ A&= \left [ \begin{array}{crc}
3a&1&0\\2&2&(a+1)\\1&0&1 \end{array} \right ].
\end{align*}

\noindent \textbf{Solution}

\begin{description}
\item (a)
\begin{align*}
\left| \begin{array}{rr} 3 & a \\ -4 & 5 \end{array} \right| = 15 + 4a = 0
\quad \mathrm{when}\ a = \tfrac{-15}{4}.
\end{align*}
Hence if $a= \tfrac{-15}{4}$, $A$ is not invertible (singular).
\item (b)
\begin{align*}
\left| \begin{array}{rr} -3 & 1+a \\ a & -2 \end{array} \right| &= 6 - a(1+a)\\
&= -a^2 - a + 6 = -(a+3)(a-2) \\
&= 0 \quad \mathrm{when}\ a=2,\ -3.
\end{align*}
Hence if $a = 2$ or $-3$, $A$ is singular.
\item (c)
Expanding down the first column,
\begin{align*}
\det{(A)} &= - \left| \begin{array}{rr} a-5 & 3 \\ 0 & -3a \end{array} \right| \\
&= 3a(a-5) = 0 \quad \mathrm{when}\ a = 0,\,5.
\end{align*}
Hence if $a=0$ or $5$, $A$ is singular.
\item (d)
Expanding along the top row,
\begin{align*}
\det{(A)} &= 3a \left| \begin{array}{rr} 2 & (a+1) \\ 0 & 1 \end{array} \right|
- \left| \begin{array}{rr} 2 & (a+1) \\ 1 & 1 \end{array} \right|\\
&= 6a - (2 - (a+1))\\
&= 7a - 1\\
&= 0 \quad \mathrm{when}\ a = \tfrac{1}{7}.
\end{align*}
Hence when $a=\tfrac{1}{7}$, $A$ is singular.
\end{description}

\item For the following system:
$$\begin{array}{rrr} 2x_1-2x_2+3x_3&=&3\\ 2x_3-x_4&=&-1\\
4x_1+5x_2+2x_3&=&1\\ x_3+2x_4&=&-3 \end{array},$$
\begin{enumerate}
\item Set up the system as $A{\bf x}={\bf b}$.
\item Find $A^{-1}.$
\item Find the solution.
\item Let ${\bf b}_1=[ \ 2 \ 0 \ 1 \ 0 \ ]^t$. Find the solution
to $A{\bf x}={\bf  b}_1$.
\item What is the solution to $A{\bf x}={\bf 0}$?
\end{enumerate}

\noindent \textbf{Solution}
\begin{description} \item (a)
$$ \left [\begin{array}{rrrr}
                       2&-2&3&0\\
                       0&0&2&-1\\
                       4&5&2&0\\
                       0&0&1&2\end{array} \right ]
\left [\begin{array}{r}
                       x_{1}\\
                       x_{2}\\
                       x_{3}\\
                       x_{4}\end{array} \right ]  =
\left [\begin{array}{r}
                       3\\
                       -1\\
                       1\\
                       -3\end{array} \right ]
$$
\item (b)
$$
A^{-1}= \left[ \begin{array}{rrrr} \vspace{1mm}
                       \frac{5}{18}&-\frac{19}{45}&\frac{1}{9}&-\frac{19}{90}\\ \vspace{1mm}
                       -\frac{2}{9}&\frac{8}{45}&\frac{1}{9}&\frac{4}{45}\\ \vspace{1mm}
                       0&\frac{2}{5}&0&\frac{1}{5}\\
                       0&-\frac{1}{5}&0&\frac{2}{5}\end{array} \right]
$$
\item (c)
\begin{align*}
{\bf x}&=A^{-1}{\bf b}\\
\left[ \begin{array}{r}x_{1}\\
                       x_{2}\\
                       x_{3}\\
                       x_{4}
\end{array} \right ] &=
\left[ \begin{array}{rrrr} \vspace{1mm}
                       \frac{5}{18}&-\frac{19}{45}&\frac{1}{9}&-\frac{19}{90}\\ \vspace{1mm}
                       -\frac{2}{9}&\frac{8}{45}&\frac{1}{9}&\frac{4}{45}\\ \vspace{1mm}
                       0&\frac{2}{5}&0&\frac{1}{5}\\
                       0&-\frac{1}{5}&0&\frac{2}{5}
\end{array} \right]
\left[ \begin{array}{r} 3\\
                       -1\\
                       1\\
                       -3
\end{array} \right] \\
\left [\begin{array}{r}x_{1}\\
                       x_{2}\\
                       x_{3}\\
                       x_{4}
\end{array} \right ] &=
\left[ \begin{array}{r} 2\\
                       -1\\
                       -1\\
                       -1
\end{array} \right]
\end{align*}
\item (d)
Rearrange to get ${\bf x}=A^{-1}{\bf b_{1}}$. Examine.
\begin{align*}
\left[ \begin{array}{r}
                       x_{1}\\
                       x_{2}\\
                       x_{3}\\
                       x_{4}\end{array} \right]
&=
\left[ \begin{array}{rrrr} \vspace{1mm}
                       \frac{5}{18}&-\frac{19}{45}&\frac{1}{9}&-\frac{19}{90}\\ \vspace{1mm}
                       -\frac{2}{9}&\frac{8}{45}&\frac{1}{9}&\frac{4}{45}\\ \vspace{1mm}
                       0&\frac{2}{5}&0&\frac{1}{5}\\
                       0&-\frac{1}{5}&0&\frac{2}{5}\end{array} \right]
\left[ \begin{array}{r}
                       2\\
                       0\\
                       1\\
                       0
\end{array} \right] \\
\left[ \begin{array}{r}x_{1}\\
                       x_{2}\\
                       x_{3}\\
                       x_{4}
\end{array} \right]&=
\left[ \begin{array}{r}\vspace{1mm} \frac{2}{3}\\
                        -\frac{1}{3}\\
                       0\\
                       0\end{array} \right]
\end{align*}
\item (e)
$A$ is invertible, so the solution to the homogeneous
system is ${\bf 0 }$. Observe: $A{\bf x}={\bf 0}$.
Rearrange to get ${\bf x}=A^{-1}{\bf 0}$.
\begin{align*}
\left[ \begin{array}{r}x_{1}\\
                       x_{2}\\
                       x_{3}\\
                       x_{4}
\end{array} \right] &=
\left[ \begin{array}{rrrr}\vspace{1mm}
                       \frac{5}{18}&-\frac{19}{45}&\frac{1}{9}&-\frac{19}{90}\\ \vspace{1mm}
                       -\frac{2}{9}&\frac{8}{45}&\frac{1}{9}&\frac{4}{45}\\ \vspace{1mm}
                       0&\frac{2}{5}&0&\frac{1}{5}\\
                       0&-\frac{1}{5}&0&\frac{2}{5}
\end{array} \right]
\left [\begin{array}{r}0\\
                       0\\
                       0\\
                       0
\end{array} \right]\\
\left[ \begin{array}{r}x_{1}\\
                       x_{2}\\
                       x_{3}\\
                       x_{4}
\end{array} \right ] &=
\left[ \begin{array}{r}0\\
                       0\\
                       0\\
                       0
\end{array} \right]
\end{align*}
\end{description}

\item For the following system:
$$
\begin{array}{rrr} x_1+3x_3+2x_4&=&4\\
                   2x_1-x_2+8x_3&=&3\\
                   3x_1+2x_2+4x_3+3x_4&=&1\\
                   2x_1+3x_2-x_3+5x_4&=&2
\end{array}
$$
\begin{enumerate}
\item Set up the system as $A{\bf x}={\bf b}$.
\item Find $A^{-1}.$
\item Find the solution.
\item Let ${\bf b}_1=[ \ 1 \ -1 \ 0 \ 2 \ ]^t$. Find the solution
to $A{\bf x}={\bf  b}_1$.
\item What is the solution to $A{\bf x}={\bf 0}$?
\end{enumerate}

\noindent \textbf{Solution} \begin{description} \item (a)
$$
\left[ \begin{array}{rrrr}
                       1&0&3&2\\
                       2&-1&8&0\\
                       3&2&4&3\\
                       2&3&-1&5
\end{array} \right]
\left[ \begin{array}{r}x_{1}\\
                       x_{2}\\
                       x_{3}\\
                       x_{4}
\end{array} \right] =
\left [\begin{array}{r}4\\
                       3\\
                       1\\
                       2\end{array} \right]
$$
\item (b)
The matrix is not invertible.
\item (c)
Write system as an augmented matrix in reduced
row-echelon form since we cannot use the inverse:
$$
\left[ \begin{array}{rrrr|r}
                         1&0&0&-31&-59\\
                         0&1&0&26&47\\
                         0&0&1&11&21\\
                         0&0&0&0&0
\end{array} \right]
$$
Let $x_4=s$.  Then $x_3=21-11s,\ x_2=47-26s$, and
$x_1=-59+31s$.
\item (d)
Write the system as an augmented matrix and row-reduce
to solve.
$$
\left[ \begin{array}{rrrr|r}
                         1&0&0&-31&-26\\
                         0&1&0&26&21\\
                         0&0&1&11&9\\
                         0&0&0&0&0
\end{array} \right]
$$
Let $x_4=s$.  Then $x_3=9-11s,\ x_2=21-26s$, and
$x_1=-26+31s$.
\item (e)
Notice that you may already use the matrix in
row-reduced echelon form with the last column containing all zero
entries.
$$
\left[ \begin{array}{rrrr|r}
                         1&0&0&-31&0\\
                         0&1&0&26&0\\
                         0&0&1&11&0\\
                         0&0&0&0&0
\end{array} \right]
$$
Let $x_4=s$.  Then $x_3=-11s,\ x_2=-26s$, and $x_1=31s$.
\end{description}

\item For
$$
A=\left[ \begin{array}{rrr} 1&5&2\\ -3&4&1\\-1&2&0 \end{array}
\right ], \quad B=\left [ \begin{array}{rrr} 2&5&-1\\ 2&-3&1\\
6&-1&1 \end{array} \right ] \mbox{\ and\ } C=\left [
\begin{array}{rrr} 0&3&1\\ -2&0&3\\ 4&2&0 \end{array} \right ],$$
find
\begin{enumerate}
\item (i) det($A$) \quad (ii) det($B$) \quad (iii) det($C$)
\item (i) det($AC$) \quad (ii) det($AB$) \quad (iii) det($BC$)
\item (i) det($A$)+det($B$) \quad (ii) det($A+B$)
\item (i) det($2A$) \quad (ii) 2det($A$)
\item (i) det($C^t$) \quad (ii) det($CA^t$)
\item det($C^{-1}$)
\end{enumerate}

\noindent \textbf{Solution} \begin{description}
\item (a) (i)
Expand along the top row
\begin{align*}
\left| \begin{array}{rrr} 1 & 5 & 2\\ -3 & 4 & 1 \\ -1 & 2 & 0 \end{array} \right|
&= 1(0-2) -5(0+1) + 2(-6+4)\\
&= -2-5-4 = -11.
\end{align*}
\item \quad\, (ii) \begin{align*}
\left| \begin{array}{rrr} 2 & 5 & -1 \\ 2 & -3 & 1 \\ 6 & -1 & 1 \end{array} \right|
&= \left| \begin{array}{rrr} 8 & 4 & 0 \\ -4 & -2 & 0 \\ 6 & -1 & 1 \end{array} \right| \quad \begin{array}{r} R_1 + R_3 \\ R_2 - R_3 \end{array}\\
&= 4 \left| \begin{array}{rrr} 2 & 1 & 0 \\ -4 & -2 & 0 \\ 6 & -1 & 1 \end{array} \right|\\
&= 4(-2) \left| \begin{array}{rrr} 2 & 1 & 0 \\ 2 & 1 & 0\\ 6 & -1 & 1 \end{array} \right| \\
&= 0 \quad \mathrm{(2\ rows\ the\ same)}.
\end{align*}
\item \quad\,  (iii)
Expand along the top row:
\begin{align*}
\left| \begin{array}{rrr} 0 & 3 & 1 \\ -2 & 0 & 3 \\ 4 & 2 & 0 \end{array} \right| &=
-3 \left| \begin{array}{rr} -2 & 3 \\ 4 & 0 \end{array} \right| + \left| \begin{array}{rr} -2 & 0 \\ 4 & 2 \end{array} \right|\\
&= -3(-12) + (-4) = 32.
\end{align*}

\item (b) (i)
det$(AC)\ =\ $det$(A)$det$(C)\ =\ (-11)(32)\ =\ -352$\\
\item \quad\, (ii)
det$(AB)\ =\ $det$(A)$det$(B)\ =\ (-11)(0)\ =\ 0$\\
\item \quad\, (iii)
det$(BC)\ =\ $det$(B)$det$(C)\ =\ (0)(32)\ =\ 0$\\
\item (c) (i)
det$(A)+$det$(B)\ =\ (-11)+(0)\ =\ -11$\\
\item \quad\, (ii)
det$(A+B)\ =\ \left | \begin{array}{rrr}
                        3&10&1\\
                        -1&1&2\\
                        5&1&1 \end{array} \right |\ =\ 101$\\
\item (d) (i)
det$(2A)\ =\ 2^3$det$(A)\ =\ 8(-11)\ =\ -88$\\
\item \quad\, (ii)
$2$det($A$)$=2(-11)=-22$\\
\item (e) (i)
det$(C^t)\ =\ $det$(C)\ =\ 32$\\
\item \quad\, (ii)
det$(CA^t)\ =\ $det$(C)$det$(A^t)\ =\ $det$(C)$det$(A)\ =\
(32)(-11)\ =\ -352$\\
\item (f)
det($C^{-1}$)$=\frac{1}{32}$ which is the reciprocal of det($C$).
\end{description}

\item Find the determinant of the following matrices.
\begin{align*}
&\mathrm{(a)} \left[ \begin{array}{rr} 6&3\\-2&0 \end{array} \right]&
&\mathrm{(b)} \left[ \begin{array}{rr} 1&-2\\-2&4\end{array} \right]&
&\mathrm{(c)} \left[ \begin{array}{rr} 7&8\\-1&5 \end{array} \right]\\
&\mathrm{(d)} \left[ \begin{array}{rr} a&5\\ a&-1\end{array} \right]&
&\mathrm{(e)} \left[ \begin{array}{cc} (b-1)&-1\\ 3&(b+2) \end{array} \right]&
&\mathrm{(f)} \left[ \begin{array}{cc} a&(a+b)\\(b-a)&b \end{array} \right]
\end{align*}

\noindent \textbf{Solution} \begin{description} \item (a)
\begin{align*}
\left| \begin{array}{rr} 6&3\\-2&0 \end{array} \right| &=a_{11}\,a_{22}-a_{12}\,a_{21}\\
&=(6)(0)-(3)(-2)\\&
=0+6=6
\end{align*}
\item (b)
\begin{align*}
\left| \begin{array}{rr} 1&-2\\-2&4\end{array} \right | &=a_{11}\,a_{22}-a_{12}\,a_{21} \\
&=(1)(4)-(-2)(-2)\\
&=4-4=0
\end{align*}
\item (c)
\begin{align*}
\left| \begin{array}{rr} 7&8\\-1&5 \end{array} \right | &=a_{11}\,a_{22}-a_{12}\,a_{21}  \\
&=(7)(5)-(8)(-1)\\
&=35+8=43
\end{align*}
\item (d)
\begin{align*}
\left| \begin{array}{rr} a&5\\ a&-1\end{array} \right| &= a_{11}\,a_{22}-a_{12}\,a_{21} \\
&=(a)(-1)-(5)(a)=-6a
\end{align*}
\item (e)
\begin{align*}
\left| \begin{array}{cc} (b-1)&-1\\ 3&(b+2) \end{array} \right|& =a_{11}\,a_{22}-a_{12}\,a_{21}\\
&=(b-1)(b+2)-(-1)(3)\\
&=(b^2+b-2)+3=b^2+b+1
\end{align*}
\item (f)
\begin{align*}
\left| \begin{array}{cc} a&(a+b)\\(b-a)&b \end{array} \right | &=a_{11}\,a_{22}-a_{12}\,a_{21} \\
&=(a)(b)-(a+b)(b-a)\\
&=ab-(b^{2}-a^{2})=a^2+ab-b^2
\end{align*}
\end{description}

\item Find the determinant of the following matrices.
\begin{align*}
&\mathrm{(a)} \left[ \begin{array}{rrr} 2&0&1\\ 4&-5&1 \\ 0&3&2 \end{array} \right]&
&\mathrm{(b)} \left[ \begin{array}{rrr} 5&7&2\\-3&6&5\\ 2&1&-1\end{array} \right]&
&\mathrm{(c)} \left[ \begin{array}{rrr} 1&6&-1\\-1&-2&1 \\ 3&1&-3\end{array} \right ]\\
&\mathrm{(d)} \left[ \begin{array}{rrr} 2&0&0\\ b&3&b\\ 1&-1&3 \end{array} \right]&
&\mathrm{(e)} \left[ \begin{array}{crr} 2a&0&1\\ (a-3)&5&4\\ 2&0&-a \end{array} \right]&
&\mathrm{(f)} \left [ \begin{array}{ccc} a&2&-3\\(a+2)&a^2&3\\0&1&(a-1)\end{array} \right]
\end{align*}

\noindent \textbf{Solution} \begin{description} \item (a)
Expand along the top row
\begin{align*}
\begin{vmatrix} 2&0&1\\ 4&\!\!\!-5&1 \\ 0&3&2 \end{vmatrix} &= 2 \left| \begin{array}{rr}-5 & 1\\3 & 2\end{array} \right| +
\left| \begin{array}{rr} 4 & -5 \\ 0 & 3 \end{array} \right| \\
&= 2(-10-3) + 12 \\
&= -26 + 12 = -14.
\end{align*}
\item (b)
Expand along the top row
$$
\left| \begin{array}{rrr} 5&7&2\\-3&6&5\\ 2&1&-1\end{array} \right| = -36
$$
\item (c)
\begin{align*}
\left| \begin{array}{rrr} 1&6&-1\\-1&-2&1 \\ 3&1&-3\end{array} \right| &=
\left| \begin{array}{rrr} 0 & 4 & 0\\-1 & -2 & 1\\0 & -5 & 0 \end{array} \right| \quad \begin{array}{r} R_1 + R_2 \\ \ \\ R_3 + 3R_2 \end{array}\\
&= 0.
\end{align*}
\item (d)
Expand along the top row
$$
\left| \begin{array}{rrr} 2&0&0\\ b&3&b\\ 1&-1&3 \end{array} \right|=18+2b
$$
\item (e)
Expand down the middle column
$$
\left|\begin{array}{crr} 2a&0&1\\ (a-3)&5&4\\ 2&0&-a \end{array} \right|=-10a^{2}-10
$$
\item (f)
Expand down the first column
\begin{align*}
\left| \begin{array}{ccc} a&2&-3\\(a+2)&a^2&3\\0&1&(a-1)\end{array} \right| &= a\left| \begin{array}{rr} a^2 & 3 \\ 1 & (a-1) \end{array} \right|
- (a+2) \left| \begin{array}{rr}2 & -3 \\1 & (a-1) \end{array} \right|\\
&= a(a^3 - a^2 - 3) - (a+2)(2a-2+3)\\
&= a^4 - a^3 - 3a - (2a^2 + 5a + 2)\\
&= a^4 - a^3 - 2a^2 - 8a - 2.
\end{align*}
\end{description}

\item Find the determinant of the following matrices.
\begin{align*}
&\mathrm{(a)} \left[ \begin{array}{rrrr} 2&1&0&0\\ 3&2&1&4\\-1&1&0&2\\8&1&0&3 \end{array} \right]&
&\mathrm{(b)} \left[ \begin{array}{rrrr} 3&0&2&1\\4&-7&7&5\\3&-6&4&8\\2&1&-1&4 \end{array} \right]\\
&\mathrm{(c)} \left[ \begin{array}{rrrr} 0&3&1&3\\-1&0&2&1\\2&1&0&-1\\3&-2&2&0\end{array} \right]&
&\mathrm{(d)} \left[ \begin{array}{rrrr} 1&0&3&5\\4&2&-1&6\\0&0&4&0\\-7&2&0&5\end{array} \right]
\end{align*}\\
\noindent \textbf{Solution} 

\begin{description} \item (a)
Expand down the third column
\begin{align*}
\left| \begin{array}{rrrr}2&1&0&0\\ 3&2&1&4\\-1&1&0&2\\ 8&1&0&3 \end{array} \right| &=
- \left| \begin{array}{rrr}2 & 1 & 0 \\ -1 & 1 & 2 \\ 8 & 1 & 3 \end{array} \right| \\
&= -\bigl[2(3-2) - (-3-16)\bigr] \\
&=-(2+19) = -21.
\end{align*}
\item (b)
\noindent The determinant is 0.
\item (c)
\noindent The determinant is 55.
\item (d)
\noindent The determinant is 432.
\end{description}

\item Find the determinants of
\begin{align*}
\mathrm{(a)}\ A& =\left[ \begin{array}{rrr}0&0&a_{13}\\0&a_{22}&a_{23}\\a_{31}&a_{32}&a_{33} \end{array} \right]&
\mathrm{(b)}\ A&=\left[ \begin{array}{rrrr}0&0&0&a_{14}\\0&0&a_{23}&a_{24}\\0&a_{32}&a_{33}&a_{34}\\a_{41}&a_{42}&a_{43}&a_{44} \end{array} \right]\\
\mathrm{(c)}\ A&=\left[ \begin{array}{rrrrr}0&0&0&0&a_{15}\\0&0&0&a_{24}&a_{25}\\0&0&a_{33}&a_{34}&a_{35}\\
0&a_{42}&a_{43}&a_{44}&a_{45}\\ a_{51}&a_{52}&a_{53}&a_{54}&a_{55} \end{array}\right]
\end{align*}

\noindent \textbf{Solution} 

\begin{description}\item (a)
Expand along the top row
\begin{align*}
\det(A) = \left| \begin{array}{rrr}0&0&a_{13}\\0&a_{22}&a_{23}\\a_{31}&a_{32}&a_{33} \end{array} \right|
= a_{13} \left| \begin{array}{rr} 0 & a_{22} \\ a_{31} & a_{32} \end{array} \right| = - a_{13}a_{31}a_{22}.
\end{align*}

\item (b)
Expand along the top row
\begin{align*}
\det(A)= \left| \begin{array}{rrrr}0&0&0&a_{14}\\0&0&a_{23}&a_{24}\\0&a_{32}&a_{33}&a_{34}\\a_{41}&a_{42}&a_{43}&a_{44} \end{array} \right|
&= -a_{14} \left| \begin{array}{rrr} 0 & 0 & a_{23} \\ 0 & a_{32} & a_{33} \\ a_{41} & a_{42} & a_{43} \end{array} \right| \\
&= a_{14}a_{23}a_{32}a_{41} \quad \mathrm{using\ (a)}.
\end{align*}

\item (c)
Expand along the top row and use (b)
\begin{align*}
\det(A) = \left| \begin{array}{rrrrr}
                        0&0&0&0&a_{15}\\
                        0&0&0&a_{24}&a_{25}\\
                        0&0&a_{33}&a_{34}&a_{35}\\
                        0&a_{42}&a_{43}&a_{44}&a_{45}\\
                        a_{51}&a_{52}&a_{53}&a_{54}&a_{55} \end{array} \right | =
a_{15}\,a_{24}\,a_{33}\,a_{42}\,a_{51}.
\end{align*}
\end{description}

\item Assume that $$\left | \begin{array}{rrr}
a&b&c\\d&e&f\\g&h&i \end{array} \right |=1.$$ Find the
determinants of the following matrices.
\begin{align*}
&\mathrm{(a)} \left[ \begin{array}{rrr} 5a&5b&5c\\5d&5e&5f\\5g&5h&5i \end{array} \right]&
&\mathrm{(b)} \left[ \begin{array}{rrr} d&e&f\\ a&b&c \\g&h&i \end{array} \right ]\\
&\mathrm{(c)} \left[ \begin{array}{rrr} -a&-b&-c\\ d&e&f\\ 3g&3h&3i\end{array} \right ]&
&\mathrm{(d)} \left[ \begin{array}{ccc} (a-d)&(b-e)&(c-f)\\ d&e&f\\ 2g&2h&2i\end{array}\right]\\
&\mathrm{(e)} \left[ \begin{array}{ccc} -d&-e&-f\\ a&b&c\\ (g+4a)&(h+4b)&(i+4c) \end{array} \right ]&
&\mathrm{(f)} \ 4\left[ \begin{array}{rrr} a&b&c\\d&e&f\\g&h&i \end{array} \right]^{-1}
\end{align*}\\

\noindent \textbf{Solution} \begin{description}\item (a)
\begin{align*}
\begin{vmatrix}5a&5b&5c\\5d&5e&5f\\5g&5h&5i \end{vmatrix}
&=(5)(5)(5) \begin{vmatrix}a&b&c\\d&e&f\\g&h&i \end{vmatrix}\\
&=5^3(1)\\
&=125
\end{align*}
\item (b)
\begin{align*}
\begin{vmatrix}d&e&f\\ a&b&c \\g&h&i \end{vmatrix}
&=(-1)\begin{vmatrix}a&b&c\\d&e&f\\g&h&i \end{vmatrix}\\
&=(-1)(1)\\
&=-1
\end{align*}
\item (c)
\begin{align*}
\left| \begin{array}{rrr}-a&-b&-c\\ d&e&f\\ 3g&3h&3i \end{array} \right|
&=(-1)(3)\left | \begin{array}{rrr}a&b&c\\d&e&f\\g&h&i \end{array} \right|\\
&=(-1)(3)(1)\\
&=-3
\end{align*}
\item (d)
\begin{align*}
\left| \begin{array}{rrr}(a-d)&(b-e)&(c-f)\\ d&e&f\\ 2g&2h&2i \end{array} \right|
\ \ \overset{R_{1}+R_{2}}{\leadsto} \ \
(2)\begin{vmatrix}a&b&c\\ d&e&f\\ g&h&i \end{vmatrix} = 2
\end{align*}
\item (e)
\begin{align*}
\left| \begin{array}{ccc}-d&-e&-f\\a&b&c\\(g+4a)&(h+4b)&(i+4c)\end{array} \right|
&=(-1)(-1) \begin{vmatrix}a&b&c\\d&e&f\\g&h&i \end{vmatrix}\\
&=(-1)(-1)(1)\\
&=1
\end{align*}
\item (f)
$$
\det(4A^{-1})=4^3(1)=64
$$
\end{description}

\item Find all values of $b$ for which the following matrices fail to be invertible.
\begin{align*}
\mathrm{(a)}\ \left[ \begin{array}{rrr} 2&1&0\\-b&3&4\\-1&3b&-1 \end{array} \right] &
&\mathrm{(b)}\ \left[ \begin{array}{crc}(b+3)&3&-1\\0&-b&(2-b)\\1&1&0\end{array} \right] &
&\mathrm{(c)}\ \left[ \begin{array}{crcr} 1&0&2&3\\(b+2)&-b&b&0\\0&2&5b&1\\0&0&(3-b)&0 \end{array} \right]
\end{align*}
\noindent The matrices fail to be invertible when the determinant
equals zero.\\

\noindent \textbf{Solution} \begin{description}\item (a)
The determinant is $-10-25b$.  Setting it equal to zero,
we get $b=-\frac{2}{5}$ so that the matrix is not invertible.
\item (b)
The determinant is $b^2-3b$.  Setting it equal to zero,
we get $b=0,3$ so that the matrix is not invertible.
\item (c) The determinant is $5b^2-3b-36$.  Setting it equal to
zero, we get $b=3,-\tfrac{12}{5}$ so that the matrix is not invertible.
\end{description}

\item For the following homogeneous systems,
\ \

{\rm (i.)}$\begin{array}{rrl} 2x_1+x_2+3x_4&=&0\\ 3x_1+2x_2&=&0\\
5x_1-x_2-2x_3&=&0\\ 3x_2-x_3&=&0 \end{array}$

\ \

{\rm (ii.)}$ \begin{array}{rrl} 5x_1+x_3+3x_4&=&0\\
9x_1-2x_4&=&0\\ -4x_1+3x_2+6x_3+5x_4&=&0\\ 3x_2+5x_3&=&0
 \end{array}$

\begin{enumerate}
\item Determine whether the system has only the trivial solution.
\item For ${\bf b}=[\ b_1\ b_2\ b_3 \ b_4 \ ]^t$, determine whether an {\bf x} can always be found such that
A{\bf x}={\bf b}.
\end{enumerate}

\ \
\noindent \textbf{Solution} \begin{description}\item (i) (a)
 To determine whether the system has only the trivial
solution we must find out whether the determinant of the
coefficient matrix is zero.  We find that the determinant for the
coefficient matrix is $-93$; therefore, the system has only the
trivial solution.
\item (i) (b)
Yes an {\bf x} can be found such that $A{\bf x}={\bf b}$
since the determinant of the coefficient matrix in invertible.
\item (ii) (a)
The determinant of the coefficient matrix for the system
is zero. Therefore the solution for the system includes the
trivial solution, but the solution is infinite.
\item (ii) (b)
Since the determinant is zero, it cannot be guaranteed
that for all {\bf b} an {\bf x} can be found such that $A{\bf
x}={\bf b}$.
\end{description}

\item This question refers to the following system.
$\begin{array}{rrl} x+2y+5z&=&-7\\ 4x+2y+3z&=&-14\\ 2x-4y+z&=&-9
\end{array}$
\begin{enumerate}
\item Find the determinant of the coefficient matrix $A$. What does
this number tell you about the solution to the system? What does
it tell you about the solution of the corresponding homogeneous
system?
\item Find the solution to the system using any method described
in previous chapters.
\item Calculate the determinant of the matrix $A_1$, where $A_1$
is the matrix found by replacing the first column in $A$ with the
column vector $b$.
\item Calculate the determinants $A_2$ and $A_3$, where $A_2$ and
$A_3$ are found in the same way as $A_1$, except the 2nd and 3rd
columns are replaced respectively.
\item Calculate the three values, $\frac{|A_1|}{|A|}$,
$\frac{|A_2|}{|A|}$, and $\frac{|A_3|}{|A|}$. Compare these values
to the solution found. What do you notice?
\end{enumerate}
\end{enumerate}
\noindent \textbf{Solution} \begin{description}\item (a)
det$(A)=-82$. Because the det$\neq0$ we know that you
can obtain a solution to the system and we also know that the only
solution to the corresponding homogeneous system is $x=0$, $y=0$
and $z=0$ ( the trival/zero solution).
\item (b)
\noindent The solution to the system is ${\bf
x}=(-3,\frac{1}{2},-1)$.
\item (c)
$A=\left [ \begin{array}{rrr}
                      1&2&5\\
                      4&2&3\\
                      2&-4&1\end{array} \right]$ \quad \quad
${\bf b}=\left [ \begin{array}{r}
                              -7\\
                              -14\\
                              -9\end{array} \right]$ \quad \quad
$A_{1}=\left [\begin{array}{rrr}
                            -7&2&5\\
                            -14&2&3\\
                            -9&-4&1\end{array} \right]$\\

det$(A_1)=246$
\item (d)
$A_{2}=\left [\begin{array}{rrr}
                            1&-7&5\\
                            4&-14&3\\
                            2&-9&1\end{array} \right]$ \quad
$A_{3}=\left [\begin{array}{rrr}
                            1&2&-7\\
                            4&2&-14\\
                            2&-4&-9\end{array} \right]$\\

det($A_{2}$)$=-41$.\\
det($A_{3}$)$=82$.
\item (e)
$\frac{|A_{1}|}{|A|}=\frac{246}{-82}=-3$\\ \\
$\frac{|A_{2}|}{|A|}=\frac{-41}{-82}=-\frac{1}{2}$\\ \\
$\frac{|A_{3}|}{|A|}=\frac{82}{-82}=-1$\\ \\
\noindent These three values are equivalent to the found solution.
\end{description}

\newpage
\markboth{}{} 