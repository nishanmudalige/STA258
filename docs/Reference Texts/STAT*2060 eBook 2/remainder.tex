Let $f(x)=a_0+a_1x+a_2x^2+\cdots +a_nx^n$ be a polynomial.  The
number $r$ is said to be a $root$ of the polynomial $f$ if
\[f(r)=0,\ {\rm or},\  a_0+a_1r+a_2r^2+\cdots +a_nr^n=0 \ \ . \] The
Remainder theorem states that $r$ is a root of $f$ if and only if
$(x-r)$ divides $f$, i.e., if and only if $(x-r)$ is a factor of
$f$.

You can use this to factorize cubic polynomials by ``guessing" a
root.  That is to say, you inspect the polynomial and try to think
of a value of $r$ which will make $f(r)=0$.  For an arbitrary
cubic polynomial this would be difficult to do, if not impossible.
However, in this course we will be dealing with polynomials many
of whose roots are integers, and so it may be possible to guess a
root.  There is no systematic way to decide upon a good guess. You
have to proceed by trial and error and test each guess by seeing
whether $f(r)=0$.  One useful observation is that the constant
term of the polynomial $f$ is equal to the product of the roots of
$f$.  Thus your guess must be a divisor of the constant term.  If
$f$ has been set up so that all its roots are integers then this
can give you a good lead as to what to guess.

For example, good guesses for a root of $x^3+x^2-5x-2$ are 1, -1,
2, -2.  In fact $x=2$ is a root because
\[
2^3+2^2-5\cdot 2 -2 = 8+4-10-2 = 0 \ \ .
\]
This means that $(x-2)$ divides $x^3+x^2-5x-2$.  We determine the
other roots by doing the division and then factorizing the
resulting quadratic.  The division procedure is just as for ``long
division" that you do in grade school.  First you divide the $x$
of the $(x-2)$ term into $x^3$.  It goes $x^2$ times, so you
multiply the $(x-2)$ by $x^2$, write the result beneath the
$x^3+x^2-5x-2$, subtract and ``bring down" the $-5x$:

\begin{center}
\newdimen\digitwidth\
  \settowidth\digitwidth{0}
  \def~{\hspace{\digitwidth}}
\def\divrule#1#2{%
  \noalign{\moveright#1\digitwidth%
      \vbox{\hrule width#2\digitwidth}}}
$(x-2)$\,\begin{tabular}[b]{@{}l@{}}
      ~~~~~\,$x^2$ \\ \hline
      \big)\begin{tabular}[t]{@{}l@{}}
        $x^3+\,x^2-5x-2$ \\
	$x^3-2x^2$  \\  \divrule{0}{7}
	~~~~\,$3x^2-5x$ %\\
%	~~52 \\ \divrule{2}{3}
%	~~125\\
%	~~117\\ \divrule{2}{3}
%	~~~~8
\end{tabular}
\end{tabular}
\end{center}

Dividing the $x$ of the $(x-2)$ into the $3x^2$ gives $3x$.  We write this at the
top, multiply $(x-2)$ by $3x^2$, write this at the bottom, subtract and ``bring
down" the $-2$:

\begin{center}
\newdimen\digitwidth\
  \settowidth\digitwidth{0}
  \def~{\hspace{\digitwidth}}
\def\divrule#1#2{%
  \noalign{\moveright#1\digitwidth%
      \vbox{\hrule width#2\digitwidth}}}
$(x-2)$\,\begin{tabular}[b]{@{}l@{}}
      ~~~~~\,$x^2+3x$ \\ \hline
      \big)\begin{tabular}[t]{@{}l@{}}
        $x^3+\,x^2-5x-2$ \\
	$x^3-2x^2$  \\  \divrule{0}{7}
	~~~~\,$3x^2-5x$ \\
	~~~~\,$3x^2-6x$ \\ \divrule{4}{8}
	~~~~~~~~~~\,$x-2$\\
%	~~117\\ \divrule{2}{3}
%	~~~~8
\end{tabular}
\end{tabular}
\end{center}

Finally, the $(x-2)$ goes into the $(x-2)$ exactly once.  Thus the quotient
polynomial is $x^2+3x+1$, and we have
\[x^3+x^2-5x-2 = (x-2)(x^2+3x+1) \ \ .\]
The roots of $x^2+3x+1$ are found to be $\frac{-3+\sqrt{5}}{2}$ and
$\frac{-3-\sqrt{5}}{2}$.

Another example is to find the roots of $3x^3-5x^2-34x+24$.  If there is an integer root it will divide 24.  There are many possibilities: plus or minus
1, 2, 3, 4, 6, 8, 12, 24.  After some trial and error we find
\[
3(-3)^3-5(-3)^2-34(-3)+24=-81-45+102+24=0 \ \ .
\]
This means that $x=-3$ is a root and $(x+3)$ is a factor.  Doing the
long division we otain:

\begin{center}
\newdimen\digitwidth\
  \settowidth\digitwidth{0}
  \def~{\hspace{\digitwidth}}
\def\divrule#1#2{%
  \noalign{\moveright#1\digitwidth%
      \vbox{\hrule width#2\digitwidth}}}
$(x+3)$\,\begin{tabular}[b]{@{}l@{}}
      ~~~~~~$3x^2-14x+~8$ \\ \hline
      \big)\begin{tabular}[t]{@{}l@{}}
        $3x^3-5x^2-34x+24$ \\
	$3x^3+9x^2$  \\  \divrule{0}{7}
	~~~~\,$-14x^2-34x$ \\
	~~~~\,$-14x^2-42x$ \\ \divrule{4}{12}
	~~~~~~~~~~\,$8x+24$\\
	~~~~~~~~~~\,$8x+24$\\ \divrule{10}{8}
	~~~~~~~~~~~~~~~~0
\end{tabular}
\end{tabular}
\end{center}

Finally,
\[
3x^2-14x+8=(3x-2)(x-4)
\]
and so
\[
3x^3-5x^2-34x+24=(x+3)(3x-2)(x-4) \ \ .
\]
The roots are therefore -3, 2/3, 4.

In conclusion, note that it is not necessary to factorize cubic equations by the
method given here.  Instead,you can use the properties of determinants.






%\end{document}


%\bye
