\baselineskip = 16pt plus 3pt minus 3pt
\begin{center}
{\Large \bf Preface}
\end{center}

\noindent Matrix algebra is useful in many areas of Science, and is especially
useful in Statistics.
It can be used to express statistical estimation succinctly and explicitly.
An example where this occurs is in
regression analysis where one can express least squares estimation
in a single expression, otherwise, one needs a paragraph to write out
all of the conditions.  Another example is in
multivariate data analysis where concepts would be very difficult
to explain without using matrix algebra.
Recently, there are situations where we observe data
in matrix form so that the usage of matrix algebra is unavoidable.

Although there are many matrix algebra texts, the intended audience is usually
for mathematicians or engineers, consequently the material in such texts reflect
the needs of the intended audiences and
therefore
it is difficult for Statistics students without the necessary background
to grasp the contents.
For these reasons, there is a need for a basic matrix algebra text
for undergraduate students who want to study Statistics.
In particular, this book attempts to explain the basic concepts using examples and exercises without proofs.
We find this approach to be more effective in helping students use the language of matrix algebra
and teachers to explain more concepts than a course that provide all proofs,
in one semester.

We would like to thank Kwangrae Kim and Kyudong Cho for their help in \LaTeX --typesetting. 