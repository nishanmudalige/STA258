\begin{enumerate}
\item Find the general solution and 2 particular solutions of the
following equations and systems:
\begin{enumerate}
\item $x-2y=6-z$
\item $\frac{5}{6}(x_2-x_1)=5x_3+1 $
\item $x-\frac{w+1}{2}=z-3y; \  y-2z=\frac{3}{2}+x$
\item $x-y+3z=1; \  \frac{1}{x}-\frac{2}{y}=0; \  z=\sqrt{2}-x$.

\noindent \textbf{Solution} \begin{description}\item (a)
To solve, assign arbitrary values for $x$ and $y$. Let
$x=t$ and $y=s$. Now solve for z.
\\
\noindent $z=-x+2y+6$
\\
\noindent $z=-t+2s+6$
\\
\noindent General Solution Set: \{ ($t$,$s$,$-t+2s+6$); $t$,$s\in$
{\tebbb R}\}
\\
\noindent Particular Solution Set \#$1$: \ Arbitrarily choose
$x=0$ and $y=0$ thus $z=6$, and the particular solution set is
\{($0$,$0$,$6$) \}
\\
\noindent Particular Solution Set \#$2$: \ Let $x=1$,$y=1$, thus
$z=7$ and \{($1$,$1$,$7$)\}
\item (b) To solve, assign arbitrary values to $x_{1}$, and
$x_{2}$. Let $x_{1}=t$ and let $x_{2}=s$. Now solve for $x_{3}$.
\\
\noindent $\frac{5}{6}(s-t)=5x_{3}+1$
\\
\noindent $\frac{5}{6}s-\frac{5}{6}t-1=5x_{3}$
\\
\noindent $\frac{1}{6}s-\frac{1}{6}t-\frac{1}{5}=x_{3}$
\\
\noindent General Solution Set: \
\{($t$,$s$,$\frac{1}{6}s-\frac{1}{6}t-\frac{1}{5}$): $s$,$t\in$
{\tebbb R} \}
\\
\noindent Particular Solution Set \#$1$: \
\{($0$,$0$,$-\frac{1}{5}$)\}
\\
\noindent Particular Solution Set \#$2$: \
\{($6$,$6$,$-\frac{1}{5}$)\}
\item (c)
\noindent Rearrange the two equations such that:
\\
\noindent Equation \#$1$: \ $w-2x-6y+2z=-1$
\\
\noindent Equation \#$2$: \ $2x-2y+4z=-3$
\\
\noindent Now, set up as an augmented matrix and put it in reduced
row-echelon form:
\\
$$\left[ \begin{array}{rrrr|r}
                     1&-2&-6&2&-1\\
                     0&2&-2&4&-3 \end{array} \right ]\quad
                     \leadsto
\left [ \begin{array}{rrrr|r}
                     1&0&-8&6&-4 \\
                     0&1&-1&2&-\frac{3}{2} \end{array} \right ]$$
\\
\noindent $w$ and $x$ are leading variables and $y$, $z$ are free
variables. Let $y=s$, $z=t$, where $s$ and $t$ are arbitrary
values.  Thus
\\
\noindent $w=-4+8s-6t$
\\
\noindent $x=-\frac{3}{2}+s-2t$
\\
\noindent General Solution Set:
\{($-4+8s-6t$,$-\frac{3}{2}+s-2t$,$s$,$t$): $s$,$t\in {\tebbb R}$
\}
\\
\noindent Particular Solution Set \# $1$:
\{($-4$,$-\frac{3}{2}$,$0$,$0$)\}
\\
\noindent Particular Solution Set \# $2$:
\{($4$,$-\frac{1}{2}$,$1$,$0$)\}
\item (d) The easiest way to find a solution for this problem is
to transform the question into an augmented matrix and then row
reduce.
\\
$$\left[ \begin{array}{rrr|r}
                     1&-1&3&1\\
                     2&-1&0&0\\
                     1&0&1&\sqrt{2} \end{array} \right ]\quad
                     \leadsto
\left [ \begin{array}{rrr|r}
                     1&0&0&\frac{3\sqrt{2}-1}{4}\\
                     0&1&0&\frac{3\sqrt{2}-1}{2}\\
                     0&0&1&\frac{\sqrt{2} +1}{4} \end{array} \right ]$$
\\
\noindent Since all variables are leading, this system has only
one solution, namely \\ $\{ \left(
\frac{3\sqrt{2}-1}{4},\frac{3\sqrt{2}-1}{2},\frac{\sqrt{2}+1}{4}
\right) \}$.
\end{description}
\end{enumerate}

\item Find the augmented matrix for the following systems:
$$\rm{(a)} \begin{array}{rrl} 8+\frac{2y}{2-x}&=&10\\
y+\frac{1}{2}-x&=&\frac{1}{3} \end{array} \quad \quad \quad
\rm{(b)} \begin{array}{rrl} \frac{x}{2}+\frac{1}{2}z&=&2(x-y)\\
\sqrt{2}-\frac{z}{3+y}&=&0\\ x-\frac{3}{7}y&=&4(\frac{1}{3}y+1)
\end{array}.$$
\noindent \textbf{Solution}
\\
$\rm{(a)} \ \    \left [
\begin{array}{rr|r}
                     1&1&2\\
                     6&-6&1\end{array} \right ]$

${\rm (b)} \ \  \left [ \begin{array}{rrr|r}
                     3&-4&-1&0\\
                     0&-\sqrt{2}&1&3\sqrt{2}\\
                     21&-37&0&84\end{array}\right]$
\item Find a system of linear equations corresponding to each of
the following augmented matrices. $$\rm{(a)} \left [
\begin{array}{rrrrrcr}
                3&4&1&0&2&\vline&-11\\
                0&0&-24&1&0&\vline&13\\
                1&-6&0&4&1&\vline&20\\ \end{array} \right ]\quad \quad
\rm{(b)} \left [ \begin{array}{rrrrcr}
                1&-9&3&3&\vline&0\\
                9&2&0&-4&\vline&0\\
                3&-7&1&8&\vline&0\\
                -3&9&-1&1&\vline&0 \end{array} \right ]$$
\noindent \textbf{Solution}
$$\rm{ (a)}\ \  \begin{array}{rrl}
                        3x_{1}+4x_{2}+x_{3}+2x_{5}&=&-11\\
                        -24x_{3}+x_{4}&=&13\\
                        x_{1}-6x_{2}+4x_{4}+x_{5}&=&20\end{array}\quad
\rm{ (b)} \ \ \begin{array}{rrl}
                        x_{1}-9x_{2}+3x_{3}+3x_{4}&=&0\\
                        9x_{1}+2x_{2}-4x_{4}&=&0\\
                        3x_{1}-7x_{2}+x_{3}+8x_{4}&=&0\\
                        -3x_1+9x_2-x_3+x_4&=&0 \end{array}$$
\item Solve the following systems.

$\rm{(a)} \left [ \begin{array}{rrrrcr}
                    1&-2&0&1&\vline&3\\
                    0&0&-2&0&\vline&2\\
                    0&0&0&1&\vline&-5\\
                    0&0&0&0&\vline&1 \end{array} \right ] \quad
                    \quad \quad
\rm{(b)} \left [ \begin{array}{rrrrrcr}
                    -1&1&0&1&3&\vline&2\\
                    0&3&0&0&2&\vline&0\\
                    0&0&1&4&0&\vline&-1\\
                    0&0&0&0&1&\vline&6 \end{array} \right ]$

(c) $ \begin{array}{rrr} x-2y+z+w&=&3\\ 2y+z&=&4\\ z-4w&=&2\\
w+3&=&0 \end{array}$

\noindent \textbf{Solution} \begin{description}\item (a)
Since matrix (a) is in row reduced form we need only to
examine it to solve the system. Since the final row of the matrix
states that $0x_{1}+0x_{2}+0x_{3}+0x_{4}+0x_{5}=1\ \Rightarrow\
0=1$, we conclude that this system is inconsistent (the system has
no solution).
\item (b)
$$\left [\begin{array}{rrrrrcr}
                      -1&1&0&1&3&\vline&2\\
                      0&3&0&0&2&\vline&0\\
                      0&0&1&4&0&\vline&-1\\
                      0&0&0&0&1&\vline&6\end{array} \right]\quad \leadsto \quad
  \left [\begin{array}{rrrrrcr}
                      1&-1&0&-1&-3&\vline&-2\\
                      0&1&0&0&\frac{2}{3}&\vline&0\\
                      0&0&1&4&0&\vline&-1\\
                      0&0&0&0&1&\vline&6\end{array} \right]$$
\\
\noindent Let ``$x_4$" equal $s$, and then back substitute
starting from the bottom. The solution set is therefore
$\{12+s,-4,-1-4s,s,6\}$.
\item (c)
$$\left [\begin{array}{rrrrcr}
                      1&-2&1&1&\vline&3\\
                      0&2&1&0&\vline&4\\
                      0&0&1&-4&\vline&2\\
                      0&0&0&1&\vline&-3\end{array} \right]\quad \leadsto \quad
  \left [\begin{array}{rrrrcr}
                      1&-2&1&1&\vline&3\\
                      0&1&\frac{1}{2}&0&\vline&2\\
                      0&0&1&-4&\vline&2\\
                      0&0&0&1&\vline&-3\end{array} \right]$$
\\
\noindent We find that $w=-3$ from the matrix. We know from the
third row of the matrix that $z-4w=2$. We sub $w=-3$ into this
equation to obtain $z=-10$. Examining the second line of the
matrix we find that $y+\frac{1}{2}z=2$. Sub $z=-10$ into the
equation and we get $y=7$. The first row of the matrix says
$x-2y+z+w=3$. From this equation we conclude $x=30$. The solution
set is therefore \{30,7,-10,-3\}.
\end{description}
\item Find the row-echelon and reduced row-echelon form of the
following matrices.

(a) $\left [ \begin{array}{rrrrcr}
                    1&5&2&-3&\vline&2\\
                    1&0&4&2&\vline&1\\
                    -1&0&0&2&\vline&3 \end{array} \right ] \quad \quad \quad
 \rm{(b)} \left [ \begin{array}{rrrcr}
                    1&3&2&\vline&1\\
                    4&1&-3&\vline&5\\
                    -1&8&9&\vline&-2 \end{array} \right ]$

(c) $\left [ \begin{array}{rrrrcr}
                    2&1&1&3&\vline&-4\\
                    0&5&1&2&\vline&3\\
                    -1&2&-3&7&\vline&1\\
                    1&8&-1&12&\vline&0\\
                    2&6&2&5&\vline&-1\end{array} \right ]$

\noindent \textbf{Solution}

\begin{description} \item (a)
$\left [\begin{array}{rrrrcr}
                     1&5&2&-3&\vline&2\\
                     1&0&4&2&\vline&1\\
                     -1&0&0&2&\vline&3 \end{array} \right ]
                     \stackrel{\stackrel{R_2-R_1}{R_3+R_1}}{\leadsto}$
$\left [\begin{array}{rrrrcr}
                     1&5&2&-3&\vline&2\\
                     0&-5&2&5&\vline&-1\\
                     0&5&2&-1&\vline&5 \end{array} \right ]
                     \stackrel{\stackrel{R_3+R_2}{-\frac{1}{5}R_2}}{\leadsto}$

$\left [\begin{array}{rrrrcr}
                     1&5&2&-3&\vline&2\\
                     0&1&-\frac{2}{5}&-1&\vline&\frac{1}{5}\\
                     0&0&4&4&\vline&4 \end{array} \right ]
                     \stackrel{\frac{1}{4}R_3}{\leadsto}$
$\left [\begin{array}{rrrrcr}
                     1&5&2&-3&\vline&2\\
                     0&1&-\frac{2}{5}&-1&\vline&\frac{1}{5}\\
                     0&0&1&1&\vline&1 \end{array} \right ]$\\


 The above matrix is in row-echelon form. Reduced row-echelon form
 is obtained in a few more steps.

 $\left [\begin{array}{rrrrcr}
                     1&5&2&-3&\vline&2\\
                     0&1&-\frac{2}{5}&-1&\vline&\frac{1}{5}\\
                     0&0&1&1&\vline&1 \end{array} \right ]                     \stackrel{R_1-5R_2}{\leadsto}$
$\left [\begin{array}{rrrrcr}
                     1&0&4&2&\vline&1\\
                     0&1&-\frac{2}{5}&-1&\vline&\frac{1}{5}\\
                     0&0&1&1&\vline&1 \end{array} \right ]
                     \stackrel{\stackrel{R_1-4R_3}{R_2+\frac{2}{5}R_3}}{\leadsto}$

               $\left [\begin{array}{rrrrcr}
                     1&0&0&-2&\vline&-3\\
                     0&1&0&-\frac{3}{5}&\vline&\frac{3}{5}\\
                     0&0&1&1&\vline&1 \end{array} \right ]$
\item (b)
 $\rm{\textrm{Row-Echelon Form}} \left [\begin{array}{rrrcr}
                                               1&3&2&\vline&1\\
                                               0&1&1&\vline&-\frac{1}{11}\\
                                               0&0&0&\vline&0\end{array}\right]$

 $\rm{\textrm{Reduced Row-Echelon Form}}\left [\begin{array}{rrrcr}
                                                   1&0&-1&\vline&\frac{14}{11}\\
                                                   0&1&1&\vline&-\frac{1}{11}\\
                                                   0&0&0&\vline&0\end{array}\right]$
\item (c)
$\rm{\textrm{Row-Echelon Form}} \left [\begin{array}{rrrrcr}
                                               1&\frac{1}{2}&\frac{1}{2}&\frac{3}{2}&\vline&-2\\
                                               0&1&\frac{1}{5}&\frac{2}{5}&\vline&\frac{3}{5}\\
                                               0&0&1&-\frac{5}{2}&\vline&\frac{5}{6}\\
                                               0&0&0&0&\vline&0\\
                                               0&0&0&0&\vline&0\end{array}\right]$

$\rm{\textrm{Reduced Row-Echelon Form}}\left[\begin{array}{rrrrcr}
                                                   1&0&0&\frac{23}{10}&\vline&-\frac{79}{30}\\
                                                   0&1&0&\frac{9}{10}&\vline&\frac{13}{30}\\
                                                   0&0&1&-\frac{5}{2}&\vline&\frac{5}{6}\\
                                                   0&0&0&0&\vline&0\\
                                                   0&0&0&0&\vline&0\end{array}\right]$
\end{description}
\item Solve the following systems by reducing the augmented matrix to row-echelon or reduced
row-echelon form. Classify the solution found as inconsistent or
consistent. If the solution is consistent, how many solutions are
there?

$
\rm{(a)} \begin{array}{rrr} 3x+2y-1z&=&-5\\ 4x+5y+z&=&-9\\
-x+3y+7z&=&4 \end{array} \quad \quad \quad \quad \quad \quad
\rm{(b)} \begin{array}{rrr} 2x+y&=&-3\\x+4y+5z&=&0\\
2x+y&=&1\\3x+5y+5z&=&-7 \end{array}$

$\rm{(c)} \begin{array}{rrr} x_1+4x_3+2x_4&=&0\\
-2x_1-4x_2+x_3&=&1 \\ 7x_1+4x_3+4x_4&=&-2
 \end{array} \quad \quad \quad \quad
\rm{(d)} \begin{array}{rrr} y+2z&=&-1\\ 2x&=&3\\ y+3z&=&2\\
2x+z&=&6 \end{array}$

$\rm{(e)} \begin{array}{rrr} x_1-4x_2+2x_3&=&5\\
x_1+4x_2+2x_3&=&3\\ x_1-12x_2+2x_3&=&7\\ 3x_1-4x_2+6x_3&=&11\\
2x_1+4x_3&=&8\\ -8x_2&=&2
\end{array}$

\noindent \textbf{Solution}

\begin{description}\item (a)
\noindent Place the system into an augmented matrix and then row
reduce until you obtain the reduced row-echelon form of the
matrix.
\\
$\rm{\textrm{Reduced Row-Echelon Form}} \left
[\begin{array}{rrr|r}
                     1&0&0&1\\
                     0&1&0&-3\\
                     0&0&1&2\end{array}\right]$
\\
\noindent The solution is consistent. There is only one solution
to the system: $\{1,-3,2\}$.
\item (b)
Write the system in augmented matrix form and then row
reduce the matrix until it is in row-echelon form.

$\rm{\textrm{Row-Echelon Form}} \left[\begin{array}{rrr|r}
                                               1&\frac{1}{2}&0&-\frac{3}{2}\\
                                               0&1&\frac{10}{7}&\frac{3}{7}\\
                                               0&0&0&1\\
                                               0&0&0&0\end{array}\right]$
\\
\noindent From the row-echelon matrix, we determine that the
system is inconsistent, i.e., $0\neq 1$.
\item (c)
Once again, find the augmented matrix for the system.
Row reduce the matrix until it is in reduced row-echelon form.

$\rm{\textrm{Reduced Row-Echelon Form}}\left[\begin{array}{rrrr|r}
                                           1&0&0&\frac{1}{3}&-\frac{1}{3}\\
                                           0&1&0&-\frac{1}{16}&-\frac{1}{16}\\
                                           0&0&1&\frac{5}{12}&\frac{1}{12}\end{array}\right]$
\\
\noindent Examining the matrix we find that there is more than one
solution to the system.  Let $x_4=s$.  Then
$x_3=\frac{1}{12}-\frac{5}{12}s$,
$x_2=-\frac{1}{16}+\frac{1}{16}s$,
$x_1=-\frac{1}{3}-\frac{1}{3}s$. The solution is consistent.
\item (d)
\noindent Once again, find the augmented matrix for the system.
Row reduce the matrix until it is in reduced row-echelon form.

$\rm{\textrm{Reduced Row-Echelon Form}}\left[\begin{array}{rrr|r}
                                               1&0&0&\frac{3}{2}\\
                                               0&1&0&-7\\
                                               0&0&1&3\\
                                               0&0&0&0\end{array}\right]$
\\
\noindent Examining the matrix we find that there is only one
solution to the system.  The solution is consistent and is
$\{\frac{3}{2},-7,3\}$.
\item (e)
Row reducing the augmented matrix of the system we
obtain:

$\left[\begin{array}{rrr|r}
                    1&-4&2&5\\
                    0&1&0&-\frac{1}{4}\\
                    0&0&0&1\\
                    0&0&0&0\\
                    0&0&0&0\\
                    0&0&0&0\end{array}\right]$
\\

\noindent The system is inconsistent and has no solution.
\end{description}
\item There are three types of possible solutions to a system of
equations. Given any system that has the same number of unknowns
as the number of equations, how could you tell from the row-echelon form of the augmented
matrix when a solution is
inconsistent, infinite or unique? State the requirement for each
type of solution. If necessary include a generic matrix for each
type to illustrate the answer.

\noindent \textbf{Solution}

\noindent Every linear system of equations will have either no
solution (inconsistent), exactly one solution (consistent), or
infinitely many solutions (consistent).

\noindent A system is inconsistent if there is no solution. If the
row echelon form of the augmented matrix for the system is

$ \left[\begin{array}{rrrcr}
                    1&*&*&\vline&*\\
                    0&1&*&\vline&*\\
                    0&0&0&\vline&c\end{array}\right]$ where
                    $c\neq0$,

\noindent then the system is inconsistent.  The bottom row of the
matrix gives an equation $$0x_{1}+0x_{2}+0x_{3}=c\ (c\neq0),$$
which is impossible.  The asterisks in the matrix signify
arbitrary numbers (it doesn't matter what value the asterisk
takes).
\\
\noindent A system has infinitely many solutions if a solution is
possible and there is a free (or non-leading) variable.  You get
an infinite number of solutions because the free variable can take
an infinite number of values.  For example, if the row-echelon
form is

$\left[\begin{array}{rrrcr}
                    1&*&*&\vline&*\\
                    0&1&*&\vline&*\\
                    0&0&0&\vline&0\end{array}\right],$

then $x_{3}$ is a free variable.  You let $x_{3}=t$ (an arbitrary
variable) and solve for $x_{1}$, $x_{2}$ in terms of $t$.
\\
\noindent A system has a unique solution if it is consistent and
there are no free (non-leading) variables.  The row-echelon form
of the matrix must be,


$ \left[\begin{array}{rrrcr}
                    1&*&*&\vline&*\\
                    0&1&*&\vline&*\\
                    0&0&1&\vline&*\end{array}\right].$

The bottom row of the matrix gives a unique value for $x_{3}$, and
unique values for $x_{1}$, $x_{2}$ are found by back-substitution.

\item Using the answer from the previous question,
determine the necessary conditions on $a, b$ and $c$ for the
following systems to
\begin{enumerate}
\item [(i)] have a unique solution;
\item [(ii)] have an infinite number of solutions or
\item [(iii)] be inconsistent.
\end{enumerate}

$\rm{(a)} \begin{array}{rrr} ax+2y&=&-1\\ 3x+by&=&1 \end{array}
\quad \quad \quad \rm{(b)} \begin{array}{rrr} x_1-x_2+2x_3&=&a\\
x_1-x_3&=&b\\ -x_2+3x_3&=&c \end{array}$

\noindent \textbf{Solution}\begin{description}\item(a)
Represent the system in augmented matrix form and row
reduce.

$\left[\begin{array}{rrcr}
                    a&2&\vline&-1\\
                    3&b&\vline&1\end{array}\right]$
                    $\leadsto$
$\left[\begin{array}{rrcr}
                    3&b&\vline&1\\
                    a&2&\vline&-1\end{array}\right]$
$\leadsto$ $\dots$ $\leadsto$
 $\left[\begin{array}{rrcr}
                    1&\frac{b}{3}&\vline&\frac{1}{3}\\
                    0&6-ab&\vline&-3-a\end{array}\right]$
\\

\noindent (i) If $6-ab=0$, $-3-a \neq 0$ then the system is
inconsistent,
\\i.e. $ab=6$, $a \neq -3$ then no solution.
\\
\noindent (ii) If $6-ab=0$, $-3-a=0$ then $x_{2}$ is a free
variable and there are an infinite number of solutions,
\\i.e. $a=-3$, $b=-2$ then there are an infinite number of solutions.
\\
\noindent (iii) If $6-ab \neq 0$ then the solution is
unique,
\\i.e. $ab \neq 6$ then a unique solution.
\item (b)
Reduce to row-echelon form:
\\

$\left[\begin{array}{rrrcr}
                    1&-1&2&\vline&a\\
                    0&1&-3&\vline&-a+b\\
                    0&0&0&\vline&-a+b+c\end{array}\right]$
\\

\noindent (i) If $-a+b+c \neq 0$ then the system is inconsistent,
\\i.e. $a \neq b+c$ then no solution.
\\
\noindent (ii) If $-a+b+c=0$ then the system is consistent.  Since
$x_{3}$ is a free variable there will be an infinite number of
solutions.
\\i.e. if $a=b+c$ then an infinite number of solutions.
\\
\noindent (iii)It is not possible for this system to have a unique
solution.
\end{description}
\item Is the solution to the following system
consistent, and if it is consistent is it unique? Explain.
\begin{eqnarray*}
                x_1-4x_2-x_3+3x_4&=&0\\
                3x_1-2x_2+x_3-5x_4&=&0\\
                8x_1+9x_2+x_3-17x_4&=&0
\end{eqnarray*}

\noindent \textbf{Solution}

\noindent Yes the system is consistent.  One can see right away
that this is a homogeneous system and thus it has one very obvious
solution namely:, $x_{1}=0$, $x_{2}=0$, $x_{3}=0$, $x_{4}=0$.
Since there are three equations and four unknowns one is sure that
the above solution is not a unique one. There are infinitely many
solutions.

\item For what values of $a$ do the following systems have a unique solution,
an infinite number of solutions or no solutions. Explain.

$\rm{(a)} \begin{array}{rrr}
            (a-1)x+y&=&0\\
            x+(a-4)y&=&0 \end{array} \quad \quad \quad
\rm{(b)} \begin{array}{rrr}
            2x+5y-3z&=&0\\
            x+4y+2z&=&0\\
            3x+ay-z&=&0 \end{array} $

\noindent \textbf{Solution}\begin{description}\item (a)
Obtain the augmented matrix in reduced form:
$\left[ \begin{array}{rr|r}
            1&a-4&0\\
            0&a^2-5a+3&0\end{array}\right ] $

\noindent(i) The system has a unique solution when $a^2-5a+3\neq
0$. That is, $a\neq \frac{5\pm\sqrt{13}}{2}$ yields the zero
solution.
\\
\noindent (ii)If $a^2-5a+3=0$, then there are infinitely many
solutions.  This occurs when $a = \frac{5\pm\sqrt{13}}{2}$.
\\
\noindent (iii) There is no value of $a$ such that there is no
solution.
(See page $1-6$ of text) A homogeneous linear system is always consistent.
\item (b)
\noindent Represent the system as an augmented matrix. Row
reduce the matrix. From the row reduced matrix:

$\left[\begin{array}{rrrcr}
                    2&5&-3&\vline&0\\
                    0&3&7&\vline&0\\
                    0&0&7a-63&\vline&0\end{array}\right]$
\\
\noindent (i) Observe that if $7a-63\neq 0$ then the system has
one solution ($x=0$,$y=0$, $z=0$). Solving for $a$ we find that
there is a unique solution (the zero solution) when $a\neq
\frac{63}{7}$.
\\
\noindent (ii) We observe, from the above matrix that if $7a-63=0$
then the system has an infinite number of solutions. Rearrange the
equation. We can state that if $a=\frac{63}{7}$ then the system
has infinite solutions.
\\
\noindent (iii) There is no value of $a$ such that there is no
solution. (See page $1-6$ of text) A homogeneous linear system is
always consistent as it always has the zero solution.
\end{description}
\item \begin{enumerate}
\item Determine the necessary conditions on $a,b,c,d,e$ and $f$
for the following system to
\begin{enumerate}
\item [(i)] have a unique solution;
\item [(ii)] have an infinite number of solutions or
\item [(iii)] be inconsistent.
\end{enumerate}
\noindent
Assume that a, b, c, d, e, f are all non-zero (so that you can divide by them).
\begin{eqnarray*}
ax+by&=&e\\
cx+dy&=&f \end{eqnarray*}
\item From the conditions in part (a), determine without solving
the type of solutions for the following systems.

$\rm{(i)} \begin{array}{rrr} 2x+4y&=&\frac{1}{4}\\
\frac{1}{3}x+5y&=&-3 \end{array} \quad \quad \rm{(ii)}
\begin{array}{rrr} -\frac{1}{4}x_1+2x_2&=&-1\\ 2x_1-16x_2&=&8
\end{array}$ \\ $\rm{(iii)} \begin{array}{rrr} 4x-2y&=&-8\\
-6x-3y&=&12 \end{array}$

\end{enumerate}

\noindent \textbf{Solution} 

\begin{description}\item (a)
$\left [\begin{array}{rrcr}
                     a&b&\vline&e\\
                     c&d&\vline&f
                     \end{array} \right ]
                     \stackrel{\frac{1}{a}R_{1}}{\leadsto}$
$\left [\begin{array}{rrcr}
                     1&\frac{b}{a}&\vline& \frac{e}{a}\\
                     c&d&\vline&f
                     \end{array} \right ]
                     \stackrel{{R_2-c R_2}}{\leadsto}$
$\left [\begin{array}{rrcr}
                     1&\frac{b}{a}~~~~&\vline& \frac{e}{a}~~~~\\
                     0&d-\frac{bc}{a}&\vline&f-\frac{ec}{a}
                     \end{array} \right ]$\\

\noindent (i) There will be a unique solution if $d-\frac{bc}{a}\neq 0$, i.e. if $ad-bc\neq0$. The solution is then

$$y=\frac{f- \frac{ec}{a}}{d- \frac{bc}{a}}= \frac{af-ec}{ad-bc}$$
\\ and by back-substitution

$$x =\frac{e}{a}-\frac{b}{a} (\frac{af-ec}{ad-bc}) \\
= \frac{ead-ebc-baf+bec}{a(ad-bc)} \\
= \frac{ed-bf}{ad-bc}.$$

\noindent (ii) The system has an infinite number of solution if
$d-\frac{bc}{a}=0$ and $f-\frac{ec}{a}=0$, i.e. $ad=bc$ and $af=ec$, i.e. $\frac{a}{c}=\frac{b}{d}=\frac{e}{f}$.
\\
\noindent (iii) The system has no solution if
$d-\frac{bc}{a}=0$, $f-\frac{ec}{a}\neq0$, i.e. $ad=bc$, $af \neq ec$, or $\frac{a}{c}=\frac{b}{d} \neq \frac{e}{f}$.
\item (b)
(i) We know that $a=2$, $b=4$, $c=\frac{1}{3}$,$d=5$,
$e=\frac{1}{4}$, and $f=-3$. Since $ad-bc\neq 0$ we know that the
system has a unique solution.
\\
\noindent (ii) In this case, $a=-\frac{1}{4}$, $b=2$, $c=2$,
$d=-16$, $e=-1$, and $f=8$. Since $ad-bc=0$ we know that the
solution is not unique.  But $\frac{a}{c}=\frac{b}{d}=\frac{e}{f}$
implies that the solution is consistent and infinite.
\\
\noindent (iii) For this question, $a=4$, $b=-2$, $c=-6$, $d=-3$,
$e=-8$, and $f=12$. We determine that $ad-bc\neq 0$ and hence the
system has a unique solution.

\end{description}
\end{enumerate}
\newpage
\markboth{}{} 