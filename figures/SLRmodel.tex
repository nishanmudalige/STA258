\documentclass[tikz,border=5pt]{standalone}
\usepackage{tikz}
\usepackage{amsmath}
\begin{document}

\begin{figure}[h]
\centering
\begin{tikzpicture}[thick]
  \draw[->] (-1,0) -- (7,0) node[right] {$x$};
  \draw[->] (0,-1) -- (0,5) node[above] {$y$};
  \filldraw[blue] (0.5, 1.0) circle (2pt);
  \filldraw[blue] (1.5, 1.4) circle (2pt);
  \filldraw[blue] (1, 1.5) circle (2pt);
  \filldraw[blue] (2, 2.4) circle (2pt);
  \filldraw[blue] (2.5, 2.1) circle (2pt);
  \filldraw[blue] (3, 2.7) circle (2pt);
  \filldraw[blue] (3.5, 3.3) circle (2pt);
  \filldraw[blue] (4, 3.4) circle (2pt);
  \filldraw[blue] (4.5, 4.2) circle (2pt);
  \filldraw[blue] (5, 3.7) circle (2pt);
  \draw[red, thick, domain=0:6] plot (\x, {0.8*\x + 0.35}) node[right] {$\hat{y} = \hat{\beta}_0 + \hat{\beta}_1 \cdot x$};
  \foreach \x/\y in {
    0.5/1.0,
    1.5/1.4,
    1/1.5,
    2/2.4,
    2.5/2.1,
    3/2.7,
    3.5/3.3,
    4/3.4,
    4.5/4.2,
    5/3.7
  } {
    \draw[black] (\x,\y) -- (\x,{0.8*\x + 0.35});
  }
\end{tikzpicture}
\caption{An illustration of a simple linear regression model. The red line is the fitted regression line, and the full black vertical lines represent the residuals (SSE). The data points deviate from the line to show errors clearly. Note that the sum of residuals is necessarily $0$.}
\end{figure}

\end{document}
