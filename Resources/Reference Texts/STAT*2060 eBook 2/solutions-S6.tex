\begin{enumerate}
\item Find a basis and the dimension for the solution spaces of
the following homogeneous systems.
\ \

\begin{enumerate}
\item $\begin{array}{rrl}
x_2-x_4+2x_5&=&0\\ 2x_1+x_2+x_4&=&0\\ x_2-x_5&=&0\\
-x_1+x_2-x_4&=&0 \end{array}$

\ \

\item $\begin{array}{rrl}
x_1+3x_2-2x_5&=&0\\ -x_1+x_3+3x_5&=&0\\ x_2-x_3+2x_4&=&0\\
3x_3+2x_5&=&0 \end{array}$
\end{enumerate}
\noindent \textbf{Solution} \begin{description} \item(a)
There are four equations in five unknowns; therefore,
there are infinitely many solutions. After row reduction, the
following solution to the system is found.
\begin{eqnarray*}
\left [ \begin{array}{rrrrr|r}
                0&1&0&-1&2&0\\
                2&1&0&1&0&0\\
                0&1&0&0&-1&0\\
                -1&1&0&-1&0&0 \end{array} \right]\leadsto
\left [ \begin{array}{rrrrr|r}
                1&0&0&0&2&0\\
                0&1&0&0&-1&0\\
                0&0&0&1&-3&0\\
                0&0&0&0&0&0 \end{array} \right]
\end{eqnarray*}
\noindent Let $x_{3}=s$ and $x_5=t$ and solve for the other
variables. Thus $x_{4}=3t$, $x_{2}=t$ and $x_{1}=-2t$.

\noindent In matrix form:
\begin{eqnarray*}
\left [ \begin{array}{r}
                x_{1}\\
                x_{2}\\
                x_{3}\\
                x_{4}\\
                x_{5} \end{array} \right] \quad =\quad
\left [ \begin{array}{r}
                0\\
                0\\
                1\\
                0\\
                0\end{array} \right] \quad + \quad
\left [ \begin{array}{r}
                -2\\
                1\\
                0\\
                3\\
                1\end{array} \right].
\end{eqnarray*}
\noindent Rewriting the column vectors as row vectors for
convenience, a basis for the solution space is $\{(0,0,1,0,0),\
(-2,1,0,3,1)\}$ and the dimension is $2$.
\item (b) There are four equations in five unknowns therefore,
there are infinitely many solutions. After row reduction, the
following solution to the system is found.
\begin{eqnarray*}
\left [\begin{array}{rrrrr|r}
                1&3&0&0&-2&0\\
                -1&0&1&0&3&0\\
                0&1&-1&2&0&0\\
                0&0&3&0&2&0 \end{array} \right] \leadsto
\left [ \begin{array}{rrrrr|r}
                1&0&0&0&-\frac{7}{3}&0\\
                0&1&0&0&\frac{1}{9}&0\\
                0&0&1&0&\frac{2}{3}&0\\
                0&0&0&1&\frac{5}{18}&0 \end{array} \right]
\end{eqnarray*}
\noindent Let $x_5=t$ and solve for the other variables. Thus
$x_{1}=\frac{7}{3}t$, $x_{2}=-\frac{1}{9}t$, $x_3=-\frac{2}{3}t$,
and $x_4=-\frac{5}{18}t$.

\noindent In matrix form:
\begin{eqnarray*}
\left [ \begin{array}{r}
                x_{1}\\
                x_{2}\\
                x_{3}\\
                x_{4}\\
                x_{5} \end{array} \right] \quad = \quad
\left [ \begin{array}{r}
                \frac{7}{3}\\
                -\frac{1}{9}\\
                -\frac{2}{3}\\
                -\frac{5}{18}\\
                1\end{array} \right].
\end{eqnarray*}
\noindent Thus, a basis for the solution space is
$\{(\frac{7}{3},-\frac{1}{9},-\frac{2}{3},-\frac{5}{18},1)\}$, or
simply $\{(42,-2,-12,-5,18)\}$, and the dimension is $1$.
\end{description}

\item Determine if ${\bf b}$ lies in the column space of $A$. If
it does, express {\bf b} as a linear combination of the column
vectors. What is the solution to $A{\bf x}={\bf b}$? $$A=\left [
\begin{array}{rrr} 1&2&-1\\ 0&1&2\\ 1&-1&1
\end{array} \right ]; \quad \quad {\bf b}=\left [ \begin{array}{r}
-2\\1\\3\end{array} \right ].$$
\noindent \textbf{Solution}

\noindent Examine $A{\bf x}={\bf b}$:

$\left [ \begin{array}{rrr|r}
                       1&2&-1&-2\\
                       0&1&2&1\\
                       1&-1&1&3\end{array} \right] \Rightarrow $
$\left [ \begin{array}{rrr|r}
                       1&0&0&1\\
                       0&1&0&-1\\
                       0&0&1&1\end{array} \right]$

\noindent The solution to $A{\bf x}={\bf b}$ is:

${\bf x}=\left [ \begin{array}{r}
                               1\\
                               -1\\
                               1\end{array} \right]$

\noindent This tells us that ${\bf b}=({\bf c}_1-{\bf c}_2+{\bf
c}_3)$.  Therefore, ${\bf b}$ does lie in the column space of $A$.
\item Find a basis for the row space of $A$.
$$A=\left [ \begin{array}{rrrrr} 2&1&0&1&3\\ -3&2&1&4&1\\
1&-3&-1&-5&-4\\ 3&-2&-1&-4&-1\end{array} \right ].$$\\

\noindent \textbf{Solution}

\noindent We can find a basis for the row space of $A$ by finding
a basis for the row space of $A$ in row-echelon form.  A matrix's
rows do not have to contain leading ones to determine how many
nonzero rows (or basis vectors) we have, however.  Reducing $A$ we
obtain:

$R=\left [ \begin{array}{rrrrr}
                         2&1&0&1&3\\
                         0&7&2&11&11\\
                         0&0&0&0&0\\
                         0&0&0&0&0 \end{array} \right ]$

\noindent The non-zero row vectors of $R$ form a basis for the row
space of $R$ and hence form a basis for the row space of $A$.
These basis vectors are:

${\bf r_{1}}=(2,1,0,1,3)$

${\bf r_{2}}=(0,7,2,11,11)$

\item Find the rank and nullity of the following matrices.
For each matrix, what can be said about the solution to $A{\bf
x}={\bf b}$ and $A{\bf x}={\bf 0}$.
\begin{enumerate}
\item $A=\left [ \begin{array}{rrrr} 1&3&2&4\\-2&3&2&4\\1&4&5&-1\\
-1&5&1&13\end{array} \right ] $
\item  $A=\left [ \begin{array}{rrrr} 2&1&3&2\\0&3&2&-1\\3&-1&2&0\\
3&1&4&1\end{array} \right ] $
\end{enumerate}

\noindent \textbf{Solution} 

\begin{description}\item (a)
$A=\left [ \begin{array}{rrrr}
                         1&3&2&4\\-2&3&2&4\\1&4&5&-1\\
-1&5&1&13 \end{array} \right ]$
                         \quad $\stackrel{}{\leadsto}$\quad
$R=\left [ \begin{array}{rrrr}
                         1&0&0&0\\
                         0&1&0&\frac{22}{7}\\
                         0&0&1&-\frac{19}{7}\\
                         0&0&0&0 \end{array} \right ]$

\noindent The rank($A$) is the dimension of the row space of $A$
which is the same as the dimension of the row space of $R$. The
dimension of the row space is $3$ and therefore the rank is $3$.

\noindent To find the nullity you could find the dimension of the
solution space of $A$ (see question \# 6); however, we already
found the rank, so nullity($A$) $=$ $n$ $-$ rank($A$) $=4-3=1$.

Since $A$ is not invertible, $A{\bf x}={\bf b}$ may not be
consistent. $A{\bf x}={\bf 0}$ has the zero solution, but also
infinitely many solutions.

\item (b)
$A=\left [ \begin{array}{rrrr}
                         2&1&3&2\\0&3&2&-1\\3&-1&2&0\\
3&1&4&1 \end{array} \right ]$
                         \quad $\leadsto$\quad
$R=\left [ \begin{array}{rrrr}
                         1&0&0&0\\
                         0&1&0&0\\
                         0&0&1&0\\
                         0&0&0&1 \end{array} \right ]$

\noindent The rank($A$) is the dimension of the row space of $A$
which is the same as the dimension of the row space of $R$. The
dimension of the row space is $4$ and therefore the rank is $4$.

\noindent To find the nullity you could find the dimension of the
solution space of $A$ (see question \# 6); however, we already
found the rank, so nullity($A$) $=$ $n$ $-$ rank($A$) $=4-4=0$.

Since $A$ is invertible, $A{\bf x}={\bf b}$ is always consistent.
$A{\bf x}={\bf 0}$ has the zero solution, and it is the only
solution.
\end{description}

\item Given $A$, and a particular solution ${\bf x}_p$ to $A{\bf
x}={\bf b}$, find the general solution to $A{\bf x}={\bf b}$.
\begin{enumerate}
\item $x_p=\left [ \begin{array}{r}  1\\ 1\\ 0\\-2\\ 1 \end{array} \right ],
\quad A=\left [ \begin{array}{rrrrr} -1&1&-1&-5&3\\ 1&2&0&-1&5\\
2&1&1&4&2 \end{array} \right ]$
\item $x_p=\left [ \begin{array}{r} 2\\-1\\ 0\\1 \end{array} \right ], \quad
A=\left [ \begin{array}{rrrr} 2&1&0&3\\ 5&1&1&3\\1&-1&2&0\\
-2&-1&1&0
\end{array} \right ]$

\end{enumerate}

\noindent \textbf{Solution} \begin{description} \item (a)
First multiply $A{\bf x_{p}}$ to obtain ${\bf b}$. We
get:

${\bf b}=\left [ \begin{array}{r}
                       13\\
                       10\\
                       -3\end{array} \right ]$

\noindent To find the general solution, write the system in
augmented matrix form.

$\left [ \begin{array}{rrrrr|r}
                          -1&1&-1&-5&3&13\\ 1&2&0&-1&5&10\\ 2&1&1&4&2&-3 \end{array}\right ]$

\noindent Row reduce to obtain:

$\left [ \begin{array}{rrrrr|r}
                          1&0&\frac{2}{3}&3&-\frac{1}{3}&-\frac{16}{3}\\
                          0&1&-\frac{1}{3}&-2&\frac{8}{3}&\frac{23}{3}\\
                          0&0&0&0&0&0 \end{array} \right ]$

\noindent Thus we let $x_{3}=r$, $x_{4}=s$ and $x_{5}=t$ and we
obtain $x_{1}=-\frac{2}{3}r-3s+\frac{1}{3}t-\frac{16}{3}$ and
$x_{2}=\frac{1}{3}r+2s-\frac{8}{3}t+\frac{23}{3}$

\noindent The general solution is:

${\bf x} = \left[ \begin{array} {r}
                                 -\frac{16}{3}\\
                                 \frac{23}{3}\\
                                 0\\
                                 0\\
                                 0\end{array} \right]$
$+r \left[ \begin{array} {r}
                          -\frac{2}{3}\\
                          \frac{1}{3}\\
                          1\\
                          0\\
                          0\end{array} \right]$
$+s \left[ \begin{array} {r}
                          -3\\
                          2\\
                          0\\
                          1\\
                          0\end{array} \right]$
$+t\left[ \begin{array} {r}
                         \frac{1}{3}\\
                         -\frac{8}{3}\\
                         0\\
                         0\\
                         1\end{array} \right].$
\item (b)
First multiply $A{\bf x_{p}}$ to obtain ${\bf b}$. We
get:

${\bf b}=\left [ \begin{array}{r}
                        6\\
                        12\\
                        3 \\
                        -3 \end{array} \right ]$

\noindent Now that we have the vector ${\bf b}$ we can place the
system in augmented matrix form and solve.

$\left [ \begin{array}{rrrr|r}
                          2&1&0&3&6\\ 5&1&1&3&12\\1&-1&2&0&3\\ -2&-1&1&0&-3\end{array} \right ]$
                         \quad $\leadsto$\quad
$\left [ \begin{array}{rrrr|r}
                          1&0&0&-1&1\\
                          0&1&0&5&4\\
                          0&0&1&3&3\\
                          0&0&0&0&0\end{array} \right ]$

\noindent Thus we let $x_{4}=t$ and we obtain $x_{1}=1+t$,
$x_{2}=4-5t$, and $x_3=3-3t$.

\noindent The general solution is:

${\bf x} = \left[ \begin{array} {r}
                                 1\\
                                 4\\
                                 3\\
                                 0\end{array} \right]$
$+t \left[ \begin{array} {r}
                          1\\
                          -5\\
                          -3\\
                          1\end{array} \right].$
\end{description}

\item For what values of $a$ will the following matrices have nullity 0.

$(a)\left [ \begin{array}{rrcr} 3&2&a&1\\ -2&4&(a+1)&2\\ 1&0&3&2\\
\frac{a}{2}&6&12&5 \end{array} \right ]\quad (b)\left  [
\begin{array}{cccc} a^3&0&0&0\\ 3&(a+1)&0&0\\ 2&0&(a^2-3)&0\\
0&0&0&a-1 \end{array} \right ] \quad $

$(c)\left [ \begin{array}{rcrr} 1&3&5&2\\ -3&(4-a)&-1&2\\
0&-a&3&5\\ 6&5&3&-5\end{array} \right ]$

The nullity of a matrix is the dimension of the nullspace of $A$.  By the Dimension Theorem
this equals the number of columns of $A$ minus the number of non-zero rows in a row echelon (or the reduced row echelon) form for $A$.  Since the matrices here are square, to check that the nullity is zero, we look for no zeros down the diagonal of the echelon form.  Alternatively, we could use the fact that nullity $A=0 \leftrightarrow detA\neq0$ (theorem \ref{thm6.3}).

\noindent \textbf{Solution} \begin{description} \item (a)

\noindent To find the nullity, solve:

$\left [ \begin{array}{rrrr|r}
                          3&2&a&1&0\\
                          -2&4&(a+1)&2&0\\
                          1&0&3&2&0\\
                          0&6&12&5&0\end{array} \right ]$
                           \quad $\leadsto$\quad
$\left [ \begin{array}{rrrr|r}
                          1&0&3&2&0\\
                          0&2&(a-9)&-5&0\\
                          0&0&1&2(a+52)/147&0\\
                          0&0&0&a^2+27a-124&0\end{array} \right ]$

\noindent For the nullity to be zero $a^2+27a-124 = (a-4)(a+31)\neq
0$. Thus $a\neq 4,-31$ when the matrix has nullity equal
to zero.
\item (b)
Notice that this is a lower triangular matrix.
Therefore, the determinant is equal to the product of its entries
on the main diagonal.  When the determinant of a matrix is not
zero, its nullity is zero.

Then $(a^3)(a+1)(a^2-3)(a-1)\neq 0$ implies that the nullity is
zero. Solving for $a$, we see that this occurs when
$a=\neq-1,0,1,\sqrt{3},-\sqrt{3}$.
\item (c)
To find the nullity, solve:

$\left [ \begin{array}{rrrr|r}
                           1&3&5&2&0\\
                           -3&(4-a)&-1&2&0\\
                           0&-a&3&5&0\\
                           6&5&3&-5&0\end{array} \right ]$
                           \quad $\leadsto$\quad
$\left [ \begin{array}{rrrr|r}
                          1&3&5&2&0\\
                          0&13&11&3&0\\
                          0&0&16&14&0\\
                          0&0&0&53a-247&0\end{array} \right ]$

\noindent For the nullity to be zero $53a-247\neq 0$. Thus $a\neq
\frac{247}{53}$ when the following matrix has nullity equal to
zero.
\end{description}

\item Find the least square solution to $A{\bf{x}} = {\bf{b}}$
when
\begin{eqnarray*}
A = \begin{bmatrix} 1 & 2 & 1\\ 2 & 3 & 4\\ 1 & 0 &
1\\1&1&1 \end{bmatrix},\quad {\bf{b}} = \begin{bmatrix} 1\\2\\
\!\!\!-1 \\ 0\end{bmatrix}.
\end{eqnarray*}

\noindent \textbf{Solution}

\begin{eqnarray*}
\hat{{\bf{x}}} &=& \left( \begin{bmatrix} 1 & 2 & 1 & 1 \\ 2 & 3 &
0 & 1 \\ 1 & 4 & 1 & 1 \end{bmatrix}
\begin{bmatrix} 1 & 2 & 1 \\ 2 & 3 & 4 \\ 1 & 0 & 1 \\ 1 & 1 & 1 \end{bmatrix} \right)^{-1}
\begin{bmatrix} 1 & 2 & 1 & 1 \\ 2 & 3 & 0 & 1 \\ 1 & 4 & 1 & 1 \end{bmatrix}
\begin{bmatrix}1 & 2 & -1 & 0 \end{bmatrix} \\
&=& \frac{1}{24} \left[ \begin{array}{rrr}  41 & -6 & -19 \\ -6 &
12 & -6 \\ -19 & -6 & 17 \end{array} \right]
\begin{bmatrix} 1 & 2 & 1 & 1 \\ 2 & 3 & 0 & 1 \\ 1 & 4 & 1 & 1 \end{bmatrix}
\begin{bmatrix}1 & 2 & -1 & 0 \end{bmatrix} \\
&=& \left[ \begin{array}{r} \vspace{1mm} -\tfrac{3}{2} \\
\vspace{1mm} 1 \\ \tfrac{1}{2} \end{array} \right].
\end{eqnarray*}

\item Find the straight line that best fits the following sets of data points
$$
(-1,\ 7.1), (0,\ 4.8), (1,\ 2.7), (2,\ 1.1).
$$

\noindent \textbf{Solution}
\begin{eqnarray*}
\hat{\bf{a}} &=& \left( \begin{bmatrix} 1 & 1 & 1 & 1 \\ -1 & 0 &
1 & 2 \end{bmatrix}
\begin{bmatrix} 1 & -1\\1 & 0\\1& 1\\1 & 2\end{bmatrix}\right)^{-1}
\begin{bmatrix} 1 & 1 & 1 & 1 \\ -1 & 0 & 1 & 2 \end{bmatrix}
\begin{bmatrix} 7.1 \\ 4.8 \\ 2.7 \\ 1.1 \end{bmatrix}\\
&=&\frac{1}{20} \left[ \begin{array}{rr} 6 & -2 \\-2 & 4
\end{array} \right]
\begin{bmatrix}1 & 1 & 1 & 1\\-1 & 0 & 1 & 2 \end{bmatrix}
\begin{bmatrix} 7.1 \\ 4.8 \\ 2.7 \\ 1.1 \end{bmatrix} \\
&=& \left[ \begin{array}{r}4.93 \\ -2.01 \end{array} \right].
\end{eqnarray*}

\item Find the quadratic polynomial that best fits the following sets of data points
$$
(-1,\ -0.9), (0,\ -1.1), (1,\ 0.8), (2,\ 5.2).
$$

\noindent \textbf{Solution}

If the equation of the quadratic is
$y= a_0 + a_1x + a_2x^2$, then we want the least square solution
to be
\begin{align*}
-0.9 &= a_0 - a_1 + a_2\\
-1.1 &= a_0\\
0.8 &= a_0 + a_1 + a_2\\
5.2 &= a_0 + 2a_1 + 4a_2
\end{align*}
that is,
\begin{align*}
\left[ \begin{array}{rrr} 1 & -1 & 1 \\ 1 & 0 & 0 \\ 1 & 1 & 1 \\
1 & 2 & 4 \end{array} \right]
\begin{bmatrix} a_0 \\ a_1 \\ a_2 \end{bmatrix} = \left[ \begin{array}{r} -0.9\\-1.1\\0.8\\5.2 \end{array} \right].
\end{align*}
The solution is
\begin{align*}
\hat{\bf{a}} &= \left( \left[ \begin{array}{rrrr} 1 & 1 & 1 & 1
\\ -1 & 0 & 1 & 2 \\ 1 & 0 & 1 & 4 \end{array}\right]
\begin{bmatrix} 1 & -1  & 1 \\1 & 0 & 0\\1& 1 & 1 \\1 & 2 & 4\end{bmatrix}\right)^{-1}
\left[ \begin{array}{rrrr} 1 & 1 & 1 & 1 \\ -1 & 0 & 1 & 2 \\ 1 &
0 & 1 & 4 \end{array}\right]
\left[ \begin{array}{r} -0.9 \\ -1.1 \\ 0.8 \\ 5.2 \end{array} \right]\\
&=\frac{1}{80} \left[ \begin{array}{rrr} 44 & 12 & -20 \\ 12 &
36 & -20 \\ -20 & -20 & 20 \end{array} \right] \left[
\begin{array}{rrrr} 1 & 1 & 1 & 1 \\ -1 & 0 & 1 & 2 \\ 1 & 0 & 1
& 4 \end{array}\right]
\left[ \begin{array}{r} -0.9 \\ -1.1 \\ 0.8 \\ 5.2 \end{array} \right]\\
&= \left[ \begin{array}{r} -1.16 \\ 0.87 \\ 1.15 \end{array}
\right].
\end{align*}
\item If $\{(2,-1,-1,4)+s(a,b,-1,1);s\in \mbox{\tebbb R}\}$ is the
general solution to the system
$$\begin{array}{rcl}x-2y+3z-w&=&c\\x+2z+4w&=&d\\y-z+2w&=&e\end{array}$$
then find $a,b,c,d,e$.

\noindent \textbf{Solution}

The point of this question is to
understand the significance of the so-called Representation
Theorem. According to this theorem, the solution to a
non-homogeneous system can be written as a particular solution (in
this case $(x,y,z,w)=(2,-1,-1,4)$) plus the set of solutions to the
associated homogeneous system. This would mean that $(a,b,-1,1)$
is a solution to the system in which all the right hand sides have
been replaced by zero.

Since (2,-1,-1,4) is a particular solution,

$\begin{array}{rcl}1\cdot 2-2\cdot (-1)+3\cdot (-1)-1\cdot4&=&c\\
1\cdot 2+2\cdot (-1)+4\cdot (4)&=&d\\
1\cdot (-1)-1\cdot (-1)+2\cdot (4)&=&e\end{array}$

i.e. $c=-3,\,d=16,\,e=8.$

Reducing the matrix of the associated homogeneous system:

$\left[\begin{array}{rrrr|r}1&-2&3&-1&0\\1&0&2&4&0\\0&1&-1&2&0\end{array}\right]\leadsto\ldots\leadsto
\left[\begin{array}{rrrr|r}1&0&0&2&0\\0&1&0&3&0\\0&0&1&1&0\end{array}\right].$

$w$ is the free variable, let $w=t$ and $x=-2t,\,y=-3t,\,z=-t.$
Any value of $w$ yields a solution to the associated homogeneous
system. Let $w=1$ (given in general solution). Then $x=-2,\,y=-3$.
The resulting vector is $(-2,-3,-1,1)$, so $a=-2,\,b=-3.$
\end{enumerate}
\newpage
\markboth{}{}