\subsection{Exercises}

\begin{enumerate}
\item  Consider the following probability distribution for $x=0,1,\ldots  , 5$:
$$p(x)=
\left(
\begin{array}{c}
5 \\
x
\end{array}
\right)
0.7^x 0.3^{5-x}
$$

\iffalse
\begin{center}
\begin{tabular}{ccccccc}\hline
X             &2       &4        &6        &8       &10\\ \hline
$p(x)$        &0.1c    &0.2c     &0.4c     &0.2c    &0.1c\\ \hline
\end{tabular}
\end{center}
\fi
\begin{enumerate}
\item Is $x$ a discrete or a continuous random variable?
\item what is the name of the probability distribution?
\item Graph the probability distribution.
\item Find the mean and standard deviation of $x$.
\item Show the mean and the 2-standard-deviation interval on each side of the mean on the graph you draw in part c.
\end{enumerate}

\item  If $x$ is a binomial random variable with $n=3$ and $p=0.3$.
\iffalse
\begin{center}
\begin{tabular}{ccccccc}\hline
X             &2       &3        &5        &8       &10\\ \hline
$p(x)$        &0.15    &0.10     &-        &0.25    &0.25\\ \hline
\end{tabular}
\end{center}
\fi

\begin{enumerate}
\item Calculate the value of $p(x)$,$x=0,1,2,3$.
\item Using the answer from part a,give the probability distribution for $x$ in tabular form.
\end{enumerate}

\item  Suppose $x$ is a binomial random variable,computer $p(x)$ ro each of the following cases:

\begin{enumerate}
\item $n=5,x=1,p=0.2$.
\item $n=3,x=0,p=0.7$.
\item $n=4,x=2,q=0.6$.
\end{enumerate}

\item  If $x$ is a binomial random variable,calculate $\mu$ and $\sigma$ ro for each of the following:

\begin{enumerate}
\item $n=25,p=0.2$.
\item $n=80,p=0.7$.
\item $n=1000,q=0.9$.
\end{enumerate}
\iffalse
\item Consider the probability distribution shown here:

\begin{center}
\begin{tabular}{ccccccc}\hline
X             &0       &1      &2 \\ \hline
$p(x)$        &0.3    &0.4     &0.3 \\ \hline
\end{tabular}
\end{center}

\begin{center}
\begin{tabular}{ccccccc}\hline
X             &0       &1      &2 \\ \hline
$p(x)$        &0.05    &0.9     &0.05 \\ \hline
\end{tabular}
\end{center}

\begin{enumerate}
\item Calculate $\mu$ and $\sigma$ for each distribution.
\item Which distribution appears to be more variable?
\end{enumerate}
\fi
\end{enumerate}



\iffalse
a) $3x-5y=4$\\
b) $4x-7y=9$\\
c) $2x_1-3x_2+5x_3=8$

\item Determine if\\ a)$x_1=0,x_2=1,x_3=-1$\\  b)$x_1=-1,x_2=-1,x_3=6$\\ is a solution to the following
systems\\
(i)
$$\begin{array}{rrr}
x_1+3x_2+2x_3&=&1\\
x_1+x_2+3x_3&=&1\\
2x_1+4x_2+3x_3&=&-1
\end{array}$$

(ii)
$$\begin{array}{rrr}
x_1+3x_2+x_3&=&2\\
3x_1+2x_2+x_3&=&1
\end{array}$$

(iii)
$$\begin{array}{rrr}
4x_1+-3x_2+x_3&=&5\\
-5x_1+9x_2+2x_3&=&8\\
2x_1-7x_2-x_3&=&-1
\end{array}$$

\item Write the systems in question 2) as augmented matrices.

\item Put the following augmented matrices  in\\
i) row echelon form\\
ii) reduced row echelon form\\
a)$\left [\begin {array}{rrrr} 2&5&\vline&3\\1&3&\vline&2\end {array}
\right ]$
b)$\left [\begin {array}{rrrrrr} 1&-1&-2&1&\vline&7\\-2&1&6&-1
&\vline&-4\\2&0&-8&1&\vline&9\end {array}\right ]$

\item Solve
\begin{eqnarray*} x_1+3x_2-x_3&=&0\\
2x_1-5x_2+x_3&=&0\\ 3x_1+7x_2-2x_3&=&0 \end{eqnarray*}

\item Solve
$$\begin{array}{rrr} x_1-2x_2+2x_3+x_4&=&-5\\
x_2-2x_3-x_4&=&1\\ x_1-x_2+x_3&=&-5\\ 2x_1-2x_2+2x_3+x_4&=&-7
\end{array}$$

\item Solve
$$\begin{array}{rrr} x_1+2x_2+2x_3+7x_4&=&1\\
2x_1+4x_2+2x_3+9x_4&=&1\\ x_1+2x_2-x_3-x_4&=&-2\\ 3x_1+6x_2+5x_3+9x_4&=&7
\end{array}$$

\item Solve
$$\begin{array}{rrr} x_1+x_2+2x_3+3x_4+3x_5&=&2\\
x_2+2x_3+x_4&=&3\\ x_1+2x_2+4x_3+5x_4+3x_5&=&7\\ x_1-x_2-2x_3-x_4+4x_5&=&-8
\end{array}$$

\item Solve
$$\begin{array}{rrr} x_1+2x_2+x_3+x_5-x_6&=&1\\
3x_1+6x_2+6x_4+6x_6&=&12\\ x_1+2x_2+x_3-2x_5+x_6&=&2\\ 2x_1+4x_2+2x_3+2x_5-3x_6&=&-1
\end{array}$$
\end{enumerate}
\fi
\iffalse
\documentclass[12pt]{article}
\usepackage{amsmath}
\usepackage{latexsym}
\usepackage{maple2e}
\addtolength{\textwidth}{1in} \addtolength{\oddsidemargin}{-0.5in}
\addtolength{\textheight}{1.6in} \addtolength{\topmargin}{-0.8in}

\newfont{\tebbb}{msbm10 scaled\magstep1}

\newtheorem{theorem}{Theorem}[section]
\newtheorem{proposition}[theorem]{Proposition}
\newtheorem{lemma}[theorem]{Lemma}
\newtheorem{corollary}[theorem]{Corollary}
\newtheorem{remark}[theorem]{Remark}
\newtheorem{example}[theorem]{Example}
\newcommand{\beq}{\begin{equation}}
\newcommand{\eeq}{\end{equation}}
\newtheorem{definition}[theorem]{Definition}


\newcommand{\cross}[2]{{{\bf{#1}} \times {\bf{#2}}}}
\newcommand{\dotprod}[2]{{{\bf{#1}} \cdot {\bf{#2}}}}
\newcommand{\real}[1]{{\mbox{\tebbb R}}^{#1}}
\newcommand{\norm}[1]{\|{\bf{#1}}\|}
\renewcommand{\theequation}{\thesection.\arabic{equation}}

\baselineskip = 20pt plus 3pt minus 3pt

%\begin{document}
\fi
\iffalse
\section{Suggested Exercises}
\label{ssec.sugexercises}
\markright{\ref{ssec.sugexercises}
\titleref{ssec.sugexercises}}
\begin{enumerate}
\item  Find the general solution to:\\
a) $3x-5y=4$\\
b) $4x-7y=9$\\
c) $2x_1-3x_2+5x_3=8$

\item Determine if\\ a)$x_1=0,x_2=1,x_3=-1$\\  b)$x_1=-1,x_2=-1,x_3=6$\\ is a solution to the following
systems\\
(i)
$$\begin{array}{rrr}
x_1+3x_2+2x_3&=&1\\
x_1+x_2+3x_3&=&1\\
2x_1+4x_2+3x_3&=&-1
\end{array}$$

(ii)
$$\begin{array}{rrr}
x_1+3x_2+x_3&=&2\\
3x_1+2x_2+x_3&=&1
\end{array}$$

(iii)
$$\begin{array}{rrr}
4x_1+-3x_2+x_3&=&5\\
-5x_1+9x_2+2x_3&=&8\\
2x_1-7x_2-x_3&=&-1
\end{array}$$

\item Write the systems in question 2) as augmented matrices.

\item Put the following augmented matrices  in\\
i) row echelon form\\
ii) reduced row echelon form\\
a)$\left [\begin {array}{rrrr} 2&5&\vline&3\\1&3&\vline&2\end {array}
\right ]$
b)$\left [\begin {array}{rrrrrr} 1&-1&-2&1&\vline&7\\-2&1&6&-1
&\vline&-4\\2&0&-8&1&\vline&9\end {array}\right ]$

\item Solve
\begin{eqnarray*} x_1+3x_2-x_3&=&0\\
2x_1-5x_2+x_3&=&0\\ 3x_1+7x_2-2x_3&=&0 \end{eqnarray*}

\item Solve
$$\begin{array}{rrr} x_1-2x_2+2x_3+x_4&=&-5\\
x_2-2x_3-x_4&=&1\\ x_1-x_2+x_3&=&-5\\ 2x_1-2x_2+2x_3+x_4&=&-7
\end{array}$$

\item Solve
$$\begin{array}{rrr} x_1+2x_2+2x_3+7x_4&=&1\\
2x_1+4x_2+2x_3+9x_4&=&1\\ x_1+2x_2-x_3-x_4&=&-2\\ 3x_1+6x_2+5x_3+9x_4&=&7
\end{array}$$

\item Solve
$$\begin{array}{rrr} x_1+x_2+2x_3+3x_4+3x_5&=&2\\
x_2+2x_3+x_4&=&3\\ x_1+2x_2+4x_3+5x_4+3x_5&=&7\\ x_1-x_2-2x_3-x_4+4x_5&=&-8
\end{array}$$

\item Solve
$$\begin{array}{rrr} x_1+2x_2+x_3+x_5-x_6&=&1\\
3x_1+6x_2+6x_4+6x_6&=&12\\ x_1+2x_2+x_3-2x_5+x_6&=&2\\ 2x_1+4x_2+2x_3+2x_5-3x_6&=&-1
\end{array}$$
\end{enumerate}



\section{Answers to activity questions and suggested
exercises}\label{ssec.answers1}\markright{\ref{ssec.answers1}
\titleref{ssec.answers1}}

{\bf Activity Questions}

\bigskip

\noindent {\bf \ref{ssec.lineq}:}
\begin{enumerate} \item \begin{enumerate}
\item[(i)] linear
\item[(ii)] non-linear
\item[(iii)] non-linear
\end{enumerate}
\item \begin{enumerate}
\item[(i)] $\{(s, \ t, \ 2-4s+t): \ s,t  \ \in \mbox{\tebbb R}\}$,
There are an infinite number of particular solutions, an example
is $x=1,\ y=0,\ z=3$, which is found by letting $t=1$ and $s=0$.
\item[(ii)] $\{(s, \ t, \ u, \ 2-s+t-6u): \ s,t,u  \ \in \mbox{\tebbb
R}\}$. An example of a particular solution is
$x_1=1, \ x_2=1, \ x_3=0,\ x_4=2$ which is found by letting
$s=1,\ t=1,\ u=0$.
\end{enumerate}
\end{enumerate}

\noindent {\bf \ref{ssec.mlinsys}}
\begin{enumerate}
\item $\left [ \begin{array}{rrrrcr}
                    1&-2&1&2&\vline&5 \\
                    -1&0&-1&0&\vline&-5 \\
                    0&1&-1&-2&\vline&-1 \end{array} \right ]$

\item $\left[ \begin{array}{rrrr}
1&-2&1&2 \\ -1&0&-1&0 \\ 0&1&-1&-2 \end{array} \right]
\left[\begin{array}{r} x_1 \\ x_2 \\ x_3 \\ x_4
\end{array} \right]=\left[ \begin{array}{r} 5 \\ -5 \\ -1
\end{array} \right]$

\end{enumerate}
$\left[ \begin{array}{rrrr} 1&-2&1&2 \\ -1&0&-1&0 \\ 0&1&-1&-2
\end{array} \right]$ is the coefficient matrix.

\bigskip

\noindent {\bf \ref{ssec.rreduction}}
\begin{enumerate}
\item divide row 2 by 2
$$\left \{ \begin{array}{rrrrrrr}
x&+&y&+&3z&=&-2\\
&&2y&-&z&=&4
\end{array} \right . \quad \leadsto \quad \left \{
\begin{array}{rrrrrrr}
x&+&y&+&3z&=&-2\\ &&y&-&\frac{1}{2}z&=&2
\end{array} \right .$$

\item interchange row 2 and 3
$$\left \{ \begin{array}{rrrrrrr}
x&+&0.1y&+&3z&=&-2\\
2x&&&-&z&=&4\\
6x&-&y&-&2z&=&1 \end{array} \right .  \leadsto
\left \{ \begin{array}{rrrrrrr}
x&+&0.1y&+&3z&=&-2\\
6x&-&y&-&2z&=&1 \\
2x&&&-&z&=&4
\end{array} \right .$$

\item row 2 minus 2 times row 1
$$\left \{ \begin{array}{rrrrr}
x&-&y&=&2\\
2x&+&y&=&1 \end{array} \right . \quad \leadsto \quad
\left \{ \begin{array}{rrrrr}
x&-&y&=&2\\
&&3y&=&-3 \end{array} \right .$$
\end{enumerate}

\bigskip

\noindent {\bf \ref{ssec.gausse}:}

\noindent {\bf (a)} \begin{enumerate}
\item $ \left [ \begin{array}{rrrcr}
                    1&-1&0&\vline&\frac{3}{2}\\
                    0&1&0&\vline&\frac{1}{2}\\
                    0&0&1&\vline&-1 \end{array} \right ]$
\item $\left [ \begin{array}{rrrrcr} 1 &-1 &-2&3&\vline&-1\\
0&0&1&
-1&\vline&2
\end{array} \right ].$
\end{enumerate}
{\bf (b)} \begin{enumerate}
\item $ \left [ \begin{array}{rrrcr}
                    1&0&0&\vline&2\\
                    0&1&0&\vline&\frac{1}{2}\\
                    0&0&1&\vline&-1 \end{array} \right ].$
\item $\left [ \begin{array}{rrrrcr} 1&-1&0&1&\vline&3\\0&0&1&-1&\vline&2
\end{array} \right ]$.
\end{enumerate}
{\bf (c)} \begin{enumerate}
\item $x_1=2,\ x_2=\frac{1}{2}, \ x_3=-1$.
\item $x_1=3+s-t, \ x_2=s, \ x_3=2+t, \ x_4=t$.
\end{enumerate}

\bigskip

\noindent {\bf Suggested Exercises}

\begin{enumerate}
\item
a) $\{(\frac{4}{3}+\frac{5}{3}t,t): t \in {\mbox{\tebbb R}}\}$\\
b) $\{(t,-\frac{9}{7}+\frac{4}{7}t): t \in {\mbox{\tebbb R}}\}$\\
c)  $\{(s, t,\frac{8}{5}-\frac{2}{5}s+\frac{3}{5}t): s,t \in {\mbox{\tebbb R}}\}$
\item
a) (i)no, (ii) yes, (iii) no,\\
b) (i)no, (ii) yes, (iii) yes,\\

\item a) $\left [\begin {array}{rrrrr} 1&3&2&\vline&1 \\ 1&1&3&\vline&1\\ 2&4&3&
\vline&-1  \end {array}\right ]$, b)$\left [\begin {array}{rrrrr} 1&3&1&\vline&2 \\3&2&1&
\vline&1\end {array}\right ]$, c)$\left [\begin {array}{rrrrr} 4&-3&1&\vline&5 \\-5&9&2&
\vline&8 \\ 2&-7&-1&\vline&-1\end {array}\right ]$
\item a) (i)$\left [\begin {array}{rrrr} 1&\frac{5}{2}&\vline&\frac{3}{2}\\0&1&\vline&1
\end {array}\right ]$, \quad (ii) $\left [\begin {array}{rrrr} 1&0&\vline&-1\\0&1&\vline&1\end {array}
\right ]$\\

b) (i)$\left [\begin {array}{rrrrrr} 1&-1&-2&1&\vline&7\\0&1&-2&-1&
\vline&-10\\0&0&0&1&\vline&15\end {array}\right ]$, \quad
(ii)$\left [\begin {array}{rrrrrr} 1&0&-4&0&\vline&-3\\0&1&-2&0&\vline&5
\\0&0&0&1&\vline&15\end {array}\right ]$

\item $x_1=0, x_2=0, x_3=0$
\item $x_1=-2,x_2=2,x_3=-1,x_4=3$
\item  no solution
\item $x_1=-5,x_2=1-2t,x_3=t,x_4=2,x_5=0$
\item
$x_1=-2-2s-2t,x_2=s,x_3=\frac{13}{3}+2t,x_4=t,x_5=\frac{5}{3},
x_6=3$

\end{enumerate}
\fi

%\end{document}
` 