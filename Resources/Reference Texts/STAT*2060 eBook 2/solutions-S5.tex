\begin{enumerate}
\item For the following question, let
$$ {\bf u}=(1,0,-2,0) \quad {\bf v}=(2,3,4,-1)  \quad {\bf
w}=(4,2,0,-1) \quad {\bf x}=(2,-3,-1,1). $$ Find
\begin{enumerate}
\item (i) ${\bf u}+{\bf v}$ \quad (ii) $2{\bf u}-3{\bf x}+{\bf v}$
\quad (iii) $c$ such that ${\bf x}+{\bf v}+c{\bf w}=(0,-2,3,1)$
\item (i) ${\bf v}\cdot {\bf w}$ \quad (ii) ${\bf x}\cdot {\bf v}$
\quad (iii) ${\bf v}\cdot({\bf x}+{\bf w}) $
\item (i) ${\bf x} \cdot (2{\bf v}-{\bf w})$ \quad (ii)
$({\bf w}\cdot {\bf x}){\bf u}-({\bf u}\cdot {\bf x}){\bf v}$
\item (i) $\| {\bf u} \|$ \quad (ii)$ \|{\bf v} \|$ \quad (iii)$\| 7{\bf x}-6{\bf w}\| $
\item The length of {\bf x}.
\item (i) $\frac{1}{\|{\bf w}\|}{\bf w}$ \quad (ii) $ \| \frac{1}{\|{\bf w}\|}{\bf w}\|$
\item The distance between the endpoints of {\bf v} and {\bf w}
\item The cosine of the angle between {\bf x} and {\bf w}. What
kind of angle is it?
\item The cosine of the angle between {\bf u} and {\bf v}. What
kind of angle is it?
\end{enumerate}
\begin{description}\item  (a) (i)
\begin{eqnarray*}
{\bf u}+{\bf v}&=&(1,0,-2,0)+(2,3,4,-1)
\\&=&(1+2,0+3,-2+4,0+(-1))
\\&=&(3,3,2,-1)
\end{eqnarray*}
\item\quad\, (ii)
\begin{eqnarray*}
2{\bf u}-3{\bf x}+{\bf v}&=&2(1,0,-2,0)-3(2,-3,-1,1)+(2,3,4,-1)
\\&=&(2,0,-4,0)+(-6,9,3,-3)+(2,3,4,-1)
\\&=&(-2,12,3,-4)
\end{eqnarray*}
\item \quad\, (iii)
\begin{eqnarray*}
&&{\bf x}+{\bf v}+c{\bf w}=(0,-2,3,1)
\\&&(2,-3,-1,1)+(2,3,4,-1)+c(4,2,0,-1)=(0,-2,3,1)
\\&&(4+4c,2c,3,-c)=(0,-2,3,1)
\\&&c=-1 satisfies this condition.
\end{eqnarray*}
\item (b) (i)
\begin{eqnarray*}
{\bf v}\cdot{\bf w}&=&(2,3,4,-1) \cdot (4,2,0,-1)
\\&=&(2)(4)+(3)(2)+(4)(0)+(-1)(-1)
\\&=&15
\end{eqnarray*}
\item\quad\, (ii)
\begin{eqnarray*}
{\bf x} \cdot {\bf v}&=&(2,-3,-1,1) \cdot (2,3,4,-1)
\\&=&4-9-4-1
\\&=&-10
\end{eqnarray*}
\item\quad\, (iii)
\begin{eqnarray*}
{\bf v} \cdot ({\bf x} + {\bf w})&=&(2,3,4,-1) \cdot \left[(2,-3,-1,1)+(4,2,0,-1)\right]
\\&=&(2,3,4,-1) \cdot (6,-1,-1,0)
\\&=&12-3-4+0
\\&=&5
\end{eqnarray*}
\item (c) (i)
\begin{eqnarray*}
{\bf x}\cdot(2{\bf v}-{\bf w})
&=&(2,-3,-1,1) \cdot \left[2(2,3,4,-1)-(4,2,0,-1) \right]
\\&=&(2,-3,-1,1) \cdot (0,4,8,-1)
\\&=&0-12-8-1
\\&=&-21
\end{eqnarray*}
\item\quad\, (ii)
\begin{eqnarray*}
({\bf w} \cdot {\bf x}){\bf u}-({\bf u} \cdot {\bf x}){\bf v}
&=&\left[(4,2,0,-1) \cdot (2,-3,-1,1)\right](1,0,-2,0)
\\&&-\left[(1,0,-2,0) \cdot (2,-3,-1,1)
\right](2,3,4,-1)
\\
&=&(1)(1,0,-2,0)-(4)(2,3,4,-1)
\\
&=&(1,0,-2,0)+(-8,-12,-16,4)
\\
&=&(-7,-12,-18,4)
\end{eqnarray*}
\item (d) (i)
\begin{eqnarray*}
\| {\bf u} \|&=&\sqrt{(1)^{2}+(0)^{2}+(-2)^{2}+(0)^{2}}
\\&=&\sqrt{1+0+4+0}
\\&=&\sqrt{5}
\end{eqnarray*}
\item\quad\, (ii)
\begin{eqnarray*}
\| {\bf v} \|&=&\sqrt{(2)^{2}+(3)^{2}+(4)^{2}+(-1)^{2}}
\\&=&\sqrt{4+9+16+1}
\\&=&\sqrt{30}
\end{eqnarray*}
\item\quad\, (iii)
\begin{eqnarray*}
\| 7{\bf x}-6{\bf w} \|&=&\| 7(2,-3,-1,1)-6(4,2,0,-1) \|
\\&=&\|(-10,-33,-7,13)\|
\\&=&\sqrt{(-10)^{2}+(-33)^{2}+(-7)^{2}+(13)^{2}}
\\&=&\sqrt{1407}
\end{eqnarray*}
\item (e) The length of ${\bf x}=\|{\bf x}\|=\sqrt{15}$\\
\item (f) (i) First, $\|{\bf w}\|=\sqrt{21}$. So:
$\frac{1}{\|{\bf w}\|}{\bf
w}=(\frac{4}{\sqrt{21}},\frac{2}{\sqrt{21}},0,-\frac{1}{\sqrt{21}})$
\item\quad\, (ii)
\begin{eqnarray*}
\|\frac{1}{\|{\bf w}\|}{\bf w}\|&=&\sqrt{(\frac{4}{\sqrt{21}})^2+(\frac{2}{\sqrt{21}})^2,0,(-\frac{1}{\sqrt{21}})^2}
\\&=&\sqrt{\frac{21}{21}}
\\&=&1
\end{eqnarray*}
\item (g)
The distance between the endpoints of {\bf v} and {\bf
w} is d$({\bf v},{\bf w})$.
\begin{eqnarray*} d({\bf v},{\bf w})
&=&\|{\bf v}-{\bf w}\|
\\&=&\sqrt{(2-4)^2+(3-2)^2+(4-0)^2+(-1+1)^2}
\\&=&\sqrt{21}
\end{eqnarray*}
\item (h)
\begin{eqnarray*}
&&\cos\theta=\frac{ {\bf x} \cdot {\bf w} }{ \|{\bf x}\|  \|{\bf
w}\|}
\\&&\cos\theta=\frac{1}{(\sqrt{15})(\sqrt{21})}
\\&&\cos\theta=\frac{1}{\sqrt{315}}
\end{eqnarray*}
The angle is acute ($\theta \doteq 86.8$ degrees).
\item (i) \begin{eqnarray*}&&\cos\theta=\frac{ {\bf u} \cdot {\bf v} }{ \|{\bf u}\|  \|{\bf
v}\| }
\\&&\cos\theta=\frac{-6}{(\sqrt{5})(\sqrt{30})}
\\&&\cos\theta=\frac{-6}{\sqrt{150}}
\end{eqnarray*}
\noindent The angle is obtuse ($\theta \doteq 119.3$ degrees).
\end{description}

\item Find a vector, $(a,b,c,d)$, that is orthogonal to
$(-1,0,0,1)$.

\noindent \textbf{Solution}

\noindent If $(a,b,c,d)$ is perpendicular to $(-1,0,0,1)$ then the
dot product of the two vectors is zero.

$(a,b,c,d) \cdot (-1,0,0,1)=0$

$\Rightarrow\ -a+d=0$

\noindent There are many solutions to $-a+d=0$. An example of one
particular vector orthogonal to $(-1,0,0,1)$ is $(1,0,1,1)$.

\item Determine which of the following operations are possible. If
they are not possible, explain why.

(a) $({\bf u} \cdot {\bf x}) \| {\bf w} \| \quad \quad$ (b) $\|
({\bf u} , {\bf v}) \|^2 \quad \quad$ (c) $({\bf u} \cdot {\bf
v},{\bf v} \cdot {\bf w})$

\noindent \textbf{Solution} \begin{description} \item (a)
The operation is possible (scalar times a scalar).
\item (b)
The operation is not possible.  The norm is not defined
for a scalar; it is used to find vector lengths only.
\item (c)
The operation is possible (vectors have scalar components).
\end{description}
\item
\begin{enumerate}
\item If ${\bf v}$ is any 3-tuple, show that $\frac{1}{\bf \|v\|}{\bf
v}$ is a unit vector. What is the direction of this unit vector?
Show that this is true if {\bf v} is any $n$-tuple.
\item For each vector {\bf u} given, find a vector of unit length in the
same direction as {\bf u}.

(i) ${\bf u}=(-1,4)$ \quad \quad \quad\quad\,\,\,(ii) ${\bf u}=(-6,1,-5,-3)$ \quad
\quad

(iii) ${\bf u}=(1,1,1,-1)$ \quad \quad (iv) ${\bf
u}=(1,5,0,-1,-7)$
\end{enumerate}

\noindent \textbf{Solution} \begin{description} \item (a)
Let ${\bf v}$ be any 3-tuple such that:

${\bf v}=(a,b,c)$ \quad \quad $a,b,c\in$ {\tebbb R}

\noindent $\|{\bf v}\|=\sqrt{a^{2}+b^{2}+c^{2}}$
\begin{eqnarray*}
\frac{1}{\|{\bf v}\|}{\bf v}=
(\frac{a}{\sqrt{a^{2}+b^{2}+c^{2}}},\frac{b}{\sqrt{a^{2}+b^{2}+c^{2}}},
\frac{c}{\sqrt{a^{2}+b^{2}+c^{2}}})
\end{eqnarray*}
\noindent We know that $\frac{1}{\|{\bf v}\|}{\bf v}$ is a unit
vector because the norm of the vector is $1$. Examine:
\begin{eqnarray*} \|\frac{1}{\|{\bf v}\|}{\bf v}\|
&=&\sqrt{  (\frac{a}{\sqrt{a^{2}+b^{2}+c^{2}}})^{2} +
(\frac{b}{\sqrt{a^{2}+b^{2}+c^{2}}})^{2} +
(\frac{c}{\sqrt{a^{2}+b^{2}+c^{2}}})^{2} }
\\&=&\sqrt{ \frac{a^{2}}{\sqrt{a^{2}+b^{2}+c^{2}}}  +
\frac{b^{2}}{\sqrt{a^{2}+b^{2}+c^{2}}} +
\frac{c^{2}}{\sqrt{a^{2}+b^{2}+c^{2}}} }
\\&=&\sqrt{\frac{a^{2}+b^{2}+c^{2}}{a^{2}+b^{2}+c^{2}} }
\\&=&\sqrt{1}=1
\end{eqnarray*}

\noindent The vector $\frac{1}{\|{\bf v}\|}{\bf v}$ points in the
same direction as the original ${\bf b}$ vector.

\noindent This is true if ${\bf v}$ is any n-tuple vector.
Examine:

${\bf v}=(x_{1},x_{2},....,x_{n})$ where
$x_{1},x_{2},...,x_{n}\in$ {\tebbb R}
\begin{eqnarray*}
\|{\bf v} \|= \sqrt{x_{1}^{2}+x_{2}^{2}+.....+x_{n}^{2}}
\end{eqnarray*}
$\frac{1}{\|{\bf v} \|}{\bf
v}=(\frac{x_{1}}{\sqrt{x_{1}^{2}+x_{2}^{2}+.....+x_{n}^{2}}},\frac{x_{2}}{\sqrt{x_{1}^{2}
+x_{2}^{2}+.....+x_{n}^{2}}},...,\frac{x_{n}}{\sqrt{x_{1}^{2}+x_{2}^{2}+.....+x_{n}^{2}}})$

\begin{enumerate} \item[$\|\frac{1}{\|{\bf v} \|}{\bf v} \|$]

$=\sqrt{(\frac{x_{1}}{
\sqrt{x_{1}^{2}+x_{2}^{2}+.....+x_{n}^{2}}})^{2}+(\frac{x_{2}}{
\sqrt{x_{1}^{2}+x_{2}^{2}+.....+x_{n}^{2}}})^{2}+...+(\frac{x_{n}}{
\sqrt{x_{1}^{2}+x_{2}^{2}+.....+x_{n}^{2}}})^{2}}$

$=\sqrt{\frac{x_{1}^{2}}{x_{1}^{2}+x_{2}^{2}+.....+x_{n}^{2}}
+\frac{x_{2}^{2}}{x_{1}^{2}+x_{2}^{2}+.....+x_{n}^{2}}+...+\frac{x_{n}^{2}}{x_{1}^{2}+x_{2}^{2}
+.....+x_{n}^{2}}}$

$=\sqrt{\frac{ x_{1}^{2}+x_{2}^{2}+...+x_{n}^{2} }
{x_{1}^{2}+x_{2}^{2}+...+x_{n}^{2} }  }$

$=\sqrt{1}=1$

\end{enumerate}

\item (b) (i)
${\bf u}=(-1,4)$

$\|{\bf u}\|=\sqrt{17}$

$\frac{1}{\|{\bf u}\|}{\bf
u}=(-\frac{1}{\sqrt{17}},\frac{4}{\sqrt{17}})$ is a vector of unit
length in the same direction as ${\bf u}$.
\item\quad\, (ii)
${\bf u}=(-6,1,-5,-3)$

$\|{\bf u}\|=\sqrt{71}$

$\frac{1}{\|{\bf u}\|}{\bf
u}=(-\frac{6}{\sqrt{71}},\frac{1}{\sqrt{71}},-\frac{5}{\sqrt{71}},-\frac{3}{\sqrt{71}}
)$ is a vector of unit length in the same direction as ${\bf u}$.
\item\quad\, (iii)
${\bf u}=(1,1,1,-1)$
$\|{\bf u}\|=\sqrt{4}$
$\frac{1}{\|{\bf u}\|}{\bf
u}=(\frac{1}{\sqrt{4}},\frac{1}{\sqrt{4}},\frac{1}{\sqrt{4}},-\frac{1}{\sqrt{4}}
)$ is a vector of unit length in the same direction as ${\bf u}$.
\item \quad\, (iv)
${\bf u}=(1,5,0,-1,-7)$
$\|{\bf u}\|=\sqrt{76}$
$\frac{1}{\|{\bf u}\|}{\bf
u}=(\frac{1}{\sqrt{76}},\frac{5}{\sqrt{76}},0,-\frac{1}{\sqrt{76}},-\frac{7}{\sqrt{76}})$
is a vector of unit length in the same direction as ${\bf u}$.
\end{description}

\item Determine $t$ so that the following vectors are of unit
length in $\mbox{\tebbb R}^4$, if possible.

(a) $(t,-\frac{1}{4},\sqrt{2},0) \quad \quad$ (b)
$(\frac{t}{2},0,-\frac{t}{6}, -t)$

\noindent \textbf{Solution} \begin{description} \item (a)
If $(t,-\frac{1}{4},\sqrt{2},0) \cdot
(t,-\frac{1}{4},\sqrt{2},0)=1$ then $(t,-\frac{1}{4},\sqrt{2},0)$
is of unit length. Thus

$t^2+(-\frac{1}{4})^2+(\sqrt{2})^2+0=1\ \Rightarrow\
t^2=-\frac{{17}}{16}$

\noindent There is no real root that will satisfy this condition;
thus, $(t,-\frac{1}{4},\sqrt{2},0)$ cannot be of unit length.
\item (b)
If $(\frac{t}{2},0,-\frac{t}{6}, -t)\cdot
(\frac{t}{2},0,-\frac{t}{6}, -t)=1$ then
$(\frac{t}{2},0,-\frac{t}{6}, -t)$ is of unit length. Thus:
$( \frac{t}{2}  )^{2}+( -\frac{t}{6}  )^{2}+( -t )^{2}=1\
\Rightarrow\ t=\pm\frac{6}{\sqrt{46}}$
\noindent Thus the vector is of unit length when
$t=\pm\frac{6}{\sqrt{46}}$.
\end{description}

\item \begin{enumerate}
\item For ${\bf u}=(4,1,0,-a)$ and ${\bf v}=(a,2,-1,-5a)$,
find $a$ so that ${\bf u} \cdot {\bf v}=0$.
\item For ${\bf u}=(a,3,-2,5)$, find $a$ so that $\|{\bf u}\|=9$.
\end{enumerate}

\noindent \textbf{Solution} \begin{description} \item (a)
${\bf u} \cdot {\bf v}=0$

$(4,1,0,-a) \cdot (a,2,-1,-5a)=0$

$4a+2+0+5a^2=0$

So $a=\frac{-2\pm \sqrt{6}i}{5}$ when ${\bf u} \cdot {\bf v} =0$,
where $i^2 = -1$ .
\item (b)
$\|{\bf u}\|=9$

$\Rightarrow \sqrt{(a)^{2}+(3)^{2}+(-2)^{2}+(5)^{2}}=9$

$(a^{2}+38)^{\frac{1}{2}}=9$

$a^{2}+38=81$

$a=\pm\sqrt{43}$

\noindent Then $a=\pm\sqrt{43}$ when $\|{\bf u}\|=9$.
\end{description}

\item Let ${\bf{u}} = (1, 2, 3, 4)$ and ${\bf{v}} = (-1, 0, 1, 1)$. Find
\begin{enumerate}
\item the vector projection of ${\bf{u}}$ on ${\bf{v}}$
\item the vector projection of ${\bf{v}}$ on ${\bf{u}}$
\item use your answer to (a) to find a vector orthogonal to ${\bf{v}}$
\end{enumerate}

\noindent \textbf{Solution} \begin{description} \item (a) Projection of ${\bf{u}}$ on ${\bf{v}}$
\begin{eqnarray*}
&=&\frac{{\bf{u}} \cdot {\bf{v}}}{{\bf{v}}\cdot{\bf{v}}}\,{\bf{v}} \\&=& \frac{-1 + 3 + 4}{-1^2 + 1^2 + 1^2}\,(-1,0,1,1)\\
&=&2(-1, 0, 1, 1) \\&=& (-2, 0, 2, 2)
\end{eqnarray*}
\item (b) Projection of ${\bf{v}}$ on ${\bf{u}}$
\begin{eqnarray*}
&=&\frac{{\bf{v}} \cdot {\bf{u}}}{{\bf{u}}\cdot{\bf{u}}}\,{\bf{u}} \\&=& \frac{6}{1^2 + 2^2 + 3^2 + 4^2}\,(1,2,3,4)\\
&=&\tfrac{1}{5}(1, 2, 3, 4)
\\&=& (\tfrac{1}{5}, \tfrac{2}{5}, \tfrac{3}{5}, \tfrac{4}{5})
\end{eqnarray*}
\item (c) ${\bf{u}}$ - (projection of ${\bf{u}}$ on ${\bf{v}}$) is orthogonal to ${\bf{v}}$
\begin{equation*}
(1,2,3,4) - (-2,0,2,2) = (3,2,1,2)
\end{equation*}
\end{description}

\item Find the equation of the plane through $(-1,2,-3)$ with normal vector $(8,1,1)$.\\

\noindent \textbf{Solution}

Plane through $(x_0, y_0, z_0)$ with normal $(a, b, c)$ has equation
$$
ax + by + cz = ax_0 + by_0 + cz_0.
$$
Required equation is
$$
8x + y + z = -8 + 2 -3,\ \mathrm{i.e.}\ 8x+y+z = -9
$$

\item
\begin{enumerate}
\item For what values of $r$ do the points $(1,2,r),\ (-2,r,3),\ (2,0,-1),\ (0,1,1)$ lie on the same plane?
\item What is the equation of the plane in this case?
\end{enumerate}

\noindent \textbf{Solution} 

\begin{description} \item (a) The equation of the plane through $(1,2,r),\ (2,0,-1),\ (0,1,1)$ is
\begin{align*}
\left| \begin{array}{rrrr} x & y & z & -1 \\ 1 & 2 & r & -1 \\ 2 & 0 & -1 & -1 \\ 0 & 1 & 1 & -1 \end{array} \right|&
= \left| \begin{array}{cccc} x & y-1 & z-1 & 0 \\ 1 & 1 & r-1 & 0 \\ 2 & -1 & -2 & 0 \\ 0 & 1 & 1 & -1 \end{array} \right| \\
&= - \left| \begin{array}{ccc} x & y-1 & z-1 \\ 1 & 1 & r-1 \\ 2 & -1 & -2 \end{array} \right| \\
&= -\bigl\{x(-2+r-1) - (y-1)(-2-2r+2)\\&\quad \quad +(z-1)(-1-2)\bigr\} \\
&= -\bigl\{ (r-3)x + 2ry - 3z - 2r+3 \bigr\} = 0, \\ & \mathrm{i.e.}\ (r-3)x + 2ry-3z = 2r-3.
\end{align*}
$(-2, r, 3)$ lies on this plane if
\begin{align*}
(r-3)(-2) + 2r\cdot r -3 \cdot 3 &= 2r-3 \\
2r^2 - 4r &= 0, \quad r= 0,2.
\end{align*}
\item (b) When $r=0$, the equation is
\begin{align*}
-3x -3z = -3, \quad \mathrm{or}\ x+z = 1.
\end{align*}
When $r=2$, the equation is
\begin{align*}
-x + 4y -3z = 1, \quad \mathrm{or}\ x-4y+3z=-1.
\end{align*}
\end{description}
\item
\begin{enumerate}
\item For what values of $r$ do the points $(1,1),\,(1,0),\,(-2,1),\,(r,2)$ lie on the same circle?
\item Find the radius of the circle and the coordinates of the center.
\end{enumerate}
\smallskip

\noindent \textbf{Solution} \begin{description} \item (a) The equation of the circle through $(1,1),\,(1,0),\,(-2,1)$ is
\begin{align*}
\left| \begin{array}{cccc} x^2 + y^2 & x & y & 1 \\ 2 & 1 & 1 & 1 \\ 5 & -2 & 1 & 1 \\ 1 & 1 & 0 & 1 \end{array} \right|
&= \left| \begin{array}{cccc} x^2 + y^2-1 & x-1 & y & 0 \\ 1 & 0 & 1 & 0 \\ 4 & -3 & 1 & 0 \\ 1 & 1 & 0 & 1 \end{array} \right| \\
&=  \left| \begin{array}{ccc} x^2 + y^2-1 & x-1 & y \\ 1 & 0 & 1 \\ 4 & -3 & 1 \end{array} \right|\\
&= (x^2 + y^2-1)\cdot 3 - (x-1)(-3) + y(-3)\\
&= 3(x^2 + y^2) + 3x - 3y -6 = 0, \\& \mathrm{i.e.}\ x^2 + y^2 + x-y = 2
\end{align*}
$(r,2)$ lies on this circle if
$$
r^2+4+r-2=2,\quad \mathrm{i.e.}\ r=0,-1.
$$
\item (b) Complete the squares:
$$
x^2 + x + \tfrac{1}{4} + y^2 - y + \tfrac{1}{4} = 2 + \tfrac{1}{4} \cdot \tfrac{1}{4} \quad \mathrm{i.e.}\ (x+\tfrac{1}{2})^2 + (y-\tfrac{1}{2})^2 = \tfrac{5}{2}.
$$
The coordinates of the center of the circle are $(-\tfrac{1}{2},\tfrac{1}{2})$ and the radius is $\sqrt{\tfrac{5}{2}}$.
\end{description}
\item If $(-2,3,r)$ lies on the plane through $(-3, 4, 2)$ with normal $(1, r, -1)$, find the value of $r$ and the equation of the plane.

\noindent \textbf{Solution}

The plane through $(-3, 4, 2)$ with normal $(1, r, -1)$ has equation
$$
x + ry - z = -3 + 4r -2 = 4r -5.
$$
$(-2,3,r)$ lies on the plane if
$$
-2 + r \cdot 3 - r = 4r -5, \quad \mathrm{i.e.}\ 3 = 2r,\ r= \tfrac{3}{2}
$$
The equation of the plane is
$$
x + \tfrac{3}{2}y -z = 1, \quad \mathrm{i.e.}\ 2x + 3y - 2z = 2.
$$
\item A parabola with certain orientation has equation
$ay^2+bx+cy+d=0$.  Find the equation of the parabola through the
points $(5,3),(0,-1),(2,1)$.

\noindent \textbf{Solution}

The system of equations
$$ay^2+bx+cy+d=0$$
$$9a+5b+3c+d=0$$
$$a+0b-c+d=0$$
$$a+2b+c+d=0$$ has a non-zero solution.  Therefore
$$\left| \begin{array}{cccc} y^2&x&y&1\\
9&5&3&1\\
1&0&-1&1\\
1&2&1&1\end{array} \right|=0.$$ Evaluating this determinant as in
section \ref{ssec.Danal}, we obtain
$$y^2+8y-8x+7=0.$$
\end{enumerate}

