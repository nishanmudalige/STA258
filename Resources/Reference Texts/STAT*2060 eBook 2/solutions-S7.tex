\begin{enumerate}
\item Find the characteristic polynomial for the matrices.

a) $\left [ \begin{array}{rrr}
                -1&-1&0\\4&2&3\\0&6&4\end{array} \right ]$
\quad b) $\left [ \begin{array}{rrr}
                0&4&2\\1&0&-3\\-2&0&2\end{array} \right ]$\\
                
\noindent \textbf{Solution} 

\begin{description}\item (a)
\begin{eqnarray*}
\left| \begin{array}{rrr} \lambda+1&1~~&0~~ \\
-4~~&\lambda-2&-3~~ \\ 0~~&-6~~&\lambda-4 \end{array}
\right|&=&(\lambda+1)(\lambda^2-6\lambda+8-18)-\left(
-4(\lambda-4) \right) \\ &=&\lambda^3-5\lambda^2-12 \lambda-26.
\end{eqnarray*}
\item (b)
\begin{eqnarray*}
\left| \begin{array}{rrr} \lambda&-4&-2 \\ -1&\lambda&3 \\
2&0&\lambda-2 \end{array} \right|&=&2(-12+2\lambda)+
(\lambda-2)(\lambda^2-4) \\ &=& \lambda^3-2\lambda^2-16.
\end{eqnarray*}
\end{description}
\item For each of the following matrices $A$, find the eigenvalues
and the corresponding eigenvectors. When possible, find an
invertible matrix $P$ and a diagonal matrix $D$ such that
$P^{-1}AP=D$.

a) $\left[ \begin{array}{rr} -7&10\\ -5&8\end{array} \right]$
\quad b) $\left[ \begin{array}{rr} 6&-5\\ 10&-9\end{array}
\right]$ \quad c) $\left[ \begin{array}{rr} -9&-4\\20&9\end{array}
\right]$

d) $\left[ \begin{array}{rr} 0&-1\\ 1&0\end{array} \right]$ \quad
e) $\left[ \begin{array}{rr} 2&1\\ -2&0\end{array} \right]$ \quad
f) $\left[ \begin{array}{rr} 1&1 \\ 0&1 \end{array} \right]$\\

\noindent \textbf{Solution}

\begin{description}\item (a)
$$\left| \begin{array}{rr} \lambda+7&-10~~ \\ 5~~&\lambda-8 \end{array}
\right|=\lambda^2-\lambda-56+50=(\lambda-3)(\lambda+2).$$ The
eigenvalues are $\lambda=3,-2$.

\noindent For $\lambda=3$, the eigenspace is the nullspace of
$\left [\begin {array}{rr} 10&-10 \\ 5&-5 \end {array}\right]
\leadsto \left [\begin {array}{rr} 1&-1 \\ 0&0 \end
{array}\right]$. Hence an eigenvector is $\left [\begin {array}{r}
1 \\ 1 \end {array}\right]$.

\noindent For $\lambda=-2$, the eigenspace is the nullspace of
$\left[ \begin {array}{rr} 5&-10 \\ 5&-10 \end {array}\right]
\leadsto \left [\begin {array}{rr} 1&-2 \\ 0&0 \end
{array}\right]$. Hence an eigenvector is $\left [\begin {array}{r}
2 \\ 1 \end {array}\right]$.

\noindent Since $\{(1,1),(2,1)\}$ is linearly independent,
$P=\left [\begin {array}{rr} 1&2 \\1&1 \end {array}\right]$
is invertible and

$\left [\begin {array}{rr} 1&2\\1&1 \end{array} \right]^{-1} \left
[\begin {array}{rr} -7&10 \\ -5&8 \end {array}\right] \left
[\begin {array}{rr} 1&2 \\ 1&1 \end {array}\right]=\left [\begin
{array}{rr} 3&0 \\ 0&-2 \end {array}\right]$. Thus $P=\left
[\begin {array}{rr} 1&2 \\ 1&1 \end {array}\right]$, $D=\left
[\begin {array}{rr} 3&0 \\ 0&-2 \end {array}\right]$.

Note that since any non-zero multiple of an eigenvector is an
eigenvector, we could have (for example) $P=\left [\begin
{array}{rr} 2&-2 \\ 2&-1 \end {array}\right]$. Likewise we could
switch the column of $P$ and of $D$, so $P=\left [\begin
{array}{rr} -2&2 \\ -1&2 \end {array}\right]$, $D=\left [\begin
{array}{rr} -2&0 \\ 0&3 \end {array}\right]$ would also be
correct. These remarks apply throughout questions $2$ and $3$.

\noindent Parts (b), (c) are similar with

\item (b) $\lambda=1$, ${\bf v}=(1,1)$; $\lambda=-4$, ${\bf v}=(1,2)$;
$P=\left [\begin {array}{rr} 1&1 \\ 1&2 \end {array}\right]$,
$D=\left [\begin {array}{rr} 1&0 \\ 0&-4 \end {array}\right]$.
\item (c) $\lambda=1$, ${\bf v}=(2,-5)$; $\lambda=-1$, ${\bf v}=(1,-2)$;
$P=\left [\begin {array}{rr} 2&1 \\ -5&-2 \end {array}\right]$,
$D=\left [\begin {array}{rr} 1&0 \\ 0&-1 \end {array}\right]$.
\item (d)
$\left| \begin {array}{rr} \lambda&1 \\ -1&\lambda \end
{array}\right]=\lambda^2+1=(\lambda+i)(\lambda-i)$, where
$i^2=-1$.

When $\lambda=i$, the eigenspace is the nullspace of $\left[
\begin {array}{rr} i&1 \\ -1&i \end {array}\right] \leadsto
\left[ \begin {array}{rr} 1&-i \\ 0&0 \end {array}\right]$. Hence
an eigenvector is $\left[ \begin {array}{r} i\\ 1 \end
{array}\right]$.

When $\lambda=-i$, the eigenspace is the nullspace of $\left[
\begin {array}{rr} -i&1 \\ -1&-i \end {array}\right] \leadsto
\left[ \begin {array}{rr} 1&i \\ 0&0 \end {array}\right]$. Hence
an eigenvector is $\left[ \begin {array}{r} -i\\ 1 \end
{array}\right]$.

The vectors $(i,1)$, $(-i,1)$ are linearly independent, because
$\left| \begin {array}{rr} i&-i\\1&1 \end {array}\right|=i+i=2i
\neq 0$.

Thus $P=\left[ \begin {array}{rr} i&-i\\1&1 \end {array}\right]$
is invertible and $D=\left[ \begin {array}{rr} i&0\\0&-i \end
{array}\right]$.
\item (e)
$\left| \begin {array}{rr} \lambda-2&-1 \\ 2~~&\lambda
\end {array} \right|= \lambda^2-2\lambda+2 \Rightarrow
\lambda = \frac{2\pm \sqrt{4-8}}{2}=1\pm i$, where $i$ is the imaginary
number satisfying $i^2=-1$.

For $\lambda=1+i$, the eigenspace is the nullspace of $\left[
\begin {array}{rr} -1+i&-1~ \\ 2~~&1+i \end {array}\right]$.  This
reduces to $\left[ \begin {array}{rr} 1&\frac{1+i}{2} \\ 0&0 \end
{array}\right]$, so an eigenvector is $\left[ \begin {array}{r} 1+i \\ -2 \end
{array}\right]$.

Similarly for $\lambda=1-i$, we require the nullspace of $\left[
\begin {array}{rr} -1-i&-1~ \\ 2~~&1-i \end {array}\right]$, which reduces to
$\left[ \begin {array}{rr} 1&\frac{1-i}{2} \\ 0&0 \end
{array}\right]$.  An eigenvector is $\left[ \begin {array}{r} 1-i \\ -2 \end
{array}\right]$.

The vectors $(1+i,-2)$, $(1-i,-2)$ are linearly independent
because $$\left| \begin {array}{rr} 1+i&1-i \\ -2~&-2~ \end
{array}\right|=-2(1+i)+2(1-i)=-4i.$$

Thus $P=\left[ \begin {array}{rr} 1+i&1-i \\ -2~&-2~ \end
{array}\right]$ is invertible and $D=\left[
\begin {array}{rr} 1+i&0~~ \\ 0~~&1-i \end {array}\right]$.
\item (f)
$\left| \begin {array}{rr} \lambda-1&-1~ \\ 0~~&\lambda-1
\end {array} \right|=(\lambda-1)^2$.

For $\lambda=1$, the eigenspace is the nullspace of $\left[
\begin {array}{rr} 0&-1 \\ 0&0 \end {array}\right] \leadsto
\left[ \begin {array}{rr} 0&1 \\ 0&0 \end {array}\right]$.

This matrix has rank 1 and nullity 1. Thus the dimension of the
eigenspace for $\lambda=1$ is 1, and so there is only eigenvector
in the basis for the eigenspace. Such a vector is $(1,0)$. Since
$\lambda=1$ is the only eigenvalue it is impossible to find two independent
eigenvectors to write as the columns of $P$. Hence
$\exists$ no invertible matrix $P$.
\end{description}
\item For each of the following matrices $A$, find the eigenvalues
and corresponding eigenvectors. When possible find an invertible
matrix $P$ and a diagonal matrix $D$ such that $P^{-1}AP=D$.

(a) $\left[ \begin {array}{rrr} -5&-12&6\\-1&-2&2
\\-5&-12&8\end {array} \right]$, (b) $\left[ \begin {array}{rrr}
-9&4&4\\-8&3&4 \\-16&8&7\end {array} \right]$\\ \\ \\(c) $\left[ \begin
{array}{rrr} 1&0&-2\\0&1&-2 \\0&1&-1\end {array} \right]$, (d)
$\left[ \begin {array}{rrr} 3&6&-18 \\-4&-11&36 \\-1&-3&10 \end
{array} \right]$.\\

\noindent \textbf{Solution} \begin{description}\item (a)
\begin{eqnarray*}
\left| \begin {array}{rrr} \lambda+5&12~~&-6~ \\1~~&\lambda+2&-2~
\\5~~&12~~&\lambda-8\end {array} \right|&=&\left| \begin {array}{rrr}
\lambda+5&0~~&-6~ \\ 1~~&\lambda-2&-2~ \\5~~&2\lambda-4&\lambda-8
\end {array} \right| \quad C_2=C_2+2C_3 \\
&=&(\lambda-2)\left| \begin{array}{rrr} \lambda+5&0&-6~\\1~~&1&-2~
\\5~~&2&\lambda-8\end {array} \right| \\ &=&(\lambda-2)\left[
(\lambda+5)(\lambda-8+4)-6(2-5)
\right] \\
&=&(\lambda-2)[\lambda^2+\lambda-2]=(\lambda-2)(\lambda+2)(\lambda-1)
\end{eqnarray*}

The eigenvalues are $\lambda=1,2,-2$.

For $\lambda=1$ the eigenspace is the nullspace of \\$ \left[
\begin {array}{rrr} 6&12&-6 \\ 1&3&-2 \\5&12&-7 \end {array}\right] \leadsto
\left[ \begin {array}{rrr} 1&0&1 \\ 0&1&-1 \\ 0&0&0 \end
{array}\right]$, and an eigenvector is $\left[
\begin {array}{r} -1 \\ 1 \\1 \end {array}\right]$.\\

For $\lambda=2$ the eigenspace is the nullspace of \\$\left[
\begin {array}{rrr} 7&12&-6 \\ 1&4&-2 \\5&12&-6 \end {array}\right] \leadsto
\left[ \begin {array}{rrr} 1&0&0 \\ 0&1&-\frac{1}{2} \\ 0&0&0 \end
{array}\right]$, and an eigenvector is $\left[
\begin {array}{r} 0 \\ 1 \\2 \end {array}\right]$.\\

For $\lambda=-2$ the eigenspace is the nullspace of \\$\left[
\begin {array}{rrr} 3&12&-6 \\ 1&0&-2 \\5&12&-10 \end {array}\right] \leadsto
\left[ \begin {array}{rrr} 1&0&-2 \\ 0&1&0 \\ 0&0&0 \end
{array}\right]$, and an eigenvector is $\left[
\begin {array}{r} 2 \\ 0 \\1 \end {array}\right]$.\\

The eigenvectors $\{(-1,1,1), (0,1,2), (2,0,1)\}$ are linearly independent, so $$P=\left[ \begin {array}{rrr} -1&0&2 \\
1&1&0 \\ 1&2&1 \end {array}\right] \  {\rm is \ invertible}, \quad D=\left[ \begin
{array}{rrr} 1&0&0 \\ 0&2&0 \\ 0&0&-2 \end {array}\right].$$
\item (b)

We saw in example $7.5$ that the eigenvalues are
$\lambda=-1,-1,3$.

For $\lambda=-1$, the eigenspace  is the nullspace of $\left[
\begin {array}{rrr} 8&-4&-4 \\ 8&-4&-4 \\16&-8&-8 \end {array}\right] \leadsto
\left[ \begin {array}{rrr} 1&-\frac{1}{2}&-\frac{1}{2} \\ 0&0&0
\\ 0&0&0 \end {array}\right]$. This has rank 1 and nullity 2, so
the dimension of the eigenspace is 2. A basis for the eigenspace
is $\{(1,0,2),(1,2,0)\}$.

For $\lambda=3$, the eigenspace  is the nullspace of $\left[
\begin {array}{rrr} 12&-4&-4 \\ 8&0&-4 \\16&-8&-4 \end {array}\right] \leadsto \\
\left[ \begin {array}{rrr} 1&0&-\frac{1}{2} \\ 0&1&-\frac{1}{2}
\\ 0&0&0 \end {array}\right]$, and an eigenvector is $\left[ \begin {array}{r}
1 \\ 1 \\ 2 \end {array}\right]$.

The vectors $\{(1,0,2),(1,2,0),(1,1,2)\}$ are linearly independent. Therefore $$P=\left[
\begin {array}{rrr} 1&1&1 \\ 0&2&1 \\ 2&0&2 \end {array}\right] \ {\rm is \  invertible},
\quad D=\left[ \begin {array}{rrr} -1&0&0 \\ 0&-1&0 \\ 0&0&3 \end
{array}\right].$$
\item (c)
\begin{eqnarray*}
\left| \begin {array}{rrr} \lambda-1&0~~&2~~ \\ 0~~&\lambda-1&2~~ \\
0~~&-1~~&\lambda+1 \end {array}\right|&=&(\lambda-1)\left| \begin
{array}{rrr} 1&0~~&2~~ \\ 0& \lambda-1&2~~\\ 0&-1~~&\lambda+1\end
{array}\right|\\&=&(\lambda-1)[(\lambda-1)(\lambda+1)+2] \\
&=&(\lambda-1)(\lambda^2+1)=(\lambda-1)(\lambda+i)(\lambda-i).
\end{eqnarray*}
The eigenvalues are $\lambda=1,i,-i$.

For $\lambda=1$, the eigenspace  is the nullspace of $\left[
\begin {array}{rrr} 0&0&2 \\ 0&0&2 \\0&-1&2 \end {array}\right] \leadsto
\left[ \begin {array}{rrr} 0&1&0 \\ 0&0&1 \\ 0&0&0 \end {array}
\right]$. Hence an eigenvector is $\left[ \begin {array}{r} 1 \\ 0
\\0 \end {array}\right]$.

For $\lambda=i$, the eigenspace  is the nullspace of $\left[
\begin {array}{rrr} -1+i&0~~&2~ \\ 0~~&-1+i&2~~ \\0~~&-1~&1+i \end
{array}\right] \\ \leadsto \left[ \begin {array}{rrr} 1&0&-1-i \\
0&1&-1-i \\ 0&0&0~~ \end {array} \right]$. Hence an eigenvector is
$\left[ \begin {array}{r} 1+i \\ 1+i \\1~~ \end {array}\right]$.

For $\lambda=-i$, the eigenspace  is the nullspace of $\left[
\begin {array}{rrr} -1-i&0~~&2~ \\ 0~~&-1-i&2~~ \\0~~&-1~&1-i \end
{array}\right] \\ \leadsto \left[ \begin {array}{rrr} 1&0&-1+i \\
0&1&-1+i \\ 0&0&0~~ \end {array} \right]$. Hence an eigenvector is
$\left[ \begin {array}{r} 1-i \\ 1-i \\1~~ \end {array}\right]$.\\

The vectors $\{(1,0,0),(1+i,1+i,1),(1-i,1-1,0)\}$ are independent,
so $P=\left[ \begin {array}{rrr} 1&1+i&1-i \\ 0&1+i&1-i \\
0&1~~&1~~\end {array}\right]$ is invertible and $D= \left[
\begin{array} {rrr} 1&0&0 \\ 0&i&0 \\ 0&0&i \end {array}\right]$.\\
\item (d)
\begin{eqnarray*}
\left| \begin {array}{rrr} \lambda-3&-6~~&18~~ \\ 4~~&\lambda+11&-36~~ \\
1~~&3~~&\lambda-10 \end{array}\right|
&=&\left| \begin{array}{rrr} \lambda-3&-6~~&18~~~ \\ 2\lambda-2& \lambda-1&0~~~ \\
1~~&3~~&\lambda-10 \end{array} \right| \,R_2\\
&=& R_2+2R_1
\\&=&(\lambda-1)\left| \begin {array}{rrr} \lambda-3&-6&18~~ \\
2~~&1&0~~ \\ 1~~&3&\lambda-10 \end {array}\right|\\
&=&(\lambda-1)\left| \begin {array}{rrr} \lambda-1&0&2\lambda-2 \\
2~~&1&0~~~ \\ 1~~&3&\lambda-10
\end{array}\right|\, R_1\\&=& R_1+2R_3 \\ &=& (\lambda-1)^2
\left| \begin {array}{rrr} 1&0&2~~~ \\ 2&1&0~~~ \\ 1&3&\lambda-10
\end {array}\right| \\ &=&(\lambda-1)^2[(\lambda-10)+2(6-1)]=\lambda(\lambda-1)^2.
\end{eqnarray*}
The eigenvalues are $\lambda=0,1,1$.

For $\lambda=0$, the eigenspace  is the nullspace of $\left[
\begin {array}{rrr} -3&-6&18 \\ 4&11&-36 \\1&3&-10 \end {array}\right] \leadsto
\left[ \begin {array}{rrr} 1&0&2 \\ 0&1&-4 \\ 0&0&0 \end {array}
\right]$. Hence an eigenvector is $\left[ \begin {array}{r} -2 \\
4 \\1 \end {array}\right]$.

For $\lambda=1$, the eigenspace  is the nullspace of $\left[
\begin {array}{rrr} -2&-6&18 \\ 4&12&-36 \\1&3&-9 \end {array}\right] \leadsto
\left[ \begin {array}{rrr} 1&3&-9 \\ 0&0&0 \\ 0&0&0 \end {array}
\right]$. This has rank 1, and nullity 2. Hence we can find
two independent eigenvectors, for example $\left[ \begin {array}{r} 9 \\
0 \\1 \end {array}\right]$, $\left[ \begin {array}{r} -3 \\
1 \\0 \end {array}\right]$.\\
The vectors $\{(-2,4,1),(9,0,1),(-3,1,0)\}$ are linearly independent, hence $P=\left[ \begin {array}{rrr} -2&9&-3 \\
4&0&1 \\ 1&1&0 \end {array}\right]$ is invertible and $D= \left[
\begin {array}{rrr} 0&0&0 \\ 0&1&0 \\ 0&0&1 \end
{array}\right]$.\\

\end{description}
\item Find the limiting vector of state probabilities for the
Markov chain with transition matrix $\left[ \begin {array}{rrr} .1&.2&.3 \\
.2&.4&.6 \\ .7&.4&.1 \end {array}\right]$\\

\noindent \textbf{Solution}

We seek an eigenvector for the eigenvalue $\lambda=1$. The
eigenspace is the nullspace of $\left[
\begin {array}{rrr} .9&-.2&-.3 \\ -.2&.6&-.6 \\-.7&-.4&.9 \end {array}\right]
\leadsto \left[ \begin {array}{rrr} 1&0&-\frac{3}{5} \\ 0&1&-\frac{6}{5} \\
0&0&0 \end {array} \right]$.\\
An eigenvector is of the form $c\left( \frac{3}{5}, \frac{6}{5},1
\right)$, $c~\epsilon~ \mbox {\tebbb R}$. We must choose $c$ to
obtain a vector of probabilities, i.e. the coordinates must sum to
1. Hence $\frac{3c}{5}+\frac{6c}{5}+c=1$, or $c=\frac{5}{14}$.
Therefore the vector we require is $\frac{5}{14} \left(
\frac{3}{5}, \frac{6}{5},1 \right)=\left( \frac{3}{14},
\frac{6}{14}, \frac{5}{14} \right)$.
\end{enumerate}
\newpage
\markboth{}{}
