\pagenumbering{arabic}\setcounter{page}{1}
\chapter{Overview}
\index{Overview}
%start relabeling as 2.1 etc
\pagestyle{myheadings}  

\setcounter{equation}{0}

\index{Statistics!Introduction}


Uncertainty is an inherent part of everyday life. 
We all face questions regarding uncertainty such as whether 
classes will go ahead as planned on any given day;
will a flight leave on time; 
will a student pass a certain course?
Uncertainties might also change depending on other factors, 
such as whether classes will still go ahead as planned when there is a snow warning in effect;
if a flight is delayed can a person still manage to make their connection;
will a student pass their course considering that the instructor is known to be a tough grader?\\

The ability to quantify uncertainty using rigorous mathematics is a powerful and useful tool.
Calculating uncertainty on an intuitive level is something that is hard-wired in our DNA, 
such as the decision to fight or flight depending on a given set of circumstances.
However we cannot always make such intuitive decisions
based purely on hunches and gut feelings.
Fortunes have been lost based on someone having a good feeling about something.
If we have information available, we should make the best prediction possible using this information.
For instance if we wanted to invest a lot of money in a company, we should use all available data such as
past sales, market and industry trends, leadership ability of the CEO, forward looking statements etc. 
and with all this information we can then predict whether our investment will be profitable.\\

In order for companies to survive and remain competitive in todays 
environment it is essential to monitor industry trends and read markets properly.
Companies that don't adapt and stick to an outdated business model
tend to pay the price.
At the other end of the spectrum, companies that understand the needs of the consumer, 
build their product around the consumer and keep evolving their product offerings based on
consumer trends tend to perform well and remain competitive.\\

Statistics is the science of uncertainty and it is clearly a very useful subject for business.
In this book you will be given an introduction to statistics and you will learn the
framework as well as the language required at the introductory level.
The material may be daunting at times, but the more you get familiar with the
subject the more comfortable you will become with it.
As business students, doing well in a statistics course will give you a competitive
edge since the ability to interpret and perform quantitative analytics are skills that are highly
desired by many employers.


