\chapter*{Preface}

This book was based on OpenInto Statistics (Second Edition). A copy of OpenInto Statistics (Second Edition) may be downloaded as a free PDF at \href{http://www.openintro.org}{\color{black}\textbf{openintro.org}}.
\vspace{3mm}

\noindent We hope readers will take away three ideas from this book in addition to forming a foundation of statistical thinking and methods.\vspace{-1mm}
\begin{enumerate}
\setlength{\itemsep}{0mm}
\item[(1)] Statistics is an applied field with a wide range of practical applications.
%\item[(2)] You don't have to be a math guru to learn from real, interesting data.
\item[(2)] You don't have to be a math whiz to learn from real, interesting data.
\item[(3)] Data is messy, and statistical tools are imperfect. But, when you understand the strengths and weaknesses of these tools, you can use them to learn about the real~world.
\end{enumerate}


%\subsection*{Textbook overview}

%{\color{red}{
%The chapters of this book are as follows:
%\begin{description}
%\setlength{\itemsep}{0mm}
%\item[1. Introduction to data.] Data structures, variables, summaries, graphics, and basic data collection techniques.
%\item[2. Probability (special topic).] The basic principles of probability. An understanding of this chapter is not required for the main content in Chapters~\ref{modeling}-\ref{multipleAndLogisticRegression}.
%\item[3. Distributions of random variables.] Introduction to the normal model and other key distributions.
%\item[4. Foundations for inference.] General ideas for statistical inference in the context of estimating the population mean.
%\item[5. Inference for numerical data.] Inference for one or two sample means using the normal model and $t$ distribution, and also comparisons of many means using ANOVA.
%\item[6. Inference for categorical data.] Inference for proportions using the normal and chi-square distributions, as well as simulation and randomization techniques.
%\item[7. Introduction to linear regression.] An introduction to regression with two variables. Most of this chapter could be covered after Chapter~\ref{introductionToData}.
%\item[8. Multiple and logistic regression.] An introduction to multiple regression and logistic regression for an accelerated course.
%\end{description}
%} }


The chapters of this book are as follows:
\begin{description}
\setlength{\itemsep}{0mm}
\item[1. Overview.] General concept of inherent certainty and the power of prediction.
\item[1. Introduction to data.] Data structures, variables, types of studies and experimental design and basic data collection techniques.
%summaries, graphics, and basic data collection techniques.
\item[2. Descriptive statistics.] Numerical measures, graphical representations of data, data summaries
\item[3. Probability.] The basic principles of probability.
%An understanding of this chapter is not required for the main content in Chapters~\ref{modeling}-\ref{multipleAndLogisticRegression}.
\item[4. Distributions of random variables.] Commonly used distributions of discrete and continuous random variables. Introduction to the normal model and other key distributions.
\item[5. Basic Foundations for inference.] General ideas for statistical inference in the context of estimating the population mean. Emphasis on sampling theory.
\item[6. Confidence intervals.] One sample confidence intervals on the mean and on proportions; two sample confidence intervals on a difference of means on on a difference of proportions;
confidence intervals on paired data.
%\item[7. Inference for categorical data.] Inference for proportions using the normal and chi-square distributions, as well as simulation and randomization techniques.
%\item[8. Introduction to linear regression.] An introduction to regression with two variables. Most of this chapter could be covered after Chapter~\ref{introductionToData}.
%\item[9. Multiple and logistic regression.] An introduction to multiple regression and logistic regression for an accelerated course.
\end{description}

\emph{OpenIntro Statistics} was written to allow flexibility in choosing and ordering course topics. The material is divided into two pieces: main text and special topics. The main text has been structured to bring statistical inference and modeling closer to the front of a course. Special topics, labeled in the table of contents and in section titles, may be added to a course as they arise naturally in the curriculum.

\subsection*{Examples, exercises, and appendices}

Examples and within-chapter exercises throughout the textbook may be identified by their distinctive bullets:

\begin{example}{Large filled bullets signal the start of an example.}
Full solutions to examples are provided and often include an accompanying table or figure.
 \end{example}

\begin{exercise}
Large empty bullets signal to readers that an exercise has been inserted into the text for additional practice and guidance. Students may find it useful to fill in the bullet after understanding or successfully completing the exercise. Solutions are provided for all within-chapter exercises in footnotes.\footnote{Full solutions are located down here in the footnote!}
\end{exercise}

%There are exercises at the end of each chapter that are useful for practice or homework assignments. Many of these questions have multiple parts, and odd-numbered questions include solutions in Appendix~\ref{eoceSolutions}.

%There are exercises at the end of each chapter that are useful for practice or homework assignments. Many of these questions have multiple parts, and odd-numbered questions include solutions in Appendix~\ref{eoceSolutions}.

%There are exercises at the end of each chapter that are useful for practice or homework assignments. Many of these questions have multiple parts, and odd-numbered questions include solutions in Appendix~\ref{eoceSolutions}.


Probability tables for the normal, $t$, and chi-square distributions are in Appendix~\ref{distributionTables}, and PDF copies of these tables are also available from \href{http://www.openintro.org}{\color{black}\textbf{openintro.org}} for anyone to download, print, share, or modify.

\subsection*{OpenIntro, online resources, and getting involved}

OpenIntro is an organization focused on developing free and affordable education materials. \emph{OpenIntro Statistics}, our first project, is intended for introductory statistics courses at the high school through university levels.

We encourage anyone learning or teaching statistics to visit \href{http://www.openintro.org}{\color{black}\textbf{openintro.org}} and get involved. We also provide many free online resources, including free course software. Data sets for this textbook are available on the website and through a companion R package.\footnote{Diez DM, Barr CD, \c{C}etinkaya-Rundel M. 2012. \texttt{openintro}: OpenIntro data sets and supplement functions. \urlwofont{http://cran.r-project.org/web/packages/openintro}.} All of these resources are free, and we want to be clear that anyone is welcome to use these online tools and resources with or without this textbook as a companion.

We value your feedback. If there is a particular component of the project you especially like or think needs improvement, we want to hear from you. You may find our contact information on the title page of this book or on the \href{http://www.openintro.org/about.php}{About} section of \href{http://www.openintro.org}{\color{black}\textbf{openintro.org}}.

\subsection*{Acknowledgements}

This project would not be possible without the dedication and volunteer hours of all those involved. No one has received any monetary compensation from this project, and we hope you will join us in extending a \emph{thank you} to all those volunteers below.

The authors would like to thank Andrew Bray, Meenal Patel, Yongtao Guan, Filipp Brunshteyn, Rob Gould, and Chris Pope for their involvement and contributions. We are also very grateful to Dalene Stangl, Dave Harrington, Jan de Leeuw, Kevin Rader, and Philippe Rigollet for providing us with valuable feedback.


