\iffalse
\documentclass[12pt]{article}
\usepackage{amsmath}
\usepackage{latexsym}

\addtolength{\textwidth}{1in} \addtolength{\oddsidemargin}{-0.5in}
\addtolength{\textheight}{1.6in} \addtolength{\topmargin}{-0.8in}

\newfont{\tebbb}{msbm10 scaled\magstep1}

\newtheorem{theorem}{Theorem}[section]
\newtheorem{proposition}[theorem]{Proposition}
\newtheorem{lemma}[theorem]{Lemma}
\newtheorem{corollary}[theorem]{Corollary}
\newtheorem{remark}[theorem]{Remark}
\newtheorem{example}[theorem]{Example}
\newcommand{\beq}{\begin{equation}}
\newcommand{\eeq}{\end{equation}}
\newtheorem{definition}[theorem]{Definition}


\newcommand{\cross}[2]{{{\bf{#1}} \times {\bf{#2}}}}
\newcommand{\dotprod}[2]{{{\bf{#1}} \cdot {\bf{#2}}}}
\newcommand{\real}[1]{{\mbox{\tebbb R}}^{#1}}
\newcommand{\norm}[1]{\|{\bf{#1}}\|}
\renewcommand{\theequation}{\thesection.\arabic{equation}}

\baselineskip = 20pt plus 3pt minus 3pt

\begin{document}
\fi
\section{Suggested Exercises}\label{ssec.se3}\markright{\ref{ssec.se3} \titleref{ssec.se3}}
\begin{enumerate}
\item Which hypothesis,the null or the alternative,is the statusquo hypothesis?Which is the research hypothesis?
\item What is a test statistics?
\item Define $\alpha$ and  $\beta$. How do they relate to Type I and Type II errors?
\item If you test a hypothesis and reject the null hypothesis in favor of the alternative hypothesis,does your test prove that the alternative hypothesis is correct?Explain.
\item List all possible results of the combinations of decisions and true states of nature ofr a test of hypothesis.
\end{enumerate}

\iffalse
\begin{enumerate}
\item  Let $A=\left[
\begin{array}{rr}
2 & 3 \\ 3 & 5 \end{array} \right ]$ and calculate:
\begin{enumerate}
\item $A^{-1}$
\item (i) $4A$ \quad (ii) $(4A)^{-1}$
\item (i) $A^t$ \quad (ii) $(A^t)^{-1}$
\item (i) $A^3$ \quad (ii) $A^{-3}.$
\end{enumerate}
\item Let $B= \left[
{\begin{array}{rr}
2 & 5 \\
1 & 3
\end{array}}
 \right]$ and $A$ as above. Find
\begin{enumerate}
\item $B^{-1}$
\item (i) $AB$ \quad (ii) $(AB)^{-1}$ \quad (iii) $B^{-1}A^{-1}$
\end{enumerate}
\item For each of the following calculate the inverse.
\begin{enumerate}
\item $\left[ {\begin{array}{rrr} 1 & 0 & 2 \\ 2 & -1 & 3 \\ 4 & 1 &
8
\end{array}}
 \right]$
\item $\left[ {\begin{array}{rrr} 1 & 2 & 1 \\ 1 & 1 & 1 \\ 3 & -1 &
1
\end{array}}
 \right]$
\item $ \left[ {\begin{array}{rrr} 1 & 3 & -4 \\ 1 & 5 & -1 \\ 3 &
13 & -6
\end{array}}
 \right]
$
\item $\left[ {\begin{array}{rrr} 0 & 1 & 1 \\ 5 & 1 & -1 \\ 2 & -3
& -3
\end{array}}
 \right]$
\item $\left [\begin {array}{rrrr} 1&0&0&0\\2&-2&0&0
\\3&1&-2&0\\1&-1&3&0\end {array}
\right ]$
\item $ \left[ {\begin{array}{rrrr} 1 & 2 & 1 & 0 \\ 0 &
1 & -1 & 0 \\ 1 & 3 & 1 & -2 \\ 1 & 4 & -2 & 4
\end{array}}
\right]$
\end{enumerate}
\item Solve the following system using matrix inversion.
\begin{eqnarray*} x_1+2x_2+x_3&=&1\\
x_2-x_3&=&-1\\ x_1+3x_2+x_3-2x_4&=&-2\\ x_1+4x_2-2x_3+4x_4&=&4
\end{eqnarray*}
\item Let $A= \left[
\begin{array}{rr}
1 & 5 \\
0 & 1
\end{array}
 \right]$, $B= \left[
\begin{array}{rr}
2 & 3 \\
1 & -2
\end{array}
 \right]$ $C= \left[
\begin{array}{rr}
3 & -5 \\
-1 & 2
\end{array}
 \right]$
 $D= \left[
\begin{array}{rr}
-3 & -5 \\
7 & 2
\end{array}
 \right]$  find
\begin{enumerate}
 \item (i) $\det(A)$ \quad (ii) $\det(B)$ \quad (iii) $\det(C)$ \quad (iv) $\det(D)$
 \item (i) $\det(A+B)$ \quad (ii) $\det(A)+\det(B)$
 \item (i) $\det(3A)$ \quad (ii) $\det(B^t)$
 \item (i) $AB$ \quad (ii) $\det(AB)$ \quad (iii) $\det(A)\det(B)$
 \item $\det(D^{-1})$
\end{enumerate}
\item Let $G=\left[
{\begin{array}{rrr}
1 & 0 & 1 \\
2 & 3 & 1 \\
-7 & 0 & -7
\end{array}}
 \right]$,
$H =  \left[
{\begin{array}{rrr}
1 & 2 & 3 \\
4 & 6 & 5 \\
3 & 2 & 7
\end{array}}
 \right]$,
$J =  \left[
{\begin{array}{rrr}
-2 & 4 & -6 \\
3 & -5 & 7 \\
9 & -6 & -3
\end{array}}
 \right]$ find
\begin{enumerate}
\item (i) $\det(G)$ \quad (ii) $\det(G^t)$ \quad
\item $\det(H)$
\item $\det(J)$
\item (i) $\det(HJ)$ (ii) $\det(GJ)$
\end{enumerate}
\item Find $\det(A)$ and $\det(A^{-1})$ for
\begin{align*}
\mathrm{(a)}\ \ A &= \left[ \begin{array}{rrrr} 1 & 2 & 1 & 0 \\ 0 & 1 & -1 & 0
\\ 1 & 3 & 1 & -2 \\ 1 & 4 & -2 & 4 \end{array} \right]&
\mathrm{(b)}\ \ A&= \left[ {\begin{array}{rrrr} 10 & -3 & -3 & -2 \\ 10 & -3 &
-4 & -3 \\ -6 & 2 & 2 & 1 \\ -3 & 1 & 1 & 1 \end{array}} \right]
\end{align*}

\item We have seen that systems of equations can be solved  using either
Gaussian elimination or with ${\bf x}=A^{-1}{\bf b}$. Determine
which of the two methods are appropriate for the following two
systems.
\begin{align*}
\mathrm{(a)}\qquad \quad \quad \ x_1-x_3&=-1& \mathrm{(b)}\ \ 2x_1+x_2-x_3&=6\\
11x_1+2x_2+x_3&=2& 3x_1-x_2&=1\\
3x_1+x_2+3x_3&=1& 9x_2+2x_3&=0
\end{align*}

\end{enumerate}

\section{Answers to activity questions and suggested exercises}
\label{answers3}\markright{\ref{answers3}
\titleref{answers3}}

{\bf Activity questions}

\bigskip

\noindent {\bf \ref{ssec.definv}:}
\begin{enumerate}
\item (a), but not (b).
\item $I$.
\end{enumerate}

\bigskip

\noindent {\bf \ref{ssec.findinv}:}

\begin{enumerate}
\item \begin{enumerate}
\item
\begin{eqnarray*}
& &
\left[ \begin{array}{rrrcrrr} 1 & 0 & 1 & \vline & 1 & 0 & 0\\
                             -1 & 1 & 2 & \vline & 0 & 1 & 0\\
                              2 & 2 & 9 & \vline & 0 & 0 & 1
\end{array} \right]
\\&&\hspace{-12mm}
\overset{\begin{smallmatrix}R_2+R_1\\R_3-2R_1\end{smallmatrix}}{\leadsto}
\left[ \begin{array}{rrrcrrr} 1 & 0 & 1 & \vline & 1 & 0 & 0\\
                              0 & 1 & 3 & \vline & 1 & 1 & 0\\
                              0 & 2 & 7 & \vline & -2 & 0 & 1
\end{array} \right]
\\&&\hspace{-11mm}
\overset{R_3-2R_2}{\leadsto}
\left[ \begin{array}{rrrcrrr} 1 & 0 & 1 & \vline & 1 & 0 & 0\\
                               0 & 1 & 3 & \vline & 1 & 1 & 0\\
                               0 & 0 & 1 & \vline & -4 & -2 & 1
\end{array} \right]
\\&&\hspace{-12mm}
\overset{\begin{smallmatrix}R_1-R_3\\R_2-3R_3\end{smallmatrix}}{\leadsto}
\left[\begin{array}{rrrcrrr} 1 & 0 & 0 & \vline & 5 & 2 & -1\\
                               0 & 1 & 0 & \vline & 13 & 7 & -3\\
                               0 & 0 & 1 & \vline & -4 & -2 & 1
\end{array} \right].
\end{eqnarray*}
The inverse is
$\left[ \begin{array}{rrr} 5&2&-1\\13&7&-3\\-4&-2&1 \end{array} \right]$.
\item
\begin{eqnarray*}&&
\left[ \begin{array}{rrrcrrr} 1 & 0 & 1 & \vline & 1 & 0 & 0\\
                               1 & 1 & 0 & \vline & 0 & 1 & 0\\
                               -1 & 1 & -2 & \vline & 0 & 0 & 1
\end{array} \right]\\
 \overset{\begin{smallmatrix}R_2-R_1\\R_3+R_1\end{smallmatrix}}{\leadsto}
&&\vspace{-10mm}\left[ \begin{array}{rrrcrrr} 1 & 0 & 1 & \vline & 1 & 0 & 0\\
                              0 & 1 & -1 & \vline & -1 & 1 & 0\\
                              0 & 1 & -1 & \vline & -1 & 0 & 1
\end{array} \right]\\ \\
\overset{R_3-R_2}{\leadsto}
&&\vspace{-10mm}\left[ \begin{array}{rrrcrrr} 1 & 0 & 1 & \vline & 1 & 0 & 0\\
                               0 & 1 & -1 & \vline & -1 & 1 & 0\\
                               0 & 0 & 0 & \vline & 0 & -1 & 1
\end{array} \right].
\end{eqnarray*}
The matrix does not reduce to $I$ so $\exists$ no inverse.
\item
\begin{eqnarray*}
&&
\left[ \begin{array}{rrrrcrrrr}  1 & 0 & 0 & 0 & \vline & 1 & 0 & 0 & 0\\
                                 0 & 0 & 0 & 1 & \vline & 0 & 1 & 0 & 0\\
                                 0 & 0 & 1 & 0 & \vline & 0 & 0 & 1 & 0\\
                                 0 & 1 & 0 & 0 & \vline & 0 & 0 & 0 & 1
\end{array} \right] \\ \\\overset{R_2\leftrightarrow R_4}{\leadsto}
&&\vspace{-10mm}\left[ \begin{array}{rrrrcrrrr}  1 & 0 & 0 & 0 & \vline & 1 & 0 & 0 & 0\\
                                 0 & 1 & 0 & 0 & \vline & 0 & 0 & 0 & 1\\
                                 0 & 0 & 1 & 0 & \vline & 0 & 0 & 1 & 0\\
                                 0 & 0 & 0 & 1 & \vline & 0 & 1 & 0 & 0
\end{array} \right].
\end{eqnarray*}
 Hence this matrix is it's own inverse.

\end{enumerate}
\end{enumerate}

\bigskip

\noindent {\bf \ref{ssec.propinv}:}\\
\vspace{0.1\baselineskip}

\noindent {\bf Hints: } To prove number 5, what does
$(A^{-1})(A^{-1})^{-1}$ equal?  To prove number 7, use the exact
same method shown in the notes, except that you now multiply
$(kA)(\frac{1}{k}A^{-1}$).

\bigskip

\noindent {\bf \ref{ssec.syseinv}:}
\begin{enumerate}
\item $A^{-1}=\left [ \begin{array}{rrr}\vspace{1mm}
                                1&0&1\\ \vspace{1mm}
                                1&1&\frac{5}{2}\\
                                0&0&-\frac{1}{2} \end{array} \right
                                ]$,\  $x_1=\left [ \begin{array}{r}\vspace{1mm}
                                -2\\ \vspace{1mm}-\frac{5}{2}\\ \frac{3}{2}
                                \end{array} \right ]$,\ $x_2=\left [ \begin{array}{r} \vspace{1mm}
                                -11\\ \vspace{1mm} -\frac{13}{2}\\ \frac{1}{2}
                                \end{array} \right ]$.
\item Same as 1.

\item ${\bf x}={\bf 0}$.

\item If any other {\bf b} are introduced into the problem, by
theorem \ref{eq2invert} we know that the system will be consistent
and have exactly one solution for every {\bf b}.
\end{enumerate}

\noindent {\bf \ref{sec.det}:}
\begin{enumerate}
\item \begin{itemize} \item[(i)]
\begin{itemize}
\item[(a)] $2\left| \begin{array}{rr}2&-2\\0&1\end{array} \right| +
\left| \begin{array}{rr}3&-2\\0&1\end{array} \right| +
3 \left| \begin{array}{rr}3&2\\0&0\end{array} \right| = 4+3+0 = 7$
\item[(b)] $1\left| \begin{array}{rr}3&-2\\0&1\end{array} \right| +
2\left| \begin{array}{rr}2&3\\0&1\end{array} \right| +
0\left| \begin{array}{rr}2&3\\3&-2\end{array} \right| = 3+4+0 = 7$
\end{itemize}
\item[(ii)]
\begin{itemize}
\item[(a)] $2\left| \begin{array}{rr}1&2\\8&3\end{array} \right| -
3\left| \begin{array}{rr}0&2\\0&3\end{array} \right| +
2\left| \begin{array}{rr}0&1\\0&8\end{array} \right| = -26+0+0 = -26$
% START HERE
\item[(b)] $-3\left| \begin{array}{rr}0&2\\0&3\end{array} \right| +
1\left| \begin{array}{rr}2&2\\0&3\end{array} \right| -
8\left| \begin{array}{rr}2&2\\0&2\end{array} \right| = 0+6-32 = -26$
\end{itemize}
\end{itemize}
\item Expand matrix (i) along the bottom row
$$
\left| \begin{array}{rr}2&-1\\3&2\end{array}\right|
$$
Expand matrix (ii) along the bottom row
$$
2 \left| \begin{array}{rr}1&2\\8&3\end{array} \right| = 2(3-16) = -26
$$
\end{enumerate}

\noindent {\bf \ref{ssec.propdet}:}
\begin{enumerate}
\item
\begin{enumerate}
\item
$$
\begin{vmatrix}2&2&1\\2&1&1\\1&2&1\end{vmatrix} \overset{R_1 - R_3}{\leadsto}
\begin{vmatrix}1&0&0\\2&1&1\\1&2&1\end{vmatrix} = \begin{vmatrix}1&1\\2&1\end{vmatrix} = 1-2 =-1
$$
\item
\begin{eqnarray*}
\left| \begin{array}{rrrr} 1&2&7&5\\2&1&3&10\\4&-1&-3&20\\8&3&9&39
\end{array} \right|
&& \hspace{-5mm}\overset{\begin{smallmatrix}C_4 -
5C_1\\C_3-3C_2\end{smallmatrix}}{\leadsto} \quad \left|
\begin{array}{rrrr} 1&2&1&0\\2&1&0&0\\4&-1&0&0\\8&3&0&-1
\end{array} \right|\\
&=&
- \left| \begin{array}{rrr} 1&2&1\\2&1&0\\4&-1&0\end{array} \right| \\
&=& - \left| \begin{array}{rr}2&1\\4&-1\end{array} \right| \\&=&
-(-2-4) = 6
\end{eqnarray*}
\end{enumerate}
\item
\begin{enumerate}
\item -3
\item 270
\item -60
\item Not enough information
\end{enumerate}
\end{enumerate}
\noindent {\bf \ref{ssec.adjoint}:}  $A$ is invertible, since
det($A$)=17;  det($A^{-1}$)$=\frac{1}{17}$.


\bigskip

\noindent {\bf Suggested Exercises}

\begin{enumerate}
\item
\begin{enumerate}\item$ \left[ {\begin{array}{rr}5&-3\\-3&2 \end{array}}
\right]$\\
\item(i) $\left[ {\begin{array}{rr}8&12\\12&20\end{array}}
 \right]$\quad
(ii) $\left[ {\begin{array}{rr} \frac {5}{4}  & -\frac {3}{4}\\ -\frac {3}{4}  & \frac {1}{2}
\end{array}} \right]$
\item (i) $\left[ {\begin{array}{rr} 2 & 3 \\ 3 & 5 \end{array}}
\right]$ \quad (ii) $\left[ {\begin{array}{rr} 5 & -3 \\ -3 &
2\end{array}} \right]$
\item (i) $\left[ {\begin{array}{rr} 89 &
144
\\ 144 & 233
\end{array}}
 \right]$\quad
(ii) $
 \left[
{\begin{array}{rr}
233 & -144 \\
-144 & 89
\end{array}}
 \right]$
 \end{enumerate}

\item \begin{enumerate}
\item $ \left[
\begin{array}{rr}
3 & -5 \\
-1 & 2
\end{array}
 \right]$\\
\item (i) $\left[ {\begin{array}{rr} 7 & 19 \\ 11 & 30
\end{array}}
 \right]$ \quad (ii) $ \left[
{\begin{array}{rr}
30 & -19 \\
-11 & 7
\end{array}}
 \right]$ \quad (iii) $ \left[ {\begin{array}{rr} 30 & -19 \\ -11 & 7
\end{array}}
 \right]$
 \end{enumerate}
\item \begin{enumerate}
\item $ \left[ {\begin{array}{rrr} -11 & 2 & 2 \\ -4 & 0 & 1 \\ 6 &
-1 & -1
\end{array}}
 \right]
$
\item $\left[ {\begin{array}{rrr}\vspace{1mm} 1 & -\frac {3}{2}& \frac {1}{2}\\ \vspace{1mm}
1 & -1 & 0 \\ -2 & \frac {7}{2}  & -\frac {1}{2}
\end{array}}
 \right]$
\item Not Possible Singular matrix
\item $\left[ {\begin{array}{rrr}\vspace{1mm} \frac {3}{2} & 0 & \frac
{1}{2}\\ \vspace{1mm} -\frac {13}{4}  & \frac {1}{2} & -\frac {5}{4}  \\ \frac
{17}{4} & -\frac {1}{2} & \frac {5}{4}
\end{array}}
 \right]$
\item Not Possible \item $\left[ {\begin{array}{rccc} 10 & 10 & -6 &
-3
\\ -3 & -3 & 2 & 1 \\  -3 & -4  &
2& 1 \\-1 & -\frac{3}{2} & \frac{1}{2} & \frac{1}{2}
\end{array}}
\right]$
\end{enumerate}
\item $x_1=0,\ x_2=0,\ x_3=1,\ x_4=\tfrac{3}{2}$
\item
\begin{enumerate}
\item (i) 1 (ii) -7 (iii) 1 (iv) 29
\item (i) -11 (ii) -6
\item (i) 9 (ii) -7
\item (i) $ \begin{vmatrix} 7 & -7 \\1 & -2 \end{vmatrix}$ (ii) -7 (iii) -7
\item $\tfrac{1}{29}$
\end{enumerate}
\item
\begin{enumerate}
\item (i) 0 (ii) 0
\item -24
\item 12
\end{enumerate}
\item
\begin{enumerate}
\item (i) 2 (ii) $\frac{1}{2}$
\item (i) 1 (ii) 1
\end{enumerate}
\item
\begin{enumerate}
\item The determinant of the system is $0$, therefore $A$ is not
invertible and the system must be solved using Gaussian
elimination.
\item The determinant is not 0; either method works.
\end{enumerate}
\end{enumerate}

%\end{document}
\fi
