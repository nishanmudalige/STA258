\iffalse
\documentclass[12pt]{article}
\usepackage{amsmath}
\usepackage{latexsym}

\addtolength{\textwidth}{1in} \addtolength{\oddsidemargin}{-0.5in}
\addtolength{\textheight}{1.6in} \addtolength{\topmargin}{-0.8in}

\newfont{\tebbb}{msbm10 scaled\magstep1}

\newtheorem{theorem}{Theorem}[section]
\newtheorem{proposition}[theorem]{Proposition}
\newtheorem{lemma}[theorem]{Lemma}
\newtheorem{corollary}[theorem]{Corollary}
\newtheorem{remark}[theorem]{Remark}
\newtheorem{example}[theorem]{Example}
\newcommand{\beq}{\begin{equation}}
\newcommand{\eeq}{\end{equation}}
\newtheorem{definition}[theorem]{Definition}


\newcommand{\cross}[2]{{{\bf{#1}} \times {\bf{#2}}}}
\newcommand{\dotprod}[2]{{{\bf{#1}} \cdot {\bf{#2}}}}
\newcommand{\real}[1]{{\mbox{\tebbb R}}^{#1}}
\newcommand{\norm}[1]{\|{\bf{#1}}\|}
\renewcommand{\theequation}{\thesection.\arabic{equation}}

\baselineskip = 20pt plus 3pt minus 3pt

%\begin{document}
\fi

\section{Suggested Exercises}\label{ssec.se5}
\markright{\ref{ssec.se5}
\titleref{ssec.se5}}
\begin{enumerate}

\item Is $W= \{ (x,y,z): x \geq 0 \}$ a subspace of $\real{3}$?  Why or why not?
\item Is $W= \{ (x,y,0): x,y \in {\mbox{\tebbb R}} \}$ a subspace of $\real{3}$?
Why or why not?
\item Let ${\bf u}=(1,-3,2)$ and ${\bf v}=(2,-1,1)$. Express
\begin{enumerate}
\item {\bf w}=(1,7,-4)
\item  {\bf x}=(2,-5,4)
\end{enumerate}
as a linear combination of ${\bf u}$ and ${\bf v}$.

\item Determine if the following vectors are linearly dependent
or linearly independent.
\begin{enumerate}
\item $S=\{(1,2), (2,3)\}$
\item $S=\{(-1,-2), (2,4) \}$
\item $S=\{(1,2,-7), (0,3,0),(1,1,-7) \}$
\item $S=\{(1,2,3), (4,6,5),(3,2,7) \}$
\item $S=\{(1,0,1,1), (2,1,3,4),(1,-1,1,-2),(0,0,2,-4) \}$
\item $S=\{(10,-3,-3,-2), (10,-3,-4,-3),(-6,2,2,1),(-3,1,1,1) \}$
\end{enumerate}

\item Determine if the following vectors span ${\mbox{\tebbb R}}^2$.
\begin{enumerate}
\item  (2,5) and (3,4)
\item  (3,4) and (4,3)
\end{enumerate}
\item Determine if the following vectors span ${\mbox{\tebbb R}}^3$.
\begin{enumerate}
\item  (0,1,1), (5,1,-1), and (2,-3,-3)
\item (1,1,3), (3,5,13), and (-4,-1,-6)
\end{enumerate}
\item  For the following sets of vectors:
\begin{enumerate}
\item Determine if $S=\{ {\bf v}_1,{\bf v}_2, {\bf v}_3 \}$ is a
basis for ${\mbox{\tebbb R}}^3$.
\item  Express ${\bf v}=(3,2,5)$
as a linear combination of ${\bf v}_1,{\bf v}_2,{\bf v}_3$ if
possible.
\begin{enumerate}
\item ${\bf v}_1=(1,2,3), {\bf v}_2=(3,6,4), {\bf v}_3=(0,0,4)$.
\item${\bf v}_1=(1,2,3), {\bf v}_2=(0,1,4), {\bf v}_3=(0,1,3)$.
\item ${\bf v}_1=(1,1,3), {\bf v}_2=(2,1,-1), {\bf v}_3=(1,1,1)$.
\end{enumerate}
\end{enumerate}
\item Let $S=\{ (5,2,0), (1,3,1), (-1,1,0) \}$ and
\\$S'=\{(-1,-6,0), (1,2,2), (2,-1,3) \}$
\begin{enumerate}
\item Verify that $S$ and $S'$ are both bases for $\real{3}$.
\item Find the coordinates of ${\bf v}=(-1,0,-2)$ relative to $S$.
\item Find the coordinates of ${\bf x}=(1,15,-1)$ relative to $S'$.
\item Find the coordinates of $({\bf w})_S=(-3,-2,1)$ relative to the
standard basis for $\real{3}$.
\end{enumerate}
\end{enumerate}

\section{Answers to activity questions and suggested exercises}
\label{answers5}\markright{\ref{answers5}
\titleref{answers5}}

{\bf Activity questions}

\bigskip

{\bf \ref{ssec.subspace}:}

You have to check the three axioms for a subspace given
in section \ref{ssec.subspace}.
\begin{enumerate}
\item  A line through the origin can be written as a set
$S=\{c(x_0,y_0): c \in \real{}\}$, where $(x_0,y_0)\neq(0,0)$.
\begin{enumerate}
\item  Axiom 1.  $S$ is obviously non-empty.
\item  Axiom 2.  If $c_1(x_0,y_0), c_2(x_0,y_0) \in S$ then their sum
$$c_1(x_0,y_0)+c_2(x_0,y_0)=(c_1+c_2)(x_0,y_0)$$
is in $S$ too, because it is a multiple of $(x_0,y_0)$.
\item  Axiom 3.  If $c_1(x_0,y_0) \in S$ and $k$ is a scalar then
$$k(c_1(x_0,y_0))=(kc_1)(x_0,y_0)$$
is in $S$, because it is a multiple of $(x_0,y_0)$.
\end{enumerate}
\item  Let $S$ be the set of solutions to $A{\bf x}={\bf 0}$.  Then $S$ is not empty
because ${\bf 0} \in S$.  Let ${\bf x_1},{\bf x_2} \in S$, and let $c$ be a scalar.
Then $A{\bf x_1}=A{\bf x_2}={\bf 0}$ and so
$$A({\bf x_1}+{\bf x_2})=A{\bf x_1}+A{\bf x_2}={\bf 0}+{\bf 0}={\bf 0}.$$
This means that $({\bf x_1}+{\bf x_2}) \in S$.  Similarly
$$A(c{\bf x_1})=c(A{\bf x_1})=c\cdot{\bf 0}={\bf 0}$$
and so $c{\bf x_1} \in S$.

\iffalse
HINT: visualize this question geometrically in the plane.
What is the sum of two parallel vectors. For the second part of
the question, if $k=0$, $0{\bf u}={\bf 0}$ which is the origin,
therefore the origin must be included for the line to be a
subspace.
\item HINT: Let {\bf x} and {\bf y} be solutions to A{\bf x}={\bf
0}. That is $A{\bf x}={\bf 0}$ and $A{\bf y}={\bf 0}$. Show that
$A({\bf x}+{\bf y})={\bf 0}$ by using the properties of matrix
arithmetic. Similarly, show that $A(k{\bf x})={\bf 0}$. For the
second part of the question, again let $k=0$.
\fi
\end{enumerate}

{\bf \ref{ssec.lincomb}:}
\begin{eqnarray*}
(1,0,0)&=&2(1,2,0)-(1,3,4)+(0,-1,4)\\
(0,1,0)&=&-\frac{1}{2}(1,2,0)+\frac{1}{2}(1,3,4)-\frac{1}{2}(0,-1,4)\\
(0,0,1)&=&-\frac{1}{8}(1,2,0)+\frac{1}{8}(1,3,4)+\frac{1}{8}(0,-1,4)
\end{eqnarray*}

{\bf \ref{ssec.li}:}
\begin{enumerate}
\item Linearly dependent.
\item Linearly independent.
\end{enumerate}

\bigskip

{\bf \ref{ssec.span}:}
\begin{enumerate}
\item $\left [\begin{array}{rrcr} 2&3&\vline&1\\
-3&-2&\vline&0\\
-4&-1&\vline&1\end{array}\right]
\leadsto
\left [\begin{array}{rrcr} \vspace{1mm} 1&\frac{3}{2}&\vline&\frac{1}{2}\\
\vspace{1mm}
0&1&\vline&\frac{3}{5}\\
0&0&\vline&0\end{array}\right]$

Yes.
\item  $\left [\begin{array}{rrcr} 1&-1&\vline&4\\
0&2&\vline&-6\\
1&3&\vline&-7\end{array}\right]
\leadsto
\left [\begin{array}{rrcr} 1&-1&\vline&4\\
0&1&\vline&-3\\
0&4&\vline&-11\end{array}\right]$

No.
\end{enumerate}

{\bf  \ref{ssec.basis}:}
\begin{enumerate}
\item $S'$
\item No, the set is linearly dependent so it is not a basis.
\end{enumerate}

{\bf \ref{ssec.dimension}:} It is only required that the third
vector not be a linear combination of the other two vectors. The
vector (0,0,1) works well, as does any other vector that makes the
corresponding homogeneous system have a non-zero coefficient
matrix determinant.

\bigskip

{\bf Suggested Exercises}
\begin{enumerate}

\item No, ${\bf u}=(1,2,3) \in W$ but $-2{\bf u} \notin W$
\item Yes
\item \begin{enumerate} \item ${\bf w}=-3{\bf u}+2{\bf v}$.
\item not possible.
\end{enumerate}
\item \begin{enumerate}
\item independent
\item dependent
\item dependent
\item independent
\item independent
\item independent
\end{enumerate}

\item (a) Yes \quad (b) Yes
\item (a) Yes \quad (b) No
\item \begin{enumerate}
\item (i) No \quad (ii) Yes \quad (iii) Yes
\item (i) Not possible \quad (ii) $c_1=3 , c_2=8 ,c_3= -12$ \quad
(iii) $c_1=\frac{5}{2} , c_2=1 ,c_3= -\frac{3}{2}$
\end{enumerate}
\item \begin{enumerate}
\item [(b)] ${\bf v}_S=(1,-2,4)$
\item [(c)] ${\bf x}_{S'}=(-2,1,-1)$
\item [(d)] ${\bf w}=(-18,-11,-2)$
\end{enumerate}

\end{enumerate}

%\end{document}
