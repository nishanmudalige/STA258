\setcounter{equation}{0}

\chapter{Vector Spaces}
\label{sec.vspace} \pagestyle{myheadings}
\markboth{\ref{sec.vspace}. \titleref{sec.vspace}}{}

Up to now, we have described vectors as $n-$tuples in
$\mbox{\tebbb R}^n$. They have been presented geometrically as
arrows in 2 and 3-space. In this section we view vectors in a
broader perspective. Ten axioms are presented. Any set that
satisfies these axioms will be called a vector space and the
elements in the set are called vectors. We will learn to identify
subspaces of a vector space, to write vectors as a linear
combination of one another and to construct a basis for a vector
space. Every vector in the vector space can be written as a linear
combination of the basis vectors.

\noindent {\bf Learning Objectives}
\index{linear combination}
\begin{itemize}
\item determine whether or not a set is a subspace of a vector space
\item express vectors as linear combinations of other vectors
\item determine whether a set of vectors is linearly independent
or linearly dependent
\item determine whether a set of vectors spans a vector space
\item determine whether a set of vectors is a basis for a vector
space
\item find the dimension of a vector space
\item find the coordinates of a vector with respect to a basis
\end{itemize}

\section{Definition of a Vector Space} \label{ssec.defvs}\markright{\ref{ssec.defvs} \titleref{ssec.defvs}}
\index{vector space}\index{basis}
Let $V$ be any non-empty set of objects on which two operations
are defined: ${\bf u}+{\bf v}$ and $k{\bf v}$.  If the following
axioms are met by all objects {\bf u,v} and {\bf w} in $V$, for
all scalars $k$ and $l$, then $V$ is called a {\bf vector space},
and the objects in $V$ are called {\bf vectors}.
\begin{enumerate}
\item For all ${\bf u, v}$ in $ V$, the sum ${\bf u}+{\bf v}$ is also in $V$ .
\item ${\bf u}+{\bf v}={\bf v}+{\bf u}$.
\item ${\bf u}+({\bf v}+{\bf w})=({\bf u}+{\bf v})+{\bf w}$.
\item There exists a zero element, ${\bf 0}$ in $V$, such that ${\bf 0+v}={\bf
v+0}={\bf v}$.
\item For all ${\bf u}$ in $V$, there exists a ${\bf -u}$ in $V$, such that
${\bf u +(-u)}={\bf (-u)+u}={\bf 0}$.
\item For all ${\bf u}$ in $V$, the product $k{\bf u}$ is also in $V$.
\item $k({\bf u}+{\bf v})=k{\bf u}+k{\bf v}$.
\item $(k+l){\bf u}=k{\bf u}+l{\bf u}$.
\item $k(l{\bf u})=(kl){\bf u}$.
\item $1{\bf u}={\bf u}$.
\end{enumerate}

A vector space need not be composed of the vectors introduced in
the previous sections.  A vector space is any set of elements that
satisfies the above ten axioms.  The elements in that set are called vectors.
Thus, a vector can be
a matrix or even a function. Similarly, although the operations of
a vector space are given as ``addition" and ``scalar
multiplication", they need not be defined in the usual sense of
addition and scalar multiplication. The only requirement is that
the ten vector space axioms be met.

In this course we will consider primarily the vector spaces
$\mbox{\tebbb R}^n$ $(n\geq1)$. It is not intended that you
memorize the ten axioms above. The axioms do no more than dictate
that vector addition and scalar multiplication behave in a
sensible way.

\section{Subspaces}
\label{ssec.subspace}\markright{\ref{ssec.subspace}
\titleref{ssec.subspace}}
\index{subspace}
A subset $W$ of a vector space $V$ is a {\bf subspace} of $V$ if
it is also a vector space under the addition and scalar
multiplication of $V$.

In the case of a subspace, it is not necessary to show that all
ten axioms are satisfied. If $W$ is a subset of $V$, most of
axioms will hold for $W$ because they hold for $V$. In fact, $W$
is a subspace of $V$ if and only if the following conditions are
met.

\begin{enumerate}
\item  $W$ is non-empty.
\item For all ${\bf u,v}$ in $W$,  ${\bf u}+{\bf v}$ is in $W$ (closed
under addition). \item For all $ k$ in $\mbox{\tebbb R}$ and for
all ${\bf u}$ in $W$, $k{\bf u}$ is in $W$ (closed under scalar
multiplication).
\end{enumerate}

Since every vector space has to have a zero vector, an easy way to
show that $W$ is non-empty is to show that ${\bf 0}~\in~W$.

\subsection*{Examples of Subspaces}

The following is a list of the subspaces of $\mbox{\tebbb R}^2$
and $\mbox{\tebbb R}^3$. Verification is left as an exercise.

\begin{enumerate}
\item Subspaces of $\mbox{\tebbb R}^2$:
\begin{itemize}
\item \{{\bf 0}\}=\{(0,0)\}, the set consisting of the zero vector alone.
This subspace consists of a single point: the origin.
\item Lines through the origin.  If ($x_0,y_0$) is a non-zero
vector in $\mbox{\tebbb R}^2$ then the set \{$c(x_0,y_0)$:$c$ $\in$ $\mbox{\tebbb R}$\},
the set of all multiples of ($x_0,y_0$), is the line through the origin and ($x_0,y_0$).
\item $\mbox{\tebbb R}^2$.  We can think of  $\mbox{\tebbb R}^2$ as a subset of itself,
and so viewed it is a vector subspace.  The subspaces \{{\bf 0}\} and $\mbox{\tebbb R}^2$
are called improper subspaces.\\
\end{itemize}
\item Subspaces of $\mbox{\tebbb R}^3$
\begin{itemize}
\item \{{\bf 0}\}=\{(0,0,0)\}, the set consisting of the zero vector alone.
This subspace consists of a single point: the origin.
\item Lines through the origin.  If ($x_0,y_0,z_0$) is a non-zero
vector in $\mbox{\tebbb R}^3$ then the set \{$c(x_0,y_0,z_0)$: $c$ $\in$ $\mbox{\tebbb R}$\},
the set of all multiples of ($x_0,y_0,z_0$), is the line through the origin and ($x_0,y_0,z_0$).
\item Planes through the origin.  If ($x_0,y_0,z_0$), ($x_1,y_1,z_1$) are non-zero points in
$\mbox{\tebbb R}^3$, not on the same straight line through the origin
(so neither is a multiple of the other),
then \{$c_0(x_0,y_0,z_0)+c_1(x_1,y_1,z_1)$: $c_0, c_1$ $\in$ $\mbox{\tebbb R}$\} is a plane
through the origin.  To evaluate the expression $c_0(x_0,y_0,z_0)+c_1(x_1,y_1,z_1)$, we extend
the vectors $(x_0,y_0,z_0)$, $(x_1,y_1,z_1)$ by factors $c_0$, $c_1$ respectively and then add
using the parallelogram law.  By varying $c_0$ and $c_1$ (remember $c_0$, $c_1$ can be negative
as well as positive) we obtain all the points in the plane through the origin $(x_0,y_0,z_0)$ and $(x_1,y_1,z_1)$.
\item $\mbox{\tebbb R}^3$.  We can think of $\mbox{\tebbb R}^3$ as a subset
of itself, and then it is a vector subspace of itself.  The subspaces \{{\bf 0}\} and $\mbox{\tebbb R}^3$ are called improper subspaces.
\end{itemize}
\item The solution set of the homogeneous linear system $A{\bf
x}={\bf 0}$ with $m$ equations and $n$ unknowns, is a subspace of
$\mbox{\tebbb R}^n$.  This set is called the solution space of the
system.
\end{enumerate}

\begin{example}
\label{exam5.subspaces1}

\noindent Let $W=\{(x,y,z): x,y,z \in ${\tebbb R} and $
x^2+y^2+z^2 \leq 1 \}$. Is $W$ a subspace of {\tebbb R}$^{3}$? Why
or why not?

To prove that $W$ is a subspace of {\tebbb R}$^{3}$ we must show
that it is non-empty,

closed under addition and closed under scalar multiplication.

Is $W$ closed under scalar multiplication?

Note that for example, $(0,1,0)~\in~W$ because $0^2+1^2+0^2=1$. However

$2(0,1,0)=$ $(0,2,0)$ does not belong to $W$ because $0^2+2^2+0^2=4$. Thus $W$

is not closed under scalar multiplication. $W$ is therefore not a subspace.

As a matter of fact, $W$ is not closed under addition either. Examine

two vectors in $W$, ${\bf v}=(1,0,0)\ \ and\ \ {\bf u}=(0,0,1)$. The sum of these

vectors is: $${\bf v}+ {\bf u}= (1,0,0)+(0,0,1)=(1,0,1)$$

but $(1,0,1)$ is not in $W$ because $1^2+0^2+1^2=1+0+1=2$ which is
not

less than or equal to 1.

\end{example}

\noindent {\bf Activity \ref{ssec.subspace}:}

\begin{enumerate}
\item Verify that vectors on a line through the origin make up a
subspace of $\mbox{\tebbb R}^2$. Why is it important that the line
pass through the origin?
\item The solution set of $A{\bf x}={\bf 0}$ consists of vectors
of the form ${\bf x}=\left [ \begin{array}{c} x_1\\x_2\\ \vdots\\
x_n \end{array} \right ]$.  Show that the set of
solutions is a subspace by verifying the three axioms. Why isn't
the set of solutions to $A{\bf x}={\bf b}$ a subspace when ${\bf
b}\neq {\bf 0}$?
\end{enumerate}

\noindent Hints are given in the section \ref{answers5}.

\section{Linear Combinations of Vectors}\index{linear combination of vectors}
\label{ssec.lincomb}\markright{\ref{ssec.lincomb}
\titleref{ssec.lincomb}}

In this section, we introduce the idea of writing one vector as
the sum of other vectors.

The vector {\bf w} is a {\bf linear combination} of the vectors
${\bf v}_1,{\bf v}_2, \ldots ,{\bf v}_r$ if it can be expressed as
follows: $${\bf w}=k_1{\bf v}_1+k_2{\bf v}_2+ \cdots + k_r{\bf
v}_r$$ where $k_i \in \mbox{\tebbb R}$.

\begin{example}
\label{exam5.standvec}

The standard unit vectors for $\mbox{\tebbb R}^3$, are {\bf
i}=$(1,0,0)$, {\bf j}=$(0,1,0)$ and {\bf k}=$(0,0,1)$.  All
vectors of the form ${\bf v}=(v_1,v_2,v_3)$ can be written as a
linear combination of the standard unit vectors. $${\bf v}=v_1{\bf
i}+v_2{\bf j}+v_3{\bf k}=v_1(1,0,0)+v_2(0,1,0)+v_3(0,0,1).$$
\end{example}

By expressing linear combinations, we are essentially referencing
a vector with respect to other known vectors.  The previous
example demonstrated this by using the standard unit vectors as
these references.  This is similar to the coordinate axes of the
Cartesian coordinate system.  When we express vectors in terms of
the coordinate system, we are expressing them with respect to the
mutually perpendicular $x,y$ and $z$ axes.

The vectors we use as reference points need not be perpendicular
to each other.  The following example shows how to express a
vector with respect to other vectors which are not necessarily
standard unit vectors.

\begin{example}
\label{exam5.lincomb} Let ${\bf u}=(-1,3,2)$ and ${\bf
v}=(6,-7,1)$.  Show that ${\bf w}=(-9,16,5)$ is a linear
combination of {\bf u} and {\bf v}.

We need to find $k_1$ and $k_2$ such that
\begin{eqnarray*}
{\bf w}&=&k_1{\bf u}+k_2{\bf v}\\
(-9,16,5)&=&k_1(-1,3,2)+k_2(6,-7,1)\\
\\ &\Rightarrow& \left \{ \begin{array}{l}
                            -9=-k_1+6k_2\\
                            16=3k_1-7k_2 \quad \quad .  \\
                            5=2k_1+k_2\end{array} \right .
\end{eqnarray*}
The augmented matrix of this system is $$\left[
\begin{array}{rrcr}-1&6&\vline&-9 \\ 3&-7&\vline&16 \\
2&1&\vline&5 \end{array} \right]~which~reduces~to~\left[
\begin{array}{rrcr}1&0&\vline&3 \\ 0&1&\vline&-1 \\
0&0&\vline&0 \end{array} \right].$$ This means that the system is
consistent and $k_1=3$, $k_2=-1$.

\noindent Therefore {\bf w} is a linear combination of {\bf u} and
{\bf v}: $$(-9,16,5)=3(-1,3,2)-(6,-7,1).$$
\end{example}
\begin{example}
\label{exam5.lincomb2}
Let ${\bf u}=(1,2,-1)$ and ${\bf v}=(0,1,3)$.  Write ${\bf w}=(-1,1,-2)$ as
a linear combination of {\bf u} and {\bf v}.

We must find $k_1, k_2$ such that
${\bf w}=k_1{\bf u}+k_2{\bf v}$,
$$i.e. (-1,1,-2)=k_1(1,2,-1)+k_2(0,1,3),$$
or
$$ 1k_1+0k_2=-1$$
$$2k_1+1k_2=1$$
$$-1k_1+3k_2=-2.$$
The augmented matrix of the system is
$$\left[\begin{array}{rrcr} 1&0&\vline&-1\\
2&1&\vline&1\\
-1&3&\vline&-2\end{array}\right]~which~reduces~to~\left[
\begin{array}{rrcr}1&0&\vline&-1 \\ 0&1&\vline&3 \\
0&0&\vline&-12 \end{array} \right].$$
The system is inconsistent, so {\bf w} cannot be written as a linear combination of
{\bf u} and {\bf v}.
\end{example}

\noindent {\bf Activity \ref{ssec.lincomb}:}

Express the vectors $(1,0,0)$, $(0,1,0)$ and $(0,0,1)$ as linear
combinations of the vectors $(1,2,0)$, $(1,3,4)$, and $(0,-1,4)$.
Answers can be found in section \ref{answers5}.

\section{Linear Independence}
\label{ssec.li} \markright{\ref{ssec.li} \titleref{ssec.li}} This
\index{linear independence}\index{linearly dependent}section deals with a topic which will prove to be very important
to this course. Give extra attention to the definitions and
statements given; do the activity questions prior to moving on.

Let $S=\{{\bf v}_1,{\bf v}_2, \ldots, {\bf v}_n\}$ be a non-empty
set of vectors, and consider the equation $$k_1{\bf v}_1+k_2{\bf
v}_2+\cdots+k_n{\bf v}_n={\bf 0}$$ where $k_1,k_2,\cdots,k_n$ are
scalars.

This system is always consistent since $k_1=k_2=\cdots=k_n=0$ is
always a solution. If that is the {\bf only} solution to the
equation, then we call the set of vectors $S$ a {\bf linearly
independent set}.  If there are other solutions, then $S$ is
called {\bf linearly dependent}.

\noindent We now give some examples dealing with linearly
independent and dependent sets.

\begin{example}
\label{exam5.standardRn} Let $S=\{ {\bf e}_1, {\bf e}_2, {\bf
e}_3, \ldots ,{\bf e}_n \}$ where \\ ${\bf e}_1=(1,0,\ldots, 0)$,
${\bf e}_2=(0,1,0,\ldots,0), \quad \ldots \quad, {\bf
e}_n=(0,0,\ldots,1)$.
\begin{eqnarray*}
k_1{\bf e}_1+k_2{\bf e}_2+\cdots+k_n{\bf
e}_n&=&(k_1,k_2,\ldots,k_n)={\bf 0}\\ &\Leftrightarrow&
k_1=k_2=\cdots=k_n=0. \end{eqnarray*} Therefore $S$ is linearly
independent.
\end{example}

\begin{example}
\label{exam5.ld}
 Let ${\bf v}_1=(2,1,0,1)$, ${\bf v}_2=(1,5,3,1)$
and ${\bf v}_3=(-3,3,3,-1)$ $$2{\bf v}_1-{\bf v}_2+{\bf
v}_3=(4,2,0,2)-(1,5,3,1)+(-3,3,3,-1)=(0,0,0,0).$$ Therefore
$S=\{{\bf v}_1,{\bf v}_2,{\bf v}_3\}$ is linearly dependent.
\end{example}

It has already been mentioned many times how important solving
linear systems is to the course, and that many of the concepts
introduced in previous sections would prove to be useful
throughout the course. The next example illustrates this point.

\begin{example}
\label{exam5.li}  Let ${\bf v}_1=(1,2,3), {\bf v}_2=(3,8,1),
\mbox{ and } {\bf v}_3=(4,4,7)$. Determine if the vectors are
linearly independent or linearly dependent
\begin{eqnarray*}
{\bf 0}&=&k_1{\bf v}_1+k_2{\bf v}_2+k_3{\bf v}_3\\
&=&k_1(1,2,3)+k_2(3,8,1)+k_3(4,4,7) \Leftrightarrow
\end{eqnarray*} $$\left \{
\begin{array}{r} k_1+3k_2+4k_3=0\\ 2k_1+8k_2+4k_3=0\\
3k_1+k_2+7k_3=0 \end{array} \right . . $$ This system can be
written in matrix form $$\left [\begin{array}{rrr}
        1&3&4\\
        2&8&4\\
        3&1&7 \end{array} \right ]
         \left [ \begin{array}{c}
                       k_1\\
                       k_2\\
                       k_3 \end{array} \right ]=
         \left [ \begin{array}{c}
                    0\\0\\0 \end{array} \right ].$$
The determinant of the matrix is not zero.  By theorem
\ref{eq3invert} this system will have only the trivial solution
(it is homogeneous).  Thus $k_1=k_2=k_3=0$ and ${\bf v}_1,{\bf
v}_2$ and ${\bf v}_3$ are linearly independent.
\end{example}

\noindent A set $S$ with two or more vectors is:

\parbox{2in}{(a) linearly dependent $\Leftrightarrow$}
\parbox[t]{4in}{at least one of the vectors in $S$ is express-

ible as a linear
combination of the

other vectors.}\\

\parbox{2in}{(b) linearly independent $\Leftrightarrow$}
\parbox[t]{4in}{ no vector in $S$ is expressible as a linear

 combination of the other vectors in $S$.}

\vspace{4mm}
This result is not surprising and explains why we use the
terminology ``linear independence''.  We said originally that a
set of vectors is linearly dependent if at least one $k_i$ is not
zero in the expression $$k_1{\bf v}_1+k_2{\bf v}_2+\cdots+k_n{\bf
v}_n={\bf 0}.$$ Assume that some $k_j$ is not zero. Notice that we
can rewrite the expression as:$$-\frac{k_1}{k_j}{\bf
v}_1-\frac{k_2}{k_j}{\bf v}_2+\cdots- \frac{k_{n}}{k_j}{\bf
v}_{n}={\bf v}_j.$$ Now, notice that we have expressed the vector
${\bf v}_j$ in terms of the other vectors. That is, we have
expressed it as a linear combination of the other vectors. It is
\textit{linearly dependent} on the other vectors. If all constants
are zero we cannot divide by any of them and thus we cannot
express one vector as a linear combination of the others. The set
of vectors must be linearly independent.

A set that contains the zero vector will always be linearly
dependent.  This is due to the fact that the equation $$k_1{\bf
v}_1+k_2{\bf v}_2+\cdots+k_r{\bf v}_r+k_0({\bf 0})={\bf 0}$$ has
solution, $k_1=k_2=\cdots=k_r=0, k_0=1$.

\begin{example}
\label{exam5.li4}
Let $S=\{(x_1,y_1),(x_2,y_2),(x_3,y_3)\}$ be a set
of 3 vectors in $\mbox{\tebbb R}^2$.  Is $S$ independent?

We have to determine whether
$$c_1(x_1,y_1)+c_2(x_2,y_2)+c_3(x_3,y_3)=(0,0)$$
has a unique solution $(c_1=c_2=c_3=0)$.  As a system of equations in $c_1, c_2, c_3$ the augmented matrix is
$$\left [\begin{array}{rrrcr} x_1&x_2&x_3&\vline&0\\
y_1&y_2&y_3&\vline&0\end{array}\right]$$
When we reduce this matrix, there will necessarily be at least one free variable because there are more columns than rows in the matrix.  The existence of a free variable means the solution is not unique, and so the vectors are dependent.
\end{example}

Although we considered three vectors in $\mbox{\tebbb R}^2$, exactly the same considerations would apply if we had $r$ vectors in $\mbox{\tebbb R}^n$ with $r>n$.  The left hand portion of the augmented matrix would have $r$ columns and $n$ rows, hence more columns than rows.  There would be at least one free variable and so a unique solution would be impossible.  Thus when $r>n$, a set of $r$ vectors in $\mbox{\tebbb R}^n$ is dependent.  Put another way, a linearly independent set in $\mbox{\tebbb R}^n$ can have at most $n$ vectors.

\begin{example}
\label{exam5.li5}
Is the set $\{(1,0,6,5), (2,3,0,1), (-1,-2,2,1)\}$ linearly independent?

We must determine whether
$$c_1(1,0,6,5)+c_2(2,3,0,1)+c_3(-1,-2,2,1)=(0,0,0,0)$$
has a unique solution $(c_1=c_2=c_3=0)$.  The matrix for this system of equations is
$$\left [\begin{array}{rrrcr} 1&2&-1&\vline&0\\
0&3&-2&\vline&0\\
6&0&2&\vline&0\\
5&1&1&\vline&0\end{array}\right]
\leadsto
\left[\begin{array}{rrrcr} 1&2&-1&\vline&0\\
0&1&-\frac{2}{3}&\vline&0\\
0&0&0&\vline&0\\
0&0&0&\vline&0\end{array}\right].$$
There is a free variable $(c_3)$, and so the solution is not unique.  The vectors are dependent.  In fact by solving the system of equations we see
$$-1(1,0,6,5)+2(2,3,0,1)+3(-1,-2,2,1)=(0,0,0,0).$$
\end{example}

In summary, if we have $r$ vectors in $\mbox{\tebbb R}^n$, then
\begin{enumerate}
\item  if $r>n$ the vectors are dependent,
\item  if $r=n$, check independence using the determinant (see example \ref{exam5.li}).
\item  if $r<n$, check independence by solving a system of equations (see example \ref{exam5.li5}).
\end{enumerate}

Geometrically, linear dependence of two vectors in 2-space means
that both vectors, with their initial points placed at the same
point, lie on the same line. One vector is a scalar multiple of
the other, thus, it is a linear combination of the other.

In $\mbox{\tebbb R}^3$, if two or three vectors lie on the same
line they will be linearly dependent as in $\mbox{\tebbb R}^2$.
Three vectors, with their initial points all placed at the same
point, will be linearly independent if none of the vectors lie in
the same plane as the other two.

\noindent {\bf Activity \ref{ssec.li}:}

Determine whether the following sets are linearly independent or
linearly dependent.

\begin{enumerate}
\item $S=\{ (2,1,0,1), \ (-1,-2,3,4), \ (3,0,3,6) \}$
\item $S=\{ (1,0,1), \ (-2,1,1), \ (4,-1,3) \}$
\end{enumerate}

The solutions can be found in section \ref{answers5}.

\section{Span of a Set of Vectors} \label{ssec.span} \markright{\ref{ssec.span} \titleref{ssec.span}}

In this section, we define the span of a set of vectors $S=\{{\bf
v}_1,{\bf v}_2, \cdots, {\bf v}_r\}\subseteq V$. The {\bf span}\index{span} of
$S$, written span$(S)$ is the set of all linear combinations of
the vectors is $S$. Hence
$$span(S)=\{ c_1{\bf v}_1+c_2{\bf v}_2+ \cdots+ c_r{\bf v}_r :
c_1,~c_2,\cdots, c_r~\epsilon~\mbox{\tebbb R} \}.$$

\noindent Other common notation for this is $$lin(S), or~
span\{{\bf v}_1,{\bf v}_2, \cdots, {\bf v}_r\}, or~lin\{{\bf
v}_1,{\bf v}_2, \cdots, {\bf v}_r\}.$$

The span of a set of vectors $S$ is always a subspace of $V$. This
is because if $c_1{\bf v}_1+c_2{\bf v}_2+ \cdots+ c_r{\bf v}_r$
and $d_1{\bf v}_1+d_2{\bf v}_2+ \cdots+ d_r{\bf v}_r$ are two
vectors in span$(S)$, then their sum,
$$ (c_1+d_1){\bf v}_1+(c_2+d_2){\bf v}_2+ \cdots+ (c_r+d_r){\bf v}_r$$
can be seen to be a vector in span$(S)$ and so span$(S)$ is closed
under vector addition. Similarly for any scalar $k$,$$k(c_1{\bf
v}_1+c_2{\bf v}_2+ \cdots+ c_r{\bf v}_r)=(kc_1){\bf v}_1+(kc_2)
{\bf v}_2+ \cdots+ (kc_r){\bf v}_r$$ and span$(S)$ is closed under
scalar multiplication.

For some sets of vectors $S$, span$(S)$ will actually equal ${
V}$. In this case we say that $S$ {\bf spans} $V$, or $S$ is a
{\bf spanning set}\index{spanning set} for $V$. The point is that when $S$ spans $V$,
{any} vector {\bf v} can be written as a linear combination
$${\bf v}=c_1{\bf v}_1+c_2{\bf v}_2+ \cdots + c_r{\bf v}_r,$$
though not necessarily uniquely.

The idea of a spanning set is intuitive in $\mbox{\tebbb R}^2$.
For example, because any vector $$(x,y)=x(1,0)+y(0,1),$$ we can
say that the set $\{(1,0),(0,1)\}$ spans $\mbox{\tebbb R}^2$. Here
we are describing $(x,y)$ relative to the $x$ and $y$-axis of
Cartesian geometry. Similarly, $$(x,y)=(x-y)(1,0)+y(1,1).$$ This
means that the set $\{(1,0),(1,1)\}$ also spans $\mbox{\tebbb
R}^2$, because any vector in $\mbox{\tebbb R}^2$ can be written as
a linear combination of $(1,0)$ and $(1,1)$. Note that
$\{(1,0),(0,1),(1,1)\}$ also spans $\mbox{\tebbb R}^2$, because
\begin{eqnarray*}
(x,y)&=&(y-x)(0,1)+x(1,1) \\ &=&-y(1,0)-x(0,1)+(x+y)(1,1)~etc.
\end{eqnarray*}

\noindent In this case, the expression for $(x,y)$ as a linear
combination of $\{(1,0),(0,1),(1,1)\}$ is not unique.

\begin{example}
\label{exam5.span} Determine whether ${\bf v}_1=(1,-2,1)$,${\bf
v}_2=(2,1,-1)$ and ${\bf v}_3=(7,-4,1)$ span the vector space
$\mbox{\tebbb R}^3$.

The question can be reworded to:  can the vector ${\bf
w}=(w_1,w_2,w_3)$ be expressed as a linear combination of ${\bf
v}_1,{\bf v}_2$ and ${\bf v}_3$? That is, can we find $a,b,c$ so
that
\begin{eqnarray*}
(w_1,w_2,w_3)&=&a(1,-2,1)+b(2,1,-1)+c(7,-4,1)\\
&=&(a+2b+7c,-2a+b-4c,a-b+c)? \\
\end{eqnarray*}

As a system of equations:
$$ \begin{array}{rrrrrrr}
                a&+&2b&+&7c&=&w_1\\
              -2a&+& b&-&4c &=&w_2\\
                a&-&b&+&c&=&w_3 \end{array}  $$

Written in matrix form:
                      $$\left [ \begin{array}{rrr}
                                        1&2&7\\
                                        -2&1&-4\\
                                        1&-1&1 \end{array} \right ] \left [ \begin{array}{c} a\\b\\c \end{array}
                                        \right ] = \left [ \begin{array}{c} w_1\\w_2\\w_3 \end{array} \right ] .$$
We must now determine whether this system is consistent.  By
theorem \ref{eq3invert}, there is a solution for all $w_1,w_2,w_3$
if and only if the coefficient matrix is invertible (if the
determinant is non-zero).  In this case, the determinant is zero,
therefore ${\bf v}_1,{\bf v}_2$ and ${\bf v}_3$ do not span
$\mbox{\tebbb R}^3$.
\end{example}
\begin{example}
\label{exam5.span2}
Do the two vectors {$(x_1,y_1,z_1),(x_2,y_2,z_2)$} span $\mbox{\tebbb R}^3$?

We have to determine whether
$$c_1(x_1,y_1,z_1)+c_2(x_2,y_2,z_2)=(w_1,w_2,w_3)$$
has a solution for {\bf all} choices of $w_1,w_2,w_3$.  As a system
of equations in $c_1,c_2,$ the matrix is
$$\left [\begin{array}{rrcr} x_1&x_2&\vline&w_1\\
y_1&y_2&\vline&w_2\\
z_1&z_2&\vline&w_3\end{array}\right].$$
When we reduce the left hand portion of the matrix
to the row-echelon form(using e.g. Gaussian elimination) the bottom row of the reduced matrix becomes
$$\left [\begin{array}{rrcr} 0&0&\vline&\alpha(w_1,w_2,w_3)\end{array}\right],$$
where $\alpha(w_1,w_2,w_3)$ is some function of $w_1,w_2,w_3$ depending on the arithmetic used during the reduction process.  It is guaranteed that the $(3,1)$ and $(3,2)$ places of the reduced matrix are both zero.  This is because there are more rows than columns.  By choosing $w_1,w_2,w_3$ so that $\alpha(w_1,w_2,w_3) \neq 0$, we achieve an inconsistent system of equations.  This means that the $(w_1,w_2,w_3)$ chosen is not in the span of
{$(x_1,y_1,z_1),(x_2,y_2,z_2)$}.  These vectors therefore do not span $\mbox{\tebbb R}^3$.
\end{example}

Although we considered two vectors in $\mbox{\tebbb R}^3$, exactly the same considerations would apply if we had $r$ vectors in $\mbox{\tebbb R}^n$, with $r<n$.  The left hand portion of the augmented matrix would have $r$ columns and $n$ rows, and hence more rows than columns.  This would guarantee a bottom row of
$$\left [\begin{array}{rrrrcr} 0&0&\cdots&0&\vline&\alpha(w_1,\cdots,w_n)\end{array}\right],$$
in the reduced form of the matrix.  By selecting $w_1,w_2,\cdots,w_n$ so that $\alpha(w_1,w_2, $ \\ $\cdots,w_n)\neq 0$ we would obtain an inconsistent system.  Not every vector $(w_1,w_2,$\\$\cdots,w_n)$ could be written as a linear combination of the $r$ given vectors.  The $r$ vectors would therefore not span $\mbox{\tebbb R}^n$.

\begin{example}
\label{exam5.span3}
Do the vectors {$(1,2),(2,-1),(-1,3)$} span $\mbox{\tebbb R}^2$?

We have to determine whether
$$c_1(1,2)+c_2(2,-1)+c_3(-1,3)=(w_1,w_2)$$
has a solution for all choices of $w_1,w_2$.  The matrix of the system of equations is
$$\left [\begin{array}{rrrcr} 1&2&-1&\vline&w_1\\
2&-1&3&\vline&w_2\end{array}\right] \leadsto
\left [\begin{array}{rrrcr} 1&0&1&\vline&\frac{(w_1+2w_2)}{5}\\
0&1&-1&\vline&\frac{(2w_1-w_2)}{5}\end{array}\right].$$
This always has a solution, e.g.
$$c_1=\frac{(w_1+2w_2)}{5},\ c_2=\frac{(2w_1-w_2)}{5},\ c_3=0.$$
Thus the vectors span $\mbox{\tebbb R}^2$.
\end{example}

In summary, if we have $r$ vectors in $\mbox{\tebbb R}^n$ then
\begin{enumerate}
\item  if $r<n$ the vectors do not span $\mbox{\tebbb R}^n$,
\item  if $r=n$, you determine whether the vectors span by evaluating a determinant
(see Example \ref{exam5.span})
\item  if $r>n$, you check whether the vectors span $\mbox{\tebbb R}^n$ by solving
a system of equations (see Example \ref{exam5.span3}).
\end{enumerate}
Many different sets of vectors can span a given vector space.
Geometrically, this is illustrated best with a plane in 3-space.
The plane is spanned by any two non-zero vectors in the plane that are not colinear.

\noindent {\bf Activity \ref{ssec.span}:}

\begin{enumerate}
\item Does $(1,0,1)$ lie in the span of \{$(2,-3,-4),(3,-2,-1)$\}?
\item Does $(4,-6,-7)$ lie in the span of \{$(1,0,1),(-1,2,3)$\}?
\end{enumerate}

\noindent The answers can be found in section \ref{answers5}.

\section{Basis of a Vector Space}\index{basis of a vector space}
\label{ssec.basis}\markright{\ref{ssec.basis}
\titleref{ssec.basis}}

In this section we define a basis of a vector space, and indicate
its importance.

For a vector space $V$, and $S=\{{\bf v}_1,{\bf v}_2, \ldots, {\bf
v}_n\}$ such that $S \subseteq V$, $S$ is called a {\bf basis} for
$V$ if :
\begin{enumerate}
\item $S$ is \textit{linearly independent}
\item $S$ $spans$ $V$.
\end{enumerate}

\noindent{\bf Note:  It is very important at this point that you
fully understand what these two terms really mean!}

The importance of a basis is that if $S=\{{\bf v}_1,{\bf v}_2,
\ldots, {\bf v}_n\}$ is a basis for $V$, then every vector in $V$
can be expressed as a {\bf unique}\index{unique} linear combination of vectors
in $S$. That is, for all ${\bf w} \in V$, there exist unique
$c_1,c_2,\cdots, c_n$ such that
$${\bf w}=c_1{\bf v}_1+c_2{\bf v}_2+\cdots+c_n{\bf v}_n.$$

It was noted in the previous section that if $S$ spans $V$, all
vectors in $V$ must be linear combinations of the vectors in $S$.
Suppose there are two different linear combinations to represent
one vector {\bf w} in $V$.  That is, suppose

\begin{eqnarray}
\label{unique1} {\bf w}&=& c_1{\bf v}_1+ \cdots +c_n{\bf v}_n\\
\label{unique2} {\bf w}&=& k_1{\bf v}_1+ \cdots +k_n{\bf v}_n.
\end{eqnarray}

Now, subtract equations \ref{unique1} and \ref{unique2}.  $${\bf
0}=(c_1-k_1){\bf v}_1 +\cdots +(c_n-k_n){\bf v}_n.$$ But remember
that $S$ is a \textit{basis} for $V$, meaning that the vectors
${\bf v}_1, {\bf v}_2, \ldots, {\bf v}_n$ are \textit{linearly
independent}. Therefore $$(c_1-k_1)=0,\quad \ldots
\quad,(c_n-k_n)=0.$$ So, we have $c_1=k_1, \ c_2=k_2 \ \ldots \
,c_n=k_n$. This shows us that the equations \ref{unique1} and
\ref{unique2} are identical and there is only one expansion for
{\bf w}. The unique representation of a vector ${\bf w} \in V$,
$${\bf w}=c_1{\bf v}_1+c_2{\bf v}_2+\cdots+c_n{\bf v}_n$$ is the
expression for {\bf w} in terms of $S$. Since the scalars
$c_1,c_2,\cdots,c_n$ are uniquely defined given {\bf w} and $S$,
we call $c_1,c_2,\ldots,c_n$ the {\bf coordinates} of {\bf w}
relative to the basis $S$. We represent this symbolically as
$$({\bf w})_S=(c_1,c_2,\ldots, c_n).$$

\begin{example}
\label{exam5.basisR3}

Let $V=\mbox{\tebbb R}^3$, ${\bf i}=(1,0,0)$, ${\bf j}=(0,1,0)$,
${\bf k}=(0,0,1)$ and $S=\{{\bf i,j,k}\}$.  $S$ is linearly
independent and spans $\mbox{\tebbb R}^3$.  Let ${\bf v}= (a,b,c)$
be a vector in $\mbox{\tebbb R}^3$, in terms of the basis $S$.  So
$(a,b,c)=a{\bf i}+b{\bf j}+c{\bf k}$ and $({\bf v})_S=(a,b,c)$.
$S$ is the {\bf standard basis} for $\mbox{\tebbb R}^3$.
\end{example}

\begin{example}
\label{exam5.basisRn}

Let $V=\mbox{\tebbb R}^n$, ${\bf e}_1=(1,0,0,\ldots,0)$, ${\bf
e}_2=(0,1,0,\ldots,0), \ldots$, ${\bf e}_n=(0,0,\ldots,1)$, and
$S=\{ {\bf e}_1,{\bf e}_2,{\bf e}_3,\ldots, {\bf e}_n\}$.  $S$ is
called the {\bf standard basis} for $\mbox{\tebbb R}^n$.
\end{example}

\noindent When a vector is expressed in terms of the standard
basis, by convention no subscript is needed.  The vector ${\bf
v}=(4,3,2)$ is actually $4{\bf i}+3{\bf j}+2{\bf k}$. The standard
basis is only one example of a basis.

There are many bases for the same vector space, as the following
example shows.

\begin{example}
\label{exam5.isitbasis} Let ${\bf v}_1=(1,3,-1)$, ${\bf
v}_2=(1,4,3)$ and ${\bf v}_3=(1,5,8)$.  Is $S=\{ {\bf v}_1, {\bf
v}_2,{\bf v}_3\}$ a basis for $\mbox{\tebbb R}^3$?

Condition 1:  $S$ spans $\mbox{\tebbb R}^3$.  Let ${\bf
b}=(b_1,b_2,b_3)$ be any vector in  $\mbox{\tebbb R}^3$. To
determine whether the set of vectors spans $\mbox{\tebbb R}^3$, we
need to find a linear combination to represent {\bf b} in terms of
$S$. Mathematically, we need to solve

$$\begin{array}{ccc} {\bf b}&=&a{\bf v}_1+b{\bf v}_2+c{\bf v}_3\\
(b_1,b_2,b_3)&=&a(1,3,-1)+b(1,4,3)+c(1,5,8)\\ \left [
\begin{array}{c} b_1\\b_2\\b_3 \end{array} \right ]&=& \left [
\begin{array}{rrr} 1&1&1\\
    3&4&5\\
    -1&3&8  \end{array} \right ] \left [ \begin{array}{c}
    a\\b\\c \end{array} \right ]
\end{array}$$
This is a linear system, with coefficient matrix $$A=\left [
\begin{array}{rrr}
    1&1&1\\
    3&4&5\\
    -1&3&8 \end{array} \right ] \mbox{ and } \rm{det}(A)=1 \neq 0.$$
    The determinant of the coefficient matrix
is not 0, thus by theorem \ref{eq3invert}, there is a solution to
the system for all {\bf b}. This means that for all {\bf b} in
$\mbox{\tebbb R}^3$, a solution $(a,b,c)$ can be found, and thus
{\bf b} can always be written in terms of $S$. In other words, $S$
spans $\mbox{\tebbb R}^3$.

 Condition 2:  $S$ is linearly independent.
Recall that the system is linearly independent if $$a{\bf
v}_1+b{\bf v}_2+c{\bf v}_3={\bf 0} \Rightarrow a=b=c=0 \mbox{ is
the only solution }.$$  Again, the coefficient matrix for the
system has a nonzero determinant, so by theorem \ref{eq3invert},
the only solution to the system is \mbox{$a=b=c=0$}. Therefore,
$S$ is linearly independent.

Both conditions for a basis are met, therefore $S$ is a basis for
$\mbox{\tebbb R}^3$.
\end{example}

\begin{example}
\label{exam5.wrtbasis} Express the vector ${\bf v}=(0,1,3)$ with
respect to $S$ the basis given in example \ref{exam5.isitbasis}.
$$ (0,1,3)=c_1(1,3,-1)+c_2(1,4,3)+c_3(1,5,8)$$
$$\Rightarrow \left
\{ \begin{array}{rrrrrrr}
                                c_1&+&c_2&+&c_3&=&0\\
                                3c_1&+&4c_2&+&5c_3&=&1\\
                                -c_1&+&3c_2&+&8c_3&=&3 \end{array} \right
                                . . $$
Using any of the methods described to solve a system of linear
equations yields the answer $c_1=-2$, $c_2=3$ and $c_3=-1$.
Therefore the vector ${\bf v}=(0,1,3)$ expressed in terms of $S$
is $({\bf v})_S=(-2,3,-1)$. \end{example}

\begin{example}
\label{exam5.wrtbasisR3} Express the vector $({\bf v})_S=(0,1,3)$
where $S$ is from example \ref{exam5.isitbasis}, in terms of the
standard basis of $\mbox{\tebbb R}^3$.
\begin{eqnarray*}
{\bf v}&=&0{\bf v}_1+1{\bf v}_2+3{\bf v}_3\\
&=&1(1,4,3)+3(1,5,8)\\ &=&(4,19,27) \\ &=& 4{\bf i}+19{\bf
j}+27{\bf k} .
\end{eqnarray*}
\end{example}

Let $S=\{{\bf v}_1,{\bf v}_2, \ldots, {\bf v}_n\}$ be any linearly
independent set of vectors in $V$.  The linear span of $S$ is a
vector subspace of $V$. Since $S$ is linearly independent and
spans lin($S$), $S$ will be a basis for lin($S$).

{\bf Activity \ref{ssec.basis}:}

\begin{enumerate}
\item Using the vectors from activity \ref{ssec.span}, which of
the two sets of vectors form a basis for
$W=\rm{lin}(S)=\rm{lin}(S')$?
\item Let $S=\{{\bf v}_1=(1,0,1,0), {\bf v}_2=(-1,2,2,0), {\bf v}_3=(0,3,4,1),
{\bf v}_4=(1,-1,$\\$0,-1) \}$ span $V$. Note that ${\bf w}={\bf
v}_1-{\bf v}_2+{\bf v}_3$ and ${\bf w}=-{\bf v}_1-2{\bf v}_2+2{\bf
v}_3+{\bf v}_4$. In this section we stated that every vector in
$V$ can be expressed as a unique linear combination of the basis
vectors. Does this example contradict that statement? Why or why
not?
\end{enumerate}

\noindent Answers can be found in section \ref{answers5}

\section{Dimension of a Vector Space}\index{dimension of a vector space}
\label{ssec.dimension} \markright{\ref{ssec.dimension}
\titleref{ssec.dimension}}

In this section, we define the dimension of a vector space. This
will give us some idea as to the nature of the sizes of vector
spaces. We think (intuitively) that this piece of paper is
two-dimensional and that the space we live in is three
dimensional. What, for example, is the dimension of the space
spanned by three $4$-tuple vectors?

The dimension of a vector space $V$ is defined to be the number of
vectors in a basis for $V$. This only makes sense if all bases for
$V$ have the same number of vectors. To see that this is true, we
consider a set of vectors $S=\{ {\bf u}_1, {\bf u}_2,{\bf u}_3\}$
in $V$, and $B=\{ {\bf v}_1, {\bf v}_2\}$ a basis of $V$. We ask
the question: is $S$ dependent or independent?

For $S$ to be independent, the equation
\begin{equation} \label{unique3}
c_1{\bf u}_1+c_2{\bf u}_2+c_3{\bf u}_3={\bf 0}
\end{equation}
must have a unique solution $(c_1=c_2=c_3=0)$. Since $\{ {\bf
v}_1, {\bf v}_2 \}$ is a basis for $V$ and ${\bf u}_1,{\bf
u}_2,{\bf u}_3 \in V$, we can write
\begin{eqnarray*}
{\bf u}_1&=&a_{11}{\bf v}_1+a_{21}{\bf v}_2 \\
{\bf u}_2&=&a_{12}{\bf v}_1+a_{22}{\bf v}_2 \\
{\bf u}_3&=&a_{13}{\bf v}_1+a_{23}{\bf v}_2
\end{eqnarray*}
for scalars $a_{ij}$. The equation \ref{unique3} becomes
$$c_1(a_{11}{\bf v}_1+a_{21}{\bf v}_2)+c_2(a_{12}{\bf v}_1+a_{22}{\bf
v}_2)+c_3(a_{13}{\bf v}_1+a_{23}{\bf v}_2)={\bf 0}, $$
$$i.e.~(c_1a_{11}+c_2a_{12}+c_3a_{13}){\bf v}_1+ (c_1a_{21}+c_2a_{22}+c_3a_{23}){\bf
v}_2={\bf 0}.$$ Since the set $\{ {\bf v}_1,{\bf v}_2 \}$ is
independent (it is a basis) we have
\begin{eqnarray*}
c_1a_{11}+c_2a_{12}+c_3a_{13}&=&0 \\
c_1a_{21}+c_2a_{22}+c_3a_{23}&=&0.
\end{eqnarray*}
The augmented matrix for this system of equations (in the unknowns
$c_1$, $c_2$, $c_3$) is $$ \left [ \begin{array}{rrrcr}
        a_{11} & a_{12} & a_{13} & \vline& 0 \\
        a_{21} & a_{22} & a_{23} & \vline& 0 \end{array} \right ].$$
When this is reduced, there must result a free (or non-leading)
variable, because there are more unknowns than equations. Thus
there is an infinite number of solutions for $c_1$, $c_2$, $c_3$.
Thus equation \ref{unique3} does not have a unique solution and
the set ${{\bf u}_1, {\bf u}_2, {\bf u}_3}$ is dependent.

This shows that if there is a basis with 2 vectors, then any set
of 3 vectors is dependent. It is an illustration of the more
general fact: {\em if a basis $\mbox {\tebbb B}$ has n vectors,
then any set S with more than n vectors is dependent}. This fact
is worth remembering.

The contrapositive of the last statement is: {\em if a set S is
independent and $\mbox {\tebbb B}$ is a basis}, {\em then $\#S
\leq \#\mbox {\tebbb B}$ }, where $\#S, \#\mbox {\tebbb B}$
denotes the number of vectors in the set. This fact is worth
remembering too.

Now let $\mbox {\tebbb B}$, $\mbox {\tebbb B}'$ be two bases of
the same vector space. Then
\begin{eqnarray*}
&~&\mbox {\tebbb B}~is~a~basis~and ~\mbox {\tebbb
B}'~is~independent,~so ~\#\mbox {\tebbb B}'\leq \#\mbox
{\tebbb B}, \\
&~&\mbox {\tebbb B}'~is~a~basis~and~ \mbox {\tebbb
B}~is~independent,~so ~\#\mbox {\tebbb B}\leq \#\mbox {\tebbb B}'.
\end{eqnarray*}
Thus any of two bases of a vector space $V$ have the same number
of vectors. This number is called the {\bf dimension} of $V$ or
$dim(V)$. For example, $dim \mbox {\tebbb R}^n=n$ because there
are $n$ vectors in the standard basis ${ {\bf e}_1,{\bf
e}_2,\ldots,{\bf e}_n}$. Any other basis of $\mbox {\tebbb R}^n$
will have $n$ vectors.

\begin{example}
\label{exam5.basishomog} Find a basis and the dimension for the
solution space (set of all solutions) to the homogeneous system:
\begin{eqnarray*}
x_1-x_3-x_4+x_5&=&0\\ 2x_1+x_2-x_4+2x_5&=&0 \quad . \\
2x_1+x_2+x_3-x_4+2x_5&=&0\\ x_2+x_3+x_4&=&0 \end{eqnarray*} There
are 4 equations in 5 unknowns, therefore, there are infinitely
many solutions. Row reducing the matrix of the system yields
$$\left[ \begin{array}{rrrrr} 1&0&-1&-1&1 \\ 2&1&0&-1&2 \\ 2&1&1&-1&2
\\ 0&1&1&1&0 \end{array} \right] {\leadsto} \left[
\begin{array}{rrrrr} 1&0&-1&-1&1 \\ 0&1&2&1&0 \\ 0&0&1&0&0
\\ 0&0&0&0&0 \end{array} \right].$$ The solution set is therefore
$$x_1=s-t \quad x_2=-s \quad x_3=0 \quad x_4=s \quad x_5=t,$$
$$i.e.~\{ s(1,-1,0,1,0)+t(-1,0,0,0,1):s,~t~\epsilon~\mbox {\tebbb
R} \}.$$

Notice that all solutions {\bf x} can be written as a linear
combination of the two vectors $${\bf v}_1=(1,-1,0,1,0),~{\bf
v}_2=(-1,0,0,0,1).$$ These vectors are linearly independent and
therefore form a basis for the solution space of the system.
Therefore {\rm lin}$\{{\bf v}_1,{\bf v}_2\}$ inside $\mbox{\tebbb
R}^5$ is a 2-dimensional vector space.

\end{example}

Let $V$ be the vector space of dimension $n$, and let $S$ be a
subset of $V$. The following table indicates the independence of
$S$ and whether $S$ spans $V$, depending on $\#S$ (the number of
vectors in $S$).
$$\#S$$
$$\cdots~~~n-2~~~n-1~|~n~|~n+1~~~n+2~~~\cdots$$

\begin{center}
\begin{picture}(35,35)(-50,-50)
\put(-180,-2){\line(1,0){300}}
\put(-160,-25){S~possibly~independent} \put(5,-25){S~dependent}
\put(-150,-50){S~does~not~span~V} \put(-5,-50){S~possibly~spans~V}
\put(-23,-27){\line(0,1){10}} \put(-40,-51){\line(0,1){10}}
\put(-162,-23){\vector(-1,0){20}} \put(1,-23){\vector(-1,0){23}}
\put(-40,-23){\vector(1,0){17}} \put(70,-23){\vector(1,0){50}}
\put(-155,-47){\vector(-1,0){28}} \put(-11,-47){\vector(-1,0){28}}
\put(-54,-47){\vector(1,0){15}} \put(90,-47){\vector(1,0){30}}
\end{picture}
\end{center}

A basis of $V$ has $n$ vectors. From the table you can see that a
basis is a minimal spanning set (in the sense that if you omit any
vector the set would no longer span) and a maximal independent set
(in the sense that if you append a vector to the set, the set
would no longer be independent).

When $S$ has precisely $n$ elements, then it is independent if and
only if it spans $V$. To see that this is true, consider the three
vectors ${(x_1,y_1,z_1),(x_2,y_2,z_2),(x_3,y_3,z_3)}$ in $\mbox
{\tebbb R}^3$ (which has dimension $3$). These vectors are
independent if and only if
$$c_1(x_1,y_1,z_1)+c_2(x_2,y_2,z_2)+c_3(x_3,y_3,z_3)=(0,0,0)$$
has a unique solution, i.e. if and only if
$$ \left[ \begin{array}{rrrcr} x_1 & x_2 & x_3 & \vline & 0
\\ y_1 & y_2 & y_3 & \vline & 0 \\ z_1 & z_2 & z_3 & \vline & 0
\end{array}  \right] \quad reduces~to \quad \left [
\begin{array}{rrrcr}
1 & 0 & 0 & \vline & 0 \\ 0 & 1 & 0 & \vline & 0 \\
0 & 0 & 1 & \vline & 0 \end{array} \right ]. $$

On the other hand, the vector span $\mbox {\tebbb R}^3$ if for any
$(a,b,c)$ we can find $c_1,c_2,c_3$ such that
$$c_1(x_1,y_1,z_1)+c_2(x_2,y_2,z_2)+c_3(x_3,y_3,z_3)=(a,b,c),$$
i.e. if and only if
$$ \left[ \begin{array}{rrrcr} x_1 & x_2 & x_3 & \vline & a
\\ y_1 & y_2 & y_3 & \vline & b \\ z_1 & z_2 & z_3 & \vline & c
\end{array}  \right]~is~a~consistent~system.$$
This will be the case precisely when the matrix reduces to
$$ \left[ \begin{array}{rrrcr} 1 & 0 & 0 & \vline & \alpha
\\ 0 & 1 & 0 & \vline & \beta \\ 0 & 0 & 1 & \vline & \gamma
\end{array}  \right]$$
for some $\alpha,\beta,\gamma$. The conditions for independence
and spanning are exactly the same:
$$ \left[ \begin{array}{rrr} x_1 & x_2 & x_3 \\
y_1 & y_2 & y_3 \\ z_1 & z_2 & z_3 \end{array}  \right] \quad
reduces~to \quad \left [ \begin{array}{rrr} 1 & 0 & 0 \\ 0 & 1 & 0
\\ 0 & 0 & 1 \end{array} \right ], $$
or
$$det\left[ \begin{array}{rrr} x_1 & x_2 & x_3 \\
y_1 & y_2 & y_3 \\ z_1 & z_2 & z_3 \end{array}  \right]\neq 0.$$
In the same way we can conclude that $n$ vector in $\mbox {\tebbb
R}^n$ (or any vector space of dimension $n$) are independent if and only if
they span the space. An easy way to check is via the determinant
as above.

\noindent {\bf Activity \ref{ssec.dimension}:} Let $S=\{{\bf
v}_1=(2,1,0), {\bf v}_2=(0,3,-1)\}$. Find a third vector so that
the set is a basis for $\real{3}$. The answers can be found in
section \ref{answers5}

\section{Summary} \label{ssec.sum5}\markright{\ref{ssec.sum5} \titleref{ssec.sum5}}

{\bf Keywords: basis of a vector space, dimension of a vector space, subspace, span, vector space }
\\ \\
This chapter contains some abstract concepts. A vector space $V$ is
defined as a set that satisfies ten given axioms. Subspaces are
defined as non-empty subsets of $V$ that are closed under scalar
multiplication and vector addition. If $c_1,c_2,\cdots,c_n$ are
scalars the expression $c_1{\bf v}_1+c_2{\bf v}_2+\cdots+c_n{\bf
v}_n$ is a linear combination of the vectors ${\bf v}_1,{\bf
v}_2,\cdots,{\bf v}_n$. The set of all linear combinations of
$S=\{ {\bf v}_1,{\bf v}_2,\cdots,{\bf v}_n \}$ is a subspace of
$V$ called the span of $S$. The set $S$ is (linearly) independent
if $c_1{\bf v}_1+c_2{\bf v}_2+\cdots+c_n{\bf v}_n={\bf
0}\Rightarrow c_1=c_2=\cdots=c_n=0$, or equivalently if none of
the ${\bf v}_1,{\bf v}_2,\cdots,{\bf v}_n$ is a linear combination
of the others. A set of vectors that is both linearly independent
and spans $V$ is a basis for $V$. Every vector in $V$ can be
uniquely written as a linear combination of the basis vectors:
$${\bf v}=a_1{\bf v}_1+a_2{\bf v}_2+\cdots+a_n{\bf v}_n.$$ The
scalars $a_1,a_2,\cdots,a_n$ are the coordinates of {\bf v} with
respect to the basis $\{ {\bf v}_1,{\bf v}_2,\cdots,{\bf v}_n \}$.
A vector space has more than one basis, but all bases of the same
vector space have the same number of vectors. This number is
called the dimension of $V$. The dimension of $\mbox {\tebbb R}^n$
is $n$.

\subsection{Exercises} \begin{enumerate}
\item   Suppose $X$ is a random variable which follows a normal distribution with $\mu=20$ and $\sigma=4$.Find the $z$ score that corresponds to each of the following $x$ value:

\begin{enumerate}
\item $x=20$
\item $x=30$
\item $x=17$
\item $x=2.88$
\end{enumerate}

\item Suppose $X$ follows the same distribution as question 1.Find each of the following probabilities :

\begin{enumerate}
\item $P(10 \leq X <12)$
\item$P(13 \leq X <16)$
\item $P(X \geq 7.62)$
\item $P(X <13.24)$
\end{enumerate}

\item Suppose $X$ is a random variable which follows a normal distribution with $\mu=45$ and $\sigma=3$.Find a value of the random variable,call it $x_{0}$,such that:

\begin{enumerate}
\item $P(X>x_{0})=0.025$
\item $P(X>x_{0})=0.95$
\item $P(X \leq x_{0})=0.8413$
\end{enumerate}


\end{enumerate}

%\end{document}
