\chapter{Sample Size Selection using Confidence Intervals}

In this section we will examine techniques to calculate the minimum sample size required to obtain a confidence interval to be within a specified margin of error.
Recall the one-sample confidence intervals we have constructed for a population mean $\mu$ are



%\begin{alignat*}{3}
%& \bar{x} && 	\pm z_{\alpha/2} \frac{\sigma}{\sqrt{n}}	&&	(when $\sigma$ is known)	\\
%& \bar{x} && 	\pm t_{\alpha/2}	 \frac{s}{\sqrt{n}}		&&	(when $\sigma$ is not known)	\\
%\end{alignat*}

\begin{alignat*}{3}
    & \bar{x} \, \pm \, z_{\alpha/2} \frac{\sigma}{\sqrt{n}}	&& \quad \text{(When } \sigma \text{ is known)}	\\[1.0em]
    & \bar{x} \, \pm \, t_{\alpha/2, n-1} \frac{s}{\sqrt{n}}		&& \quad	\text{(When } \sigma \text{ is not known)}
\end{alignat*}

and the one-sample confidence interval for a proportion is

\[
\hat p \pm z_{\alpha/2} \sqrt{\frac{\hat p(1-\hat p)}{n}}
\]

%\begin{center}
%\begin{tabular}{llllllllll}
%$\bar{x} \, \pm \, z_{\alpha/2} \frac{\sigma}{\sqrt{n}}$	& (when $\sigma$ is known)		\\[1.0em]
%$\bar{x} \, \pm \, t_{\alpha/2, n-1} \frac{s}{\sqrt{n}}$	& (when $\sigma$ is not known)	\\[1.0em]
%
%$\hat{p} \, \pm \, z_{\alpha/2} \sqrt{\frac{\hat p(1-\hat p)}{n}}$
%\end{tabular}
%\end{center}

%\begin{align*}
%\bar x \pm z_{\alpha/2}\,\frac{\sigma}{\sqrt{n}}
%\quad
%(\sigma\ \text{known}) \\
%\bar x \pm t_{n-1,\;\alpha/2}\,\frac{s}{\sqrt{n}}
%\quad
%(\sigma\ \text{unknown})
%\end{align*}

%\subsection*{One–Sample Confidence Interval for a Proportion}
%\[
%\hat p \pm z_{\alpha/2}
%\sqrt{\frac{\hat p(1-\hat p)}{n}}
%\]


\subsection{Empirical Rule}

For any sample from a population that is close to Normally distributed:

\begin{itemize}
    \item about 68\% of all observations will lie in the interval \(\mu \pm \sigma\)
    \item about 95\% of all observations will lie in the interval \(\mu \pm 2\sigma\)
    \item about 99.7\% of all observations will lie in the interval \(\mu \pm 3\sigma\)
\end{itemize}

This suggests that for a sample drawn from a population that is approximately to Normally distributed, we can approximate the standard deviation using

\[
\hat{\sigma} \approx \frac{\text{Sample Range}}{4}.
\]




\section{Calculating Sample Size for a Confidence Interval on a Mean}
\label{secSampleSizeCIMean}

We will examine how to calculate the minimum sample size $n$ for a confidence interval on a mean for a margin of error $E$ at confidence level $1-\alpha$.
\subsection*{When \(\sigma\) is Known}
For a desired margin of error \(E\) and confidence level \(1-\alpha\):
%\[
%E = z_{\alpha/2}\,\frac{\sigma}{\sqrt{n}}
%\quad\Longrightarrow\quad
%n = \big(\frac{z_{\alpha/2}\,\sigma}{E}\biggr)^2,
%\]
\[
E = z_{\alpha/2}\,\frac{\sigma}{\sqrt{n}}
\]


We can rearrange $E$ to calculate the sample size using
\[
	n = \left( \frac{z_{\alpha/2}\,\sigma}{E} \right)^2
\]
where $n$ is always \emph{rounded up} to the next integer.\\

%\subsection*{Empirical Rule (Crude \(\sigma\) Estimate)}
%If the underlying population is approximately Normal:
%\begin{itemize}
%  \item About 68\% of observations lie in \(\mu \pm \sigma\).
%  \item About 95\% of observations lie in \(\mu \pm 2\sigma\).
%  \item About 99.7\% of observations lie in \(\mu \pm 3\sigma\).
%\end{itemize}
%Hence one can use \(\sigma \approx \tfrac{\text{sample range}}{4}\) as a rough estimate.

%\section{Worked Example: Pharmaceutical Analysis}

\begin{example}
\label{exSampleSizePharmaceutical}
%A laboratory measures the concentration of an active ingredient; each
%measurement follows \(\mathcal{N}(\mu,\,\sigma=0.0068)\). Management
%requires accuracy \(\pm0.005\) (grams per liter) with 95\% confidence.

A manufacturer of pharmaceutical products analyzes a specimen from each batch of a product to verify the concentration of the active ingredient. The chemical analysis is not perfectly precise. Repeated measurements on the same specimen give slightly different results. Suppose we know that the results of repeated measurements follow a Normal distribution with mean \(\mu\) equal to the true concentration and standard deviation \(\sigma = 0.0068\) grams per liter. (That the mean of the population of all measurements is the true concentration says that the measurements process has no bias. The standard deviation describes the precision of the measurement.) The laboratory analyzes each specimen \(n\) times and reports the mean result.\\

Management asks the laboratory to produce results accurate to within \(\pm 0.005\) with 95\% confidence. How many measurements must be averaged to comply with this request?


\[
n = \left( \displaystyle \frac{z_{0.025}\,\sigma}{E} \right)^2
    = \Bigl(\frac{1.96 \times 0.0068}{0.005}\Bigr)^2
    \approx 7.1
\]
Since the sample size should be a whole number, we round our result up to \(n=8\) measurements.
\end{example}

In Example \ref{exSampleSizePharmaceutical}, we note that 7 measurements will give a slightly larger margin of error than desired, and 8 measurements a slightly smaller margin of error, the lab must take 8 measurements on each specimen to meet management’s demand. Always round up to the next higher whole number when finding \(n\).


\begin{example}

Planning value \(\sigma=22.50\), desired margin \(E=2\).

\begin{enumerate}
  \item 90\% confidence, \(z_{0.05}=1.65\):
    \[
    n = \Bigl(\frac{1.65 \times 22.50}{2}\Bigr)^2
      \approx 344.6
    \quad\Longrightarrow\quad n = 345.
    \]
  \item 95\% confidence, \(z_{0.025}=1.96\):
    \[
    n = \Bigl(\frac{1.96 \times 22.50}{2}\Bigr)^2
      \approx 486.2
    \quad\Longrightarrow\quad n = 487.
    \]
  \item 99\% confidence, \(z_{0.005}=2.58\):
    \[
    n = \Bigl(\frac{2.58 \times 22.50}{2}\Bigr)^2
      \approx 842.5
    \quad\Longrightarrow\quad n = 843.
    \]
\end{enumerate}


\end{example}

%\section{Homework Problems}
%
%\subsection*{Rod Lengths}
%Construct a 95\% CI for the mean length of iron rods, knowing lengths range 0.96 m to 1.04 m, and the total width of the CI should be 0.05 m. Find \(n\).
%
%\subsection*{Rainfall pH}
%Estimate the mean pH within \(\pm0.1\) with 99\% confidence, given \(\sigma\approx0.5\). Find \(n\).


\section{Calculating Sample Size for a Confidence Interval on a Proportion}



We will examine how to calculate the minimum sample size $n$ for a confidence interval on a proportion for a margin of error $E$ at confidence level $1-\alpha$.
For sample of size \(n\) with unknown population proportion \(p\):
\[
\hat p \pm z^* \sqrt{ \displaystyle \frac{\hat p(1-\hat p)}{n}}.
\]
where the margin of error is
\[
E=z^*\sqrt{ \displaystyle \frac{p^*(1-p^*)}{n}}
\]
we can rearrange $E$ to calculate the sample size using
\[
n = \left( \displaystyle \frac{z^*}{E} \right)^2\,p^*(1-p^*),
\]
where \(p^*\) can be a \emph{planning value}, which is value obtained from prior information such as a pilot study.
If we do not have any prior information on \(p^*\), we can use \(p^* = 0.5\) which is the conservative value that maximizes the product $p^*(1-p^*)$.




\begin{example}
Aisha Shariff and Yvette Ye are the candidates for mayor in a large city. You are planning a sample survey to determine what percent of the voters plan to vote for Shariff. This is a population proportion \(p\). You will contact an SRS of registered voters in the city. You want to estimate \(p\) with 95\% confidence and a margin of error no greater than 3\%, or 0.03. How large a sample do you need?\\


For a 95\% CI on \(p\): \(z_{0.025} = 1.96\). Margin of error = 0.03. Since no information on a good estimate of \(p\), use \(p^{*} = 0.5\).

\[
1.96 \sqrt{\frac{(0.5)(1 - 0.5)}{n}} \le 0.03
\quad\Longrightarrow\quad
n = \left(\frac{1.96}{0.03}\right)^{2} (0.5)(0.5) \approx 1067.1.
\]
Round up:
\[
n = 1068.
\]

\end{example}






\begin{example}
The percentage of people not covered by health care insurance in 2007 in the USA was 15.6\%. A congressional committee has been charged with conducting a sample survey to obtain more current information.
\begin{enumerate}
    \item What sample size would you recommend if the committee’s goal is to estimate the current proportion of individuals without health care insurance with a margin of error of 0.03? Use a 95\% confidence level.
    \item Repeat part (a) using a 99\% confidence level.
\end{enumerate}


\[
\text{(a) } n 
= 
\left(\frac{z^{*}}{E}\right)^{2}
p^{*}(1-p^{*}), 
\quad z^{*} = 1.96, \; E = 0.03,\; p^{*} = 0.156.
\]
\[
n
=
\left(\frac{1.96}{0.03}\right)^{2}
(0.156)(1 - 0.156)
\approx 563.
\]

\[
\text{(b) } n 
= 
\left(\frac{2.58}{0.03}\right)^{2}
(0.156)(1 - 0.156)
\approx 974.
\]

\end{example}




\begin{example}

A consumer advocacy group would like to find the proportion of consumers who bought the newest generation of iPhone and were happy with their purchase. How large a sample should they take to estimate \(p\) with 2\% margin of error and 90\% confidence?


\textbf{Parameters:}
\[
\begin{aligned}
E &= 0.02, \quad\text{(margin of error)}\\
\text{Confidence level} &= 0.90 \;\Longrightarrow\; \alpha = 0.10,\;\alpha/2 = 0.05,\\
z^{*} &= z_{0.05} = 1.645,\\
p^{*} &= 0.5,\quad p^{*}(1 - p^{*}) = 0.5 \times 0.5 = 0.25.
\end{aligned}
\]

\bigskip

\textbf{Sample‐Size Formula:}
\[
n \;=\; \left(\frac{z^{*}}{E}\right)^{2}\;p^{*}(1 - p^{*}).
\]

\noindent Substituting \(z^{*} = 1.645\), \(E = 0.02\), and \(p^{*}(1 - p^{*}) = 0.25\):
\[
\begin{aligned}
n 
&= \left(\frac{1.645}{0.02}\right)^{2} \times 0.25 
= \bigl(82.25\bigr)^{2} \times 0.25 
= 6{,}764.0625 \times 0.25 \\[0.5em]
&= 1{,}691.015625.
\end{aligned}
\]

Since \(n\) must be a whole number and we always round up to ensure the margin of error is at most \(2\%\), we take
\[
n = 1{,}692.
\]
%
%\bigskip
%
%\textbf{Conclusion:} The required sample size is
%\[
%\boxed{n = 1{,}692.}
%\]

\end{example}


