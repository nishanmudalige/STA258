\setcounter{chapter}{3} % Makes this Chapter 4
\chapter{Normal Approximation to the Binomial Distribution}

\section{Definitions and Setup}

\begin{definitionbox}{Statistic}
A \textit{statistic} is a function of the observable random variables in a sample and known constants.

Since statistics are functions of the random variables observed in a sample, they themselves are random variables. As such, all statistics have a corresponding probability distribution, which we refer to as their \textit{sampling distribution}.
\end{definitionbox}
\begin{tcolorbox}[title=Review from STA256,
  colback=gray!5, 
  colframe=gray!40!black, 
  coltitle=black, 
  colbacktitle=gray!20,  % <- softer title bar
  fonttitle=\bfseries,
  sharp corners=south,
  breakable]


\textbf{Bernoulli Distribution:}

A Bernoulli trial is a single experiment with two outcomes:
\begin{itemize}
  \item Success: \( X = 1 \) with probability \( p \)
  \item Failure: \( X = 0 \) with probability \( 1 - p \)
\end{itemize}

\medskip

\begin{center}
\renewcommand{\arraystretch}{1.3}
\begin{tabular}{|c|c|c|}
\hline
\( X = x \) & 0 & 1 \\
\hline
\( P(X = x) \) & \( 1 - p \) & \( p \) \\
\hline
\end{tabular}
\end{center}

The probability mass function (PMF) is:
\[
f(x) = p^x (1 - p)^{1 - x}, \quad x \in \{0, 1\}
\]

\medskip

\textbf{Binomial Distribution:}

A binomial distribution arises from \( n \) independent Bernoulli trials. Let:
\[
X = \text{number of successes in } n \text{ trials}
\]
Then:
\[
X \sim \text{Binomial}(n, p)
\]

where:
\begin{itemize}
  \item Each trial results in either success (with probability \( p \)) or failure (with probability \( 1 - p \))
  \item \( X \in \{0, 1, \dots, n\} \)
\end{itemize}

The PMF is:
\[
P(X = x) = \binom{n}{x} p^x (1 - p)^{n - x}
\]

\medskip

\textbf{Moment Generating Function (MGF):}

The moment generating function (MGF) of a random variable \( X \) is defined as:
\[
M_X(t) = \mathbb{E}[e^{tX}]
\]
The MGF uniquely characterizes the distribution of \( X \) (if it exists in an open interval around 0), and it can be used to compute moments such as the mean and variance.


\end{tcolorbox}

\section{Bernoulli Distribution}

Consider the random experiment of rolling a die once. Define the random variable:

\[
X_i =
\begin{cases}
1 & \text{if the } i\text{-th roll is a six}, \\
0 & \text{otherwise}
\end{cases}
\]

Then \( X_i \sim \text{Bernoulli}(p) \), where \( p = P(\text{rolling a six}) \). 


The expected value and variance of a Bernoulli random variable are:

\[
\mu = \mathbb{E}(X) = p, \quad \sigma^2 = \mathrm{Var}(X) = p(1 - p)
\]



\section{Sampling Distribution of the Sum and MGF Derivation}

Consider determining the sampling distribution of the sample total:
\[
T_n = X_1 + X_2 + \dots + X_n
\]
Suppose \( X_i \overset{iid}{\sim} \text{Bernoulli}(p) \). Then the moment-generating function of \( T_n \) is:

\begin{align*}
M_{T_n}(t) &= \mathbb{E}[e^{t T_n}] \\
           &= \mathbb{E}\left[e^{t(X_1 + X_2 + \dots + X_n)}\right] \\
           &= \mathbb{E}\left[e^{tX_1} e^{tX_2} \dots e^{tX_n} \right] \quad \text{(independence)} \\
           &= \mathbb{E}[e^{tX_1}] \cdot \mathbb{E}[e^{tX_2}] \cdots \mathbb{E}[e^{tX_n}] \\
           &= M_{X_1}(t) \cdot M_{X_2}(t) \cdots M_{X_n}(t) \\
           &= \left[pe^t + (1 - p)\right]^n
\end{align*}


Since this is the MGF of a binomial random variable with parameters \( n \) and \( p \), we conclude:

\[
T_n \sim \text{Binomial}(n, p)
\]
\medskip
\begin{tcolorbox}[title=Example: Binomial Distribution from Die Rolls,
  colback=blue!5, colframe=blue!40!black, coltitle=black,
  colbacktitle=blue!10!white, fonttitle=\bfseries, breakable]

We can think of rolling a die \( n \) times as an example of the binomial setting. Each roll gives either a six (a “success”) or a number different from six (a “failure”).

Knowing the outcome of one roll doesn’t tell us anything about the others, so the \( n \) rolls are independent.

If we call a six a success, then:
\begin{itemize}
  \item The probability of success on each trial is \( p = P(\text{rolling a six}) = \frac{1}{6} \)
  \item The probability of failure is \( 1 - p = \frac{5}{6} \)
\end{itemize}

Let \( Y \) be the number of sixes rolled in \( n \) trials. Then \( Y \sim \text{Binomial}(n, p) \), and the distribution of \( Y \) is called a \textbf{binomial distribution}.
\end{tcolorbox}


\section{Binomial Distribution Summary} % or \section{4.4 Binomial Distribution} if numbering manually

A random variable \( Y \) is said to have a \textbf{binomial distribution} based on \( n \) trials with success probability \( p \) if and only if:

\[
p(y) = \frac{n!}{y!(n - y)!} p^y (1 - p)^{n - y}, \quad y = 0, 1, 2, \dots, n \quad \text{and} \quad 0 \leq p \leq 1
\]

\medskip

The expected value and variance of a binomial random variable are:

\[
\mu = \mathbb{E}(Y) = np, \quad \sigma^2 = \mathrm{Var}(Y) = np(1 - p)
\]

\subsection{Visualizing the PMF of Binomial Distributions}

\textbf{R Output:}


\begin{center}
\textit{Image will be inserted here for Slide 11}
\end{center}

\textbf{Probability Mass Functions (PMFs) for increasing \(n\):}

\vspace{0.5em}

The following plots display the probability mass functions (PMFs) for a binomial distribution with \( p = \frac{1}{6} \) and increasing values of \( n \). As \( n \) increases, the binomial distribution begins to resemble a normal distribution.

\begin{itemize}
  \item Slide 12: Image will be inserted here
  \item Slide 13: Image will be inserted here
  \item Slide 14: Image will be inserted here
  \item Slide 15: Image will be inserted here
  \item Slide 16: Image will be inserted here
  \item Slide 17: Image will be inserted here
  \item Slide 18: Image will be inserted here
  \item Slide 19: Image will be inserted here
  \item Slide 20: Image will be inserted here
  \item Slide 21: Image will be inserted here
  \item Slide 22: Image will be inserted here
  \item Slide 23: Image will be inserted here
\end{itemize}

\section{Sampling Distribution of a Sample Proportion and the Normal Approximation}

Draw a \textit{Simple Random Sample (SRS)} of size \( n \) from a large population that contains proportion \( p \) of “successes”. Let \( \hat{p} \) be the \textbf{\textit{sample proportion}} of successes:

\[
\hat{p} = \frac{\text{number of successes in the sample}}{n}
\]

Then:

\begin{itemize}
  \item The \textbf{mean} of the sampling distribution of \( \hat{p} \) is \( p \).
  \item The \textbf{standard deviation} of the sampling distribution is \( \sqrt{ \frac{p(1 - p)}{n} } \).
\end{itemize}


\begin{center}
\textit{Image will be inserted here (Slide 24)}
\end{center}

According to the Central Limit Theorem (CLT), the sampling distribution of a sample proportion becomes approximately normal as the sample size increases.

That is:
\[
\hat{p} \sim \mathcal{N}\left(p, \sqrt{\frac{p(1 - p)}{n}}\right)
\]

This approximation is most accurate when both \( np \geq 10 \) and \( n(1 - p) \geq 10 \).

These are called the \textbf{success-failure conditions}.

\medskip

\textit{Key Point:} When the success-failure conditions are met, the normal approximation to the sampling distribution of \( \hat{p} \) can be used for probability calculations.
\subsection*{Conditions for Using the Normal Approximation}

\vspace{0.5em}

Suppose \( X \sim \text{Binomial}(n, p) \). Then:

\[
\mu = np, \quad \sigma^2 = np(1 - p)
\]

\medskip

\textbf{Binomial probabilities can be approximated by the normal distribution:}
\[
X \approx \mathcal{N}(np, \, np(1 - p))
\]

This approximation is \textit{useful for large \( n \)} and valid under the following conditions:

\begin{tcolorbox}[title=Standard Conditions,
  colback=blue!5, 
  colframe=blue!50!black, 
  coltitle=black,
  colbacktitle=blue!20, % ← LIGHTER blue for the title bar
  fonttitle=\bfseries,
  sharp corners=south,
  boxrule=0.5pt,
  enhanced,
  width=\textwidth,
  breakable]
The binomial setting holds (i.e., independent trials, fixed \( n \), same probability \( p \)) and

\[
np \geq 10 \quad \text{and} \quad np(1 - p) \geq 10
\]
\end{tcolorbox}

\vspace{0.5em}

Alternatively, a more conservative criterion for using the normal approximation is:
\[
n > 9 \cdot \left( \frac{\max(p, \, 1 - p)}{\min(p, \, 1 - p)} \right)
\]

\medskip

These ensure that the binomial distribution is sufficiently symmetric and smooth to approximate with the normal distribution.

\medskip
\section{Sampling Distribution of $\hat{p}$}
\subsection*{Bernoulli Distribution (Binomial with $n = 1$)}

\[
X_i =
\begin{cases}
1 & \text{if the $i$-th roll is a six} \\
0 & \text{otherwise}
\end{cases}
\]

\[
\mu = \mathbb{E}(X_i) = p, \quad \sigma^2 = \mathrm{Var}(X_i) = p(1 - p)
\]

Let $\hat{p}$ be our estimate of $p$. Note that $\hat{p} = \frac{1}{n} \sum_{i=1}^{n} X_i = \bar{X}$.
Let $\hat{p} = \frac{\text{\# successes } (X)}{\text{sample size } (n)}$

Recall that for $X \sim \text{Binomial}(n, p)$:
\[
X \overset{\cdot}{\sim} \mathcal{N}(np, np(1 - p))
\]

Let $\hat{p} = \frac{X}{n}$

\paragraph*{Mean of $\hat{p}$:}

\[
\mathbb{E}(\hat{p}) = \mathbb{E} \left( \frac{X}{n} \right) = \frac{1}{n} \cdot \mathbb{E}(X) = \frac{1}{n} \cdot np = p
\]

\paragraph*{Variance of $\hat{p}$:}

\[
\mathrm{Var}(\hat{p}) = \mathrm{Var} \left( \frac{X}{n} \right) = \frac{1}{n^2} \cdot \mathrm{Var}(X) = \frac{1}{n^2} \cdot np(1 - p) = \frac{p(1 - p)}{n}
\]

By the Central Limit Theorem (CLT), for sufficiently large $n$:
\[
\hat{p} \sim \mathcal{N} \left( p, \frac{p(1 - p)}{n} \right)
\]

\paragraph*{Standardization of $\hat{p}$:}
\[
Z = \frac{\hat{p} - p}{\sqrt{ \frac{p(1 - p)}{n} }}
\]

If $n$ is large, then by the Central Limit Theorem:
\[
\bar{X} \approx \mathcal{N} \left( \mu, \frac{\sigma}{\sqrt{n}} \right)
\quad \Rightarrow \quad
\hat{p} \sim \mathcal{N} \left( p, \sqrt{\frac{p(1 - p)}{n}} \right)
\]
\subsection*{Example: Normal Approximation for Proportions}

\textbf{Problem:}

In the last election, a state representative received 52\% of the votes cast. One year after the election, the representative organized a survey that asked a random sample of 300 people whether they would vote for him in the next election. If we assume that his popularity has not changed, what is the probability that more than half the sample would vote for him?

\vspace{1em}
\subsubsection*{Solution 1 (using Normal Approximation)}

We want to determine the probability that the sample proportion is greater than 50\%. In other words, we want to find $P(\hat{p} > 0.50)$.

We know that the sample proportion $\hat{p}$ is roughly Normally distributed with mean $p = 0.52$ and standard deviation
\[
\sqrt{p(1 - p)/n} = \sqrt{(0.52)(0.48)/300} = 0.0288.
\]
Thus, we calculate
\[
P(\hat{p} > 0.50) = P\left( \frac{\hat{p} - p}{\sqrt{p(1-p)/n}} > \frac{0.50 - 0.52}{0.0288} \right)
\]
\[
= P(Z > -0.69) = 1 - P(Z < -0.69) \quad \text{(Z is symmetric)}
\]
\[
= P(Z > -0.69) = 1 - P(Z > 0.69)
\]
\[
= 1 - 0.2451 = 0.7549.
\]

If we assume that the level of support remains at 52\%, the probability that more than half the sample of 300 people would vote for the representative is 0.7549.
\paragraph*{R code (Normal approximation)}

Just type in the following:

\begin{verbatim}
1 - pnorm(0.50, mean = 0.52, sd = 0.0288)
## [1] 0.7562982
\end{verbatim}

Recall that, \texttt{pnorm} will give you the area to the left of 0.50, for a Normal distribution with mean 0.52 and standard deviation 0.0288.
\subsubsection*{Solution 2 (using Binomial)}

We want to determine the probability that the sample proportion is greater than 50\%. In other words, we want to find $P(\hat{p} > 0.50)$. We know that 
$n = 300$ and $p = 0.52$. \\
Thus, we calculate
\begin{align*}
P(\hat{p} > 0.50) &= P\left(\frac{\sum_{i=1}^{n} x_i}{n} > 0.50\right) \\
&= P\left(\sum_{i=1}^{300} x_i > 150\right) \\
&= 1 - P\left(\sum_{i=1}^{300} x_i \leq 150\right) \\
&\text{(it can be shown that } Y = \sum_{i=1}^{300} x_i \text{ has a Binomial distribution with} \\
&n = 300 \text{ and } p = 0.52\text{)} \\
&= 1 - F_Y(150)
\end{align*}

\paragraph*{R code (using Binomial distribution )}

Just type in the following:

\begin{verbatim}
1- pbinom(150, size = 300, prob = 0.52);
## [1] 0.7375949
\end{verbatim}

Recall that, \texttt{pbinom} will give you the CDF at 150, for a Binomial distribution with $n = 300$ and $p = 0.52$.

\subsection*{Solution 3 (using continuity correction)}


We have that $n = 300$ and $p = 0.52$.
Thus, we calculate
\begin{align*}
P(\hat{p} > 0.50) &= P\left( \frac{\sum_{i=1}^{n} x_i}{n} > 0.50 \right) \\
&= P\left( \sum_{i=1}^{300} x_i > 150 \right) \\
&= 1 - P\left( \sum_{i=1}^{300} x_i \leq 150 \right) \\
&\text{(it can be shown that } Y = \sum_{i=1}^{300} x_i \text{ has a Binomial distribution with} \\
&n = 300 \text{ and } p = 0.52\text{)}. \\
&\approx 1 - P\left( \sum_{i=1}^{300} x_i \leq 150.5 \right) \quad \text{(continuity correction)} \\
&= 1 - P\left( \frac{\sum_{i=1}^{300} x_i}{n} \leq \frac{150.5}{300} \right) \\
&= 1 - P(\hat{p} \leq 0.5017) \\
&= 1 - P\left( Z \leq -0.6354 \right) \quad \text{(Why?)}
\end{align*}
\paragraph*{R code (Normal approximation with continuity correction)}

Just type in the following:

\begin{verbatim}
1 - pnorm(0.5017, mean = 0.52, sd = 0.0288)
## [1] 0.7374216
\end{verbatim}

Recall that, \texttt{pnorm} will give you the area to the left of 0.5017, for a Normal distribution with mean 0.52 and standard deviation 0.0288.

\section{Continuity Correction}

Suppose that $Y$ has a Binomial distribution with $n = 20$ and $p = 0.4$. We will find the exact probabilities that $Y \leq y$ and compare these to the corresponding values found by using two Normal approximations. One of them, when $X$ is Normally distributed with $\mu_X = np$ and $\sigma_X = \sqrt{np(1 - p)}$.
\vspace{1em}
The other one, $W$, a shifted version of $X$.

\vspace{1em}

For example,
\[
P(Y \leq 8) = 0.5955987
\]

As previously stated, we can think of $Y$ as having approximately the same distribution as $X$.
\[
P(Y \leq 8) \approx P(X \leq 8)
= P\left[ \frac{X - np}{\sqrt{np(1 - p)}} \leq \frac{8 - 8}{\sqrt{20(0.4)(0.6)}} \right]
= P(Z \leq 0) = 0.5
\]

\vspace{1em}

\[
P(Y \leq 8) \approx P(W \leq 8.5)
= P\left[ \frac{W - np}{\sqrt{np(1 - p)}} \leq \frac{8.5 - 8}{\sqrt{20(0.4)(0.6)}} \right]
= P(Z \leq 0.2282) = 0.5902615
\]
% Images from slides 37 and 38 will be inserted here.
{\noindent\textbf{Example: Normal Approximation with Continuity Correction (Exact Probability)}\\


Fifty-one percent of adults in the U. S. whose New Year’s resolution was to exercise more achieved their resolution. You randomly select 65 adults in the U. S. whose resolution was to exercise more and ask each if he or she achieved that resolution. What is the probability that exactly forty of them respond yes?\\

We are given that $p = 0.51$, $n = 65$, and we want to find $P(X = 40)$ where $X \sim Binomial(n = 65, p = 0.51)$.\\

\textbf{Step 1: Use Normal Approximation}
\vspace{0.5 em}
We use normal approximation to the binomial. First, compute the mean and standard deviation:
\[
\mu = np = 65 \times 0.51 = 33.15, \qquad \sigma^2 = np(1 - p) = 65 \times 0.51 \times 0.49 = 16.485, \quad \sigma = \sqrt{16.485} \approx 4.06
\]

We apply continuity correction:
\[
P(X = 40) = P(39.5 \leq X \leq 40.5)
\]

\[
= P\left(\frac{39.5 - 33.15}{4.06} \leq Z \leq \frac{40.5 - 33.15}{4.06}\right) = P(1.56 \leq Z \leq 1.81)
\]

From the standard normal table:
\[
= P(Z \leq 1.81) - P(Z \leq 1.56) = 0.0594 - 0.0352 = 0.0242
\]

So the approximate probability is:
\[
P(X = 40) \approx 0.0242
\]

\vspace{1em}
\begin{frame}{Normal Approximation to Binomial}

Let $X = \sum_{i=1}^{n} Y_i$ where $Y_1, Y_2, \ldots, Y_n$ are iid Bernoulli random variables. Note that $X = n\hat{p}$.

\begin{enumerate}
    \item $n\hat{p}$ is approximately Normally distributed provided that $np \geq 10$ and $n(1 - p) \geq 10$.

    \item Another criterion is that the Normal approximation is adequate if
    \[
    n > 9 \left( \frac{\text{larger of $p$ and $q$}}{\text{smaller of $p$ and $q$}} \right)
    \]

    \item The expected value: $E(\hat{p}) = np$.

    \item The variance: $V(\hat{p}) = np(1 - p) = npq$.
\end{enumerate}

\end{frame}

\vspace{1\baselineskip}  % Adds vertical space after the frame





