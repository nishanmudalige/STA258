%\pagenumbering{arabic}
%\setcounter{page}{1}
\setcounter{chapter}{11}
\chapter{One Sample Hypothesis Test on a Proportion and Variance}
%\index{Introduction}
%\label{sec.matrix}
%start relabeling as 2.1 etc
%\pagestyle{myheadings}  \markboth{\ref{sec.matrix}.
%\titleref{sec.matrix}}{}
%\setcounter{equation}{0}

Inferential statistics is a powerful method for statistical analysis, because it allows people to analyze a lot parameters. Similarly to confidence interval, testing hypothesis can be applied to proportion and variance as well. Also, we use the exact same structure for one sample hypothesis test on a proportion and variance.

\section{One Sample Hypothesis Test on a Proportion}

Suppose we have assume the proportion of a criteria from a population $p$ is equal to our parameter $p_0$ (null hypothesis $H_0: p = p_0$). While, the question is: how do we know whether our assumption is correct or not? We need to use testing hypothesis on proportion to verify. \\

\textbf{Step 1. Stating the Structure of Testing Hypothesis}

First of all, let's proceed with a table to see all the cases:

\begin{center}
\begin{figure}[H]
\centering
\begin{tabular}{ c c c }
Cases & Null Hypothesis & Alternative Hypothesis \\
     1	   & $H_0: p = p_0$ & $H_a: p > p_0$ \\
     2	   & $H_0: p = p_0$ & $H_a: p < p_0$ \\
     3    & $H_0: p = p_0$ & $H_a: p \neq p_0$ \\
\end{tabular}
\caption{All possible cases of one sample hypothesis test on a proportion ($p$ represents the actual proportion of a population)}
\end{figure}
\end{center}
\vspace{-0.75cm}
We are not going to proceed with all three cases in a single question. You need to be able to identify which case of testing hypothesis are going to be applied from question.\\

\textbf{Step 2. Computing Test Statistics}

After that we need to compute our test statistics, as the following definition provides:

\begin{definition}[Test statistics of one sample hypothesis test on a proportion]
The test statistics of one sample hypothesis test on a proportion is given by: \[ Z^* = \frac{\hat{p} - p_{\text{\scriptsize$0$}}}{\sqrt{\frac{p_{\text{\scriptsize$0$}}(1-p_{\text{\scriptsize$0$}})}{n}}}.\]
In this case, $n$ means the sample size, $\hat{p}$ is the parameter of the proportion of the population, which is calculated by $\hat{p} = \frac{\text{number of successes in the sample}}{n}$. Also, the reference distribution is standard normal distribution: $N(0,1)$.
\end{definition}

Note that be careful while you are computing the test statistics, because it directly affects the final answer.\\

\textbf{Step 3. Finding the $p$ - value}

i. When is structure of testing hypothesis is $H_0: p = p_0$, $H_a: p > p_0$:

\pgfmathdeclarefunction{gauss}{2}{\pgfmathparse{1/(#2*sqrt(2*pi))*exp(-((x-#1)^2)/(2*#2^2))}%
}
\begin{figure}[h]
\begin{center}
\begin{tikzpicture}
\begin{axis}[no markers, domain=0:10, samples=100,
axis lines*=left,
height=6cm, width=10cm,
xticklabels={}, ytick=\empty,
enlargelimits=false, clip=false, axis on top,]
\addplot [fill=blue!20, draw=none, domain=-3:3] {gauss(0,1)} \closedcycle;
\addplot [fill=blue!20, draw=none, domain=-3:-2] {gauss(0,1)} \closedcycle;
\addplot [fill=blue, draw=none, domain=2:3] {gauss(0,1)} \closedcycle;
\addplot [fill=blue!20, draw=none, domain=-2:-1] {gauss(0,1)} \closedcycle;
\addplot [fill=blue, draw=none, domain=1:2] {gauss(0,1)} \closedcycle;
\node at (axis cs:2,0.02) [below, yshift=-3.5pt] {\textbf{p-value}};
\end{axis}
\end{tikzpicture}
\end{center}
\caption{An illustration of p-value when $H_a: p > p_0$ for one hypothesis test on a proportion.}
\end{figure}

In case (i), calculating $P[Z > Z_*]$ as your p-value. Then, comparing with significant level: $\alpha$.

ii. When is structure of testing hypothesis is $H_0: p = p_0$, $H_a: p < p_0$:

\pgfmathdeclarefunction{m_gauss}{2}{\pgfmathparse{1/(#2*sqrt(2*pi))*exp(-((x-#1)^2)/(2*#2^2))}%
}
\begin{figure}[h]
\begin{center}
\begin{tikzpicture}
\begin{axis}[no markers, domain=0:10, samples=100,
axis lines*=left,
height=6cm, width=10cm,
xticklabels={}, ytick=\empty,
enlargelimits=false, clip=false, axis on top,]
\addplot [fill=blue, draw=none, domain=-3:3] {gauss(0,1)} \closedcycle;
\addplot [fill=blue, draw=none, domain=-3:-2] {gauss(0,1)} \closedcycle;
\addplot [fill=blue!20, draw=none, domain=2:3] {gauss(0,1)} \closedcycle;
\addplot [fill=blue, draw=none, domain=-2:-1] {gauss(0,1)} \closedcycle;
\addplot [fill=blue!20, draw=none, domain=1:2] {gauss(0,1)} \closedcycle;
\end{axis}
\end{tikzpicture}
\end{center}
\caption{An illustration of p-value when $H_a: p < p_0$ for one hypothesis test on a proportion.}
\end{figure}

In case (ii), calculating $P[Z < Z_*]$ as your p-value. Then, comparing with significant level: $\alpha$.

iii. When is structure of testing hypothesis is $H_0: p = p_0$, $H_a: p \neq p_0$:

\pgfmathdeclarefunction{k_gauss}{2}{\pgfmathparse{1/(#2*sqrt(2*pi))*exp(-((x-#1)^2)/(2*#2^2))}%
}
\begin{figure}[h]
\begin{center}
\begin{tikzpicture}
\begin{axis}[no markers, domain=0:10, samples=100,
axis lines*=left,
height=6cm, width=10cm,
xticklabels={}, ytick=\empty,
enlargelimits=false, clip=false, axis on top,]
\addplot [fill=blue!20, draw=none, domain=-3:3] {gauss(0,1)} \closedcycle;
\addplot [fill=blue, draw=none, domain=-3:-2] {gauss(0,1)} \closedcycle;
\addplot [fill=blue, draw=none, domain=2:3] {gauss(0,1)} \closedcycle;
\addplot [fill=blue, draw=none, domain=-2:-1] {gauss(0,1)} \closedcycle;
\addplot [fill=blue, draw=none, domain=1:2] {gauss(0,1)} \closedcycle;
\end{axis}
\end{tikzpicture}
\end{center}
\caption{An illustration of p-value when $H_a: p \neq p_0$ for one hypothesis test on a proportion.}
\end{figure}

In case (iii), calculating $2\cdot P[Z > |Z_*|]$ as your p-value. Then, comparing with significant level: $\alpha$.\\

\textbf{Step 4: Comparing P-value with $\alpha$-level}

If p-value is less than $\alpha$-level, then we reject the null hypothesis ($H_0$) and accept the alternative hypothesis ($H_a$). Otherwise, If p-value is greater than $\alpha$-level, then we do not reject the null hypothesis ($H_0$) and reject the alternative hypothesis ($H_a$).\\

\textbf{Step 5: Final Conclusion}
If we reject the null hypothesis, then we conclude that: there is sufficient evidence to reject the null hypothesis. If we do not reject the null hypothesis, then we conclude that: there is insufficient evidence to reject the null hypothesis.\\

\textbf{Conditions on One Sample Test Hypothesis on a Proportion}
\begin{itemize}
	\item 1. Random sample;
	\item 2. Independent sample: each observations are independent to others;
	\item 3. Sufficient sample.
\end{itemize}









